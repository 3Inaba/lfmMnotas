\usepackage[utf8]{inputenc}
\usepackage[english]{babel}
%\usepackage[paper=a4paper, margin=1 in, top=1 in, bottom=1 in]{geometry}
\usepackage[
    paperwidth=7in,
    paperheight=10in,
    inner=1in,
    outer=0.75in,
    top=0.8in,
    bottom=0.8in,
    %headsep=0.2in,
    headheight=14pt,
]{geometry}

%---------------------------------------------------------------
% Símbolos y notaciones. ---------------------------------------
%---------------------------------------------------------------
\usepackage{amsmath, amssymb, amsthm}
\usepackage{mathrsfs}
\usepackage{systeme}
\usepackage{etoolbox}
\usepackage{mathtools}
%---------------------------------------------------------------
\newcommand{\cond}{%
	\mathbin{\raisebox{0pt}{\makebox[1.5em]{%
			\setlength{\unitlength}{0.5em}%
			\linethickness{0.3pt}%
			\begin{picture}(-0.1,0.1)
				\polygon*(0.5,0.9)(0.5,0.1)(1,0.5)
				\polygon*(0.5,0.560)(0.5,0.440)(-1.25,0.440)(-1.25,0.560)
			\end{picture}%
	}}}%
}
\newcommand{\bicond}{%
	\mathbin{\raisebox{0pt}{\makebox[1.5em]{%
			\setlength{\unitlength}{0.5em}%
			\linethickness{0.3pt}%
			\begin{picture}(-0.1,0.1)
				\polygon*(0.85,0.9)(0.85,0.1)(1.35,0.5)
				\polygon*(-1.35,0.5)(-0.85,0.1)(-0.85,0.9)
				\polygon*(-0.85,0.560)(-0.85,0.440)(0.85,0.440)(0.85,0.560)
			\end{picture}%
	}}}%
}

\newcommand{\tl}[1]{\text{#1}} %texto

\newcommand{\fbf}{\textsl{f.b.f.}\;}
\newcommand{\fbfs}{\textsl{f.b.f.s}\;}
\newcommand{\raz}{\vdash}

\newcommand{\uclass}{{\mathcal{V}}}
\newcommand{\nat}{\mathbb{N}}
%\newcommand{\nat}{{\boldsymbol{\omega}}}
\newcommand{\intg}{\mathbb{Z}}
\newcommand{\rat}{\mathbb{Q}}
\newcommand{\real}{\mathbb{R}}
\newcommand{\complex}{\mathbb{C}}

\newcommand{\fdef}[3]{#1 : #2 \longrightarrow #3}
\newcommand{\suc}[1]{#1^{+}}
\newcommand{\Sucf}{\mathrm{S}}
\newcommand{\sucf}[1]{\mathrm{S}(#1)}

\newcommand{\biota}{\boldsymbol{\iota}}

\newcommand{\id}{\mathrm{Id}}

\newcommand{\fdefc}[5]{
	\raisebox{-5pt}{
		\(
		\begin{array}{@{}l@{\,}l@{}}
			#1 : & #2 \longrightarrow #3 \\[-5pt]
			& \scalebox{0.75}{$#4 \mapsto #5$}
		\end{array}
		\)
	}
}
\newcommand{\idfunc}[2]{\fdefc{\id}{#1}{#1}{#2}{#2}}

\newcommand{\difsim}{%
	\raisebox{0.2pt}{\makebox[0.77778em]{%
			\setlength{\unitlength}{0.5em}%
			\linethickness{0.3pt}%
			\begin{picture}(-0.1,0.1)
				\polygon(-0.5,0)(0.5,0)(0,0.8660)
			\end{picture}%
	    }
    }%
}

\newcommand{\equip}{\sim}
\newcommand{\isom}{\cong}
\makeatletter
\patchcmd{\endproof}
  {\popQED}
  {\vspace{-0.5cm}\par\pushQED{\qed}\popQED}
  {}{}
\makeatother
\renewcommand{\qedsymbol}{Q.E.D.}

\newcommand{\pp}{\mathrm{p}}
\newcommand{\pP}{\mathrm{P}}
\newcommand{\pps}{\mathrm{p}}
\newcommand{\pPs}{\boldsymbol{\mathrm{P}}}
\newcommand{\pqs}{\mathrm{q}}
\newcommand{\pQs}{\boldsymbol{\mathrm{Q}}}
\newcommand{\prs}{\mathrm{r}}
\newcommand{\pRs}{\boldsymbol{\mathrm{R}}}

%---------------------------------------------------------------
% Fuentes. -----------------------------------------------------
%---------------------------------------------------------------
\usepackage{helvet}
\usepackage{times}
\usepackage{bookman}
\usepackage{times}
\usepackage{lmodern}
\usepackage{calligra}

%---------------------------------------------------------------
% Esconder hipervínculos. --------------------------------------
%---------------------------------------------------------------
\usepackage[hidelinks]{hyperref}

%---------------------------------------------------------------
% Utilidades de texto
\usepackage{setspace}
\usepackage{enumerate}
\usepackage{enumitem}
\usepackage{varwidth}

%---------------------------------------------------------------
% Gráficos, tablas y portada. ----------------------------------
%---------------------------------------------------------------
\usepackage{tikz}
\usepackage{graphicx}
\usepackage{caption}
\usepackage{parskip}
\usepackage{ragged2e}
\usepackage{background}
\usepackage{pict2e}
%---------------------------------------------------------------
\captionsetup[table]{name=Tabla, labelfont=sl, font=it}

%---------------------------------------------------------------
% Estilos, entornos y organización de secciones. ---------------
%---------------------------------------------------------------
\usepackage{subfiles}
\usepackage{fancyhdr}
\usepackage{titlesec}
%---------------------------------------------------------------
\titleformat{\chapter}[display]
{\fontsize{30}{0}\selectfont\centering\bfseries}
{\fontsize{40}{0}\selectfont\thechapter\vspace{-1cm}}
{1em}
{}

\titleformat{\section}
{\normalfont\Large\bfseries}
{{\normalfont\S}\thesection \hspace{-0.5cm}}
{1em}
{}

\titleformat{\subsection}
{\itshape\bfseries\large}
{\hspace{-0.5cm}}
{0.8em}
{}

\theoremstyle{plain}
\newtheorem{axiom}{Axiom}

\newtheoremstyle{definicion}
{3pt} % Space above
{3pt} % Space below
{} % Body font
{} % Indent amount
{\itshape\bfseries \color{black}} % Theorem head font
{:} % Punctuation after theorem head
{.5em} % Space after theorem head
{ } % Theorem head spec
\theoremstyle{definicion}
\newtheorem*{dfn}{{Definition}}

\newtheoremstyle{them}
{3pt}{3pt}
{\itshape}{}
{\scshape \bfseries \color{black}}{:}
{.5em}{ }
\theoremstyle{them}
\newtheorem{thm}{{Theorem}}[chapter]
\newtheorem{prop}{Proposition}[chapter]
\newtheorem{lema}{Lema}[chapter]

%---------------------------------------------------------------
% Colorsitos. --------------------------------------------------
%---------------------------------------------------------------
\usepackage[dvipsnames]{xcolor}
%---------------------------------------------------------------
\definecolor{mazul}{HTML}{1F2A44}
\definecolor{sazul}{HTML}{4B6587}
\definecolor{hazul}{HTML}{AFCDE7}
\definecolor{sgazul}{HTML}{6C7A89}
\definecolor{gwhite}{HTML}{F1F6FA}

%---------------------------------------------------------------
% Tabla de contenidos. -----------------------------------------
%---------------------------------------------------------------
\usepackage[nottoc]{tocbibind}
\iffalse
\addto\captionsenglish{
    \renewcommand{\contentsname}{\normalfont \color{blue} \calligra Tabla de contenidos}
}
\fi

%---------------------------------------------------------------
% Texto en cajas por si acaso. ---------------------------------
%---------------------------------------------------------------
\usepackage{tcolorbox}
\tcbuselibrary{theorems}
\tcbuselibrary{breakable}
\tcbuselibrary{skins}