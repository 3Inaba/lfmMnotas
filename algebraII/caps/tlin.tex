\chapter{Transformaciones lineales}
\section{Transformaciones lineales}
\subsection{Primeras definiciones y resultados}
\begin{dfn}[Transformación lineal]
	Sean $V, W$ \Fevs{F}. La relación $$\tl{T}{V}{W}{\alpha}$$ se dice \textbf{transformación lineal} si es una función tal que $\forall \beta, \gamma \in V$ y $\forall c \in F$
	$$ T((c\beta + \gamma)) = cT(\beta) + T(\gamma) .$$
\end{dfn}

\begin{ejem}
	Si $V = \{f \in \mathbb{R}^\mathbb{R} \mid f \text{ es función polinomial}\}$. La función $\fdefc{T}{V}{V}{f}{f'}$ es una \tlc.
\end{ejem}

\begin{teo}
	Si $V$ es un \Fev{F} de dimensión finita $n$ y sea $\base{} = \cv{\alpha}{n}$ una base ordenada de $V$. Sea $W$ un \Fev{F} y $\beta_1, \ldots, \beta_n \in W$. Entonces existe una única transformación lineal $\tl{T}{V}{W}{\alpha}$ tal que
	$$ T(\alpha_j) = \beta_j \; \forall j \in \inte{1}{n}. $$
	
	\begin{proof}
		Primero probamos el asignamiento total. Como \base{} es base de $V$, $\forall \alpha \in V$ existen únicos $c_{\alpha_1}, \ldots, c_{\alpha_n} \in F$ tales que $\alpha = \sum\limits_{i=1}^{n} c_{\alpha_i} \alpha_i$. Se tiene, con $T(\alpha_i) = \beta_i$, que $T(\alpha) = T\left(\sum\limits_{i=1}^{n} c_{\alpha_i} \alpha_i\right) = \sum\limits_{i=1}^{n} c_{\alpha_i} T(\alpha_i) = \sum\limits_{i=1}^{n} c_{\alpha_i} \beta_i \in W$. Así la relación $\tl{T}{V}{W}{\alpha}$ es de asignamiento total total.\\
		
		Ahora, para el asignamiento único. Bajo la forma en que se definió la relación, como \base{} es base de $V$, $\alpha$ admite una única combinación lineal de los elementos de la base dada, luego $\tl{T}{V}{W}{\alpha}$ es de asignamiento único.\\
		
		Entonces $\tl{T}{V}{W}{\alpha}$ es una función.\\
		
		$\tl{T}{V}{W}{\alpha}$ es lineal. En efecto, sean $\gamma, \delta \in V$, $d \in F$, se tiene que
		\begin{align*}
			T(d\gamma + \delta) &= T\left( d \sum\limits_{i=1}^{n} c_{\gamma_i} \alpha_i + \sum\limits_{i=1}^{n} c_{\delta_i} \alpha_i \right)\\
			&= T\left(\sum\limits_{i=1}^{n} [dc_{\gamma_i} + c_{\delta_i}] \alpha_i\right) \\
			&= \sum\limits_{i=1}^{n}[dc_{\gamma_i} + c_{\delta_i}] T(\alpha_i) \\
			&= d\sum\limits_{i=1}^{n} c_{\gamma_i}T(\alpha_i) + \sum\limits_{i=1}^{n} c_{\delta_i} T(\alpha_i) \\
			&= dT(\gamma) + T(\delta).
		\end{align*}
		
		Como $\gamma$ y $\delta$ son arbitrarios, $\tl{T}{V}{W}{\alpha}$ es transformación lineal.\\
		
		Suponga que $U(\alpha_j) = T(a_j) \; \forall j \in \inte{1}{n}$, con $\tl{U}{V}{W}{\alpha}$ transformación lineal. Ahora bien, $\forall \gamma \in V$,
		
		\begin{align*}
			\gamma &= \sum\limits_{i=1}^{n} c_{\gamma_i} \alpha_i \\
			U(\gamma) &= U\left(\sum\limits_{i=1}^{n} c_{\gamma_i} \alpha_i\right) \\
			&= \sum\limits_{i=1}^{n} c_{\gamma_i} U(\alpha_i) \\
			&= \sum\limits_{i=1}^{n} c_{\gamma_i} T(\alpha_i) \\
			&= T(\sum\limits_{i=1}^{n} c_{\gamma_i} \alpha_i) \\
			&= T(\gamma).
		\end{align*}
	\end{proof}
\end{teo}

% Ejemplo comentado.
\iffalse
\begin{ejem}
	Sean $V = \conmodr{3} \times \conmodr{3}$, $W = \conmodr{3} \times \conmodr{3} \times \conmodr{3}$ y $\fdefc{T}{V}{W}{(\clamodr{x}{5}, \clamodr{y}{5})}{(\clamodr{x}{5} + \clamodr{y}{5}, \clamodr{-x}{5}, \clamodr{y}{5})}$.
	
	$V$ y $W$ son $\Fevs{\conmodr{3}}$.\\
	
	Sean $\alpha_1 = (\clamodr{2}{5}, \clamodr{4}{5})$ y $\alpha_2 = (\clamodr{3}{5}, \clamodr{2}{5})$. $\{\alpha_1, \alpha_2\}$ es una base de $V$ (demuéstrelo), luego
	\begin{align*}
		T(\alpha_1) = T(\clamodr{2}{5}, \clamodr{4}{5}) &= (\clamodr{1}{5}, \clamodr{3}{5}, \clamodr{4}{5}) \text{ y } \\
		T(\alpha_2) = T(\clamodr{3}{5}, \clamodr{2}{5}) &= (\clamodr{0}{5}, \clamodr{2}{5}, \clamodr{2}{5}).
	\end{align*}
\end{ejem}
\fi

\begin{prop}
	Sean $V,W$ \Fevs{F}, $\tl{T}{V}{W}{\alpha}$ una transformación lineal. Entonces $T(V)$ es un subespacio de $W$.
	
	\begin{proof}
		Sean $\alpha_w, \beta_w \in T(v), c \in F$. En particular existen $\alpha_v, \beta_v \in V$ tales que $T(\alpha_v) = \alpha_w$ y $T(\beta_v) = \beta_w$, así $c\alpha_w + \beta_w \in T(v)$. Luego $T(V)$ es un \Fsev{F} de $W$.
	\end{proof}
\end{prop}

\begin{dfn}[Kernel de una transformación lineal]
	Si $V,W$ son \Fevs{F} y $\tl{T}{V}{W}{\alpha}$ es una transformación lineal \textit{(T.l.)} se define el \textbf{kernel} de la transformación lineal $\tl{T}{V}{F}{\alpha}$ (denotado como $\ker_T$) como
	$$ \ker_T = \{\alpha \in V \mid T(\alpha) = 0_W\} .$$
\end{dfn}

\begin{prop}
	Si $V,W$ son \Fevs{F}, $\tl{T}{V}{W}{\alpha}$ una \tlc. Entonces $\ker_T$ es un \Fsev{F} de $V$.
	
	\begin{proof}
		$0_V \in \ker_T$, puesto que $T(0_V) = 0_W$. Sean $\alpha, \beta \in \ker_T, c \in F$, en particular $T(\alpha) = T(\beta) = 0_W$. Ahora bien $T(c\alpha + \beta) = cT(\alpha) + T(\beta) = c0_W + 0_W = 0_W$. Por lo tanto como $c, \alpha, \beta$ son arbitrarios, $c\alpha + \beta \in \ker_T$. Luego $\ker_T$ es un \Fsev{F} de $V$.
	\end{proof}
\end{prop}

\begin{dfn}[Rango de una \tlc]
	Sean $V,W$ \Fevs{F}, $V$ de dimensión finita y $\tl{T}{V}{W}{\alpha}$ \tlc. Se define el \textbf{rango} de $\tl{T}{V}{W}{\alpha}$ (denotado como $\ran{T}$ como
	$$\nul{T} := \dim(T(V)).$$
\end{dfn}

\begin{dfn}[Nulidad de una \tlc]
	Sean $V,W$ \Fevs{F}, $V$ de dimensión finita y $\tl{T}{V}{W}{\alpha}$ \tlc. Se define la \textbf{nulidad} de $\tl{T}{V}{W}{\alpha}$ (denotada por $\nul{T}$) como
	$$\nul{T} := \dim(\ker_T).$$
\end{dfn}

\begin{teo} \label{teo:TeoremaDeLaDimensión}
	Sean $V,W$ \Fevs{F}, $V$ de dimensión finita, $\tl{T}{V}{W}{\alpha}$ una \tlc, entonces $\ran{T} + \nul{T} = \dim(V)$.
	
	\begin{proof}
		Si $V = \{0_V\}$, es claro.\\
		Suponga que $V \not= \{0_V\}$. Si $\ker_T = \{0_V\}$, entonces $\forall \alpha \in V \setminus \{0_V\}$, $T(\alpha) \not= 0_V$. Sea $\base{} = \cv{\alpha}{n}$ base de $V$. Veamos que $\base{_{T(V)}} = \{T(\alpha_1), \ldots, T(\alpha_n)\}$ es base de $T(v)$.\\
		En efecto, $\Legen{\base{_{T(V)}}} = T(V)$, pues sea $\gamma_W \in T(V)$, entonces existe $\gamma_V \in V$ tal que $T(\gamma_V) = \gamma_W$. Ahora bien, para $\gamma_V$ existen $c_1, \ldots, c_n \in F$ tales que $\sum\limits_{i=1}^{n} c_i \alpha_i = \gamma_V$. En particular $\gamma_W = T(\gamma_V) = T\left( \sum\limits_{i = 1}^{n} c_i \alpha_i \right)$, entonces como son elementos arbitrarios, $\Legen{\base{_{T(V)}}} = T(V)$.\\
		Suponga que existen $d_1, \ldots, d_n \in F$ no todos cero, tales que $\sum\limits_{i=1}^{n} d_i T(\alpha_i) = 0$, así $T\left( d_i \alpha_i \right) = 0$, de modo que $\sum\limits_{i=1}^{n} d_i \alpha_i \in \ker_T = \{0_V\}$. Luego $\sum\limits_{i=1}^{n} d_i \alpha_i = 0$. Así $d_i = 0 \; \forall i \in \inte{1}{n}$, luego $\base{_{T(V)}}$ es \li. Así $\base{_{T(V)}}$ es base de $T(V)$ y $\dim(T(V)) = n$, $\dim(\ker_T)$ y $\dim(V) = \dim(T(V)) + \dim(\ker_T)$.
		
		Se deja como ejercicio la demostración para los casos restantes:
		\begin{enumerate}[label=\roman*), leftmargin=2em]
			\item $\ker_T \not= 0_V$.
			\item $\ker_T = V$.
		\end{enumerate}
	\end{proof}
\end{teo}

\begin{dfn}[Rango de filas de $A$]
	El \textbf{rango de filas} de $A$ (denotado rango de filas $(A)$) es la dimensión del espacio de filas de $A$.
\end{dfn}

\begin{teo} \label{teo:ComprobarTL}
	Si $A \in \Mmnc{m}{n}{F}$, entonces $\text{rango de filas}(A) = \text{rango de columnas}(A)$.
	
	\begin{proof}
		Sea $\fdefc{T}{F^{n\times1}}{F^{m\times1}}{\alpha}{T(\alpha) = A\alpha}$ (se deja como ejercicio comprobar que $T$ es una \tlc).\\
		Cuando $A\alpha = \left[\begin{array}{c}
			0 \\
			\vdots \\
			0
		\end{array}\right]$ $\alpha$ es solución del sistema asociado a la matriz $A$. Entonces $$\ker_T = \left\{ \left(\begin{array}{c}
			\alpha_1 \\
			\vdots \\
			\alpha_n
		\end{array}\right) \in F^n \; \Biggr| \; \left(\begin{array}{c}
			\alpha_1 \\
			\vdots \\
			\alpha_n
		\end{array}\right) \in \mathcal{S}_{Ax} = 0_{F^m} \right\}$$
		y
		
		$$T(V) = \left\{\left(\begin{array}{c}
			\beta_1 \\
			\vdots \\
			\beta_n
		\end{array}\right) \in F^m \;\Biggr|\; Ax = \left(\begin{array}{c}
			\beta_1 \\
			\vdots \\
			\beta_n
		\end{array}\right) \right\}$$
		
		Si $A_1, \ldots, A_n$ son las columnas de $A$, entonces $A\alpha = \alpha_1A_1 + \cdots + \alpha_nA_n$, así $T(V) = \Legen{\{A_1, \ldots A_n\}}$. Es decir, $T(V)$ es el espacio de columnas de $A$. Por lo tanto $\ran{T} = \text{rango de columnas de } A$. Así, $\dim(\ker_T) + \text{rango de columnas de } A = n$.\\
		
		Ahora, recuerde que el espacio de soluciones tiene una base que consta de $n - r$ vectores donde $r$ es el rango de filas de $A$ y entonces, si $S$ es el espacio solución de $A$, 
		$$\dim(S) = \dim(\ker_T) = n - r,$$
		de donde $\dim(\ker_T) + r = n$. Entonces es evidente que 
		$$\text{rango de filas de }A = \text{rango de columnas de} A.$$
	\end{proof}
\end{teo}

\subsection{Álgebra de transformaciones lineales}

\begin{teo} \label{teo:EspacioDeTLEsUnEV}
	Si $V,W$ son \Fevs{F} y $\tl{T}{V}{W}{\alpha}$, $\tl{U}{V}{W}{\alpha}$ son \tlc, entonces $\fdefc{T+U}{V}{W}{\alpha}{[T+U](\alpha):=T(\alpha) + U(\alpha)}$ es una \tlc y si $c \in F$, se define la función $\fdefc{cT}{V}{W}{\alpha}{[cT](\alpha):=cT(\alpha)}$ es una \tlc.\\
	Entonces si $$\Legen{V,W}:=\left\{ \tl{T}{V}{W}{\alpha} \in W^V \mid \tl{T}{V}{W}{\alpha} \text{ es \tlc} \right\}$$ $(\Legen{V,W}, +, \cdot, F)$ es un \Fev{F}.
	
	\begin{proof}
		Ejercicio.
	\end{proof}
\end{teo}

\begin{teo} \label{teo:DimensionDeLVWMostrarAEsLI}
	Si $V$ es un \Fev{F} de dimensión finita $n$, $W$ es un $F$-espacio vectorial de dimensión finita $m$, entonces $\Legen{V,W}$ es de dimensión finita $mn$.
	
	\begin{proof}
		Sean $\base{_V} = \cv{\alpha}{n}$, $\base{_W} = \cv{\beta}{n}$ bases ordenadas de $V$ y $W$ respectivamente. $\forall (p,q) \in \inte{1}{m} \times \inte{1}{n}$ se define
		$$ \fdefc{T^{p,q}}{V}{W}{\alpha_i}{T^{p,q}(\alpha_i) := \left\{\begin{array}{c}
				0, \text{ si } i \not= q \\
				\beta_p, \text{ si } i = q
			\end{array}\right. = \delta_{iq} \beta{p} }$$
		de acuerdo con un teorema anterior existe una única transformación lineal de $V$ a $W$ que satisface estas condiciones. Se afirma que $$\mathscr{A} = \left\{T \in \Legen{V,W} \mid \tl{T}{V}{W}{\alpha} = \tl{T^{p,q}}{V}{W}{\alpha}\right\}$$ es una base de $\Legen{V,W}$.\\
		
		Sea $\tl{T}{V}{W}{\alpha}$. $\forall j \in \inte{1}{n}$, sea $A_{1j}, \ldots, A_{mj}$ las coordenadas del vector $T(\alpha_j)$ respecto a la base $\base{_W}$ es decir $T(\alpha_j) = \sum\limits_{p=1}^{m} A_{pj} \beta_p$.
		
		Sea $\tl{U}{V}{W}{\alpha}$, $\forall j \in \inte{1}{n}$ $U(\alpha_j) = \sum\limits_{p=1}^{m} \sum\limits_{q=1}^{n} A_{pq}T^{p,q}(\alpha_j) = \sum\limits_{p=1}^{m} \sum\limits_{q=1}^{n} A_{pq} \delta_{jq} \beta_p = \sum\limits_{p=1}^{m} A_{pj} \beta_p = T(\alpha_j)$, en consecuencia $U = T$.  Luego $\Legen{\mathscr{A}} = \Legen{V,W}$. Queda como ejercicio mostrar que $\mathscr{A}$ es \li.
	\end{proof}
\end{teo}

\begin{teo}
	Si $V,W,Z$ son \Fevs{F} y $\tl{T}{V}{W}{\alpha}$, $\tl{U}{W}{Z}{\alpha}$ son \tlc, entonces
	$$\fdefc{UT}{V}{Z}{\alpha}{[UT](\alpha):=U(T(\alpha))}\text{ es \tlc.}$$
	
	\begin{proof}
		$[UT](0_V) = U(T(0_V)) = U(0_W) = 0_Z$. Sean $c \in F, \alpha,\beta \in V$,
		\begin{align*}
			[UT](c\alpha + \beta) &= U(T(c\alpha + \beta)) \\
			&= U(cT(\alpha) + T(\beta)) \\
			&= cU(T(\alpha)) + U(T(\beta)) \\
			&= c[UT](\alpha) + [UT](\beta).
		\end{align*}
	\end{proof}
\end{teo}

\begin{dfn}[Operador lineal]
	Si $V$ es un \Fev{F}, un operador lineal se dice que es una transformación lineal $\tl{T}{V}{V}{\alpha}$ y se denota por
	$$\tl{T^n}{V}{V}{\alpha}$$
	al operador lineal compuesto $n$-veces, y se define
	$$\fdefc{T^0}{V}{V}{\alpha}{\alpha} = \fdefc{I}{V}{V}{\alpha}{\alpha}.$$
\end{dfn}

\begin{lema} \label{lema:OperacionesEnLVW}
	Si $V$ es un \Fev{F} y $\tl{U}{V}{V}{\alpha}$, $\tl{T_1}{V}{V}{\alpha}$, $\tl{T_2}{V}{V}{\alpha}$ operadores lineales, $c \in F$, $\fdefc{I}{V}{V}{\alpha}{\alpha}$, entonces
	
	\begin{enumerate}[label=\roman*), leftmargin=2em]
		\item $IU = UI = U$,
		\item $U(T_1 + T_2) = UT_1 + UT_2$,
		\item $(T_1 + T_2)U = T_1 U + T_2 U$,
		\item $c(UT_1) = (cU)T_1 = U(cT_1)$.
	\end{enumerate}
	
	\begin{proof}
		Ejercicio.
	\end{proof}
\end{lema}

\begin{dfn}[Transformación lineal invertible]
	Si $V,W$ son \Fevs{F} y $\tl{T}{V}{W}{\alpha}$ es una \tlc, se dice que $\tl{T}{V}{W}{\alpha}$ es \textbf{invertible} si existe $\tl{U}{W}{V}{\alpha}$ tal que $\fdefc{UT}{V}{V}{\alpha}{\alpha}$ y $\fdefc{TU}{W}{W}{\alpha}{\alpha}$.
\end{dfn}

\begin{teo}
	Si $V,W$ son \Fevs{F} y $\tl{T}{V}{W}{\alpha}$, si $\tl{T}{V}{W}{\alpha}$ es invertible entonces $\tl{T^{-1}}{W}{V}{\alpha}$ es \tlc.
	
	\begin{proof}
		Al ser $\tl{T}{V}{W}{\alpha}$ \tlc invertible, en particular, $\tl{T}{V}{W}{\alpha}$ es biyectiva. Luego, existe una única función $\tl{T^{-1}}{V}{W}{\alpha}$. Sean $c \in F, \alpha, \beta \in W$, entonces por ser $\tl{T}{V}{W}{\alpha}$ suprayectiva existen $\alpha_V, \beta_V \in V$ tal que $T(\alpha_V) = \alpha$ y $T(\beta_V) = \beta$. Así $T^{-1}(c\alpha + \beta) = T^{-1}(cT(\alpha_V) + T(\beta_V))$ pero $T$ es lineal, entonces $T^{-1}(T(c\alpha_V + \beta_V)) = I_V(c\alpha_V + \beta_V) = c\alpha_V + \beta_V = cT^{-1}(\alpha) + T^{-1}(\beta)$. Luego $T^{-1}$ es \tlc.
	\end{proof}
\end{teo}

\begin{teo} \label{teo:TNoSingularSSITMandaBasesABases}
	Sea $\tl{T}{V}{W}{\alpha}$ una \tlc. Entonces $T$ es no singular si y sólo si $T$ para todo conjunto $\cv{\alpha}{n} \subseteq V$ \li se tiene que $\{T(\alpha_1), \ldots, T(\alpha_n)\} \subseteq W$ es \li.
	
	\begin{proof}
		Ejercicio.
	\end{proof}
\end{teo}

\begin{teo} \label{teo:313}
	Si $V,W$ son \Fevs{F} de dimensión finita $n$, $\tl{T}{V}{W}{\alpha}$, las siguientes proposiciones son lógicamente equivalentes.
	\begin{enumerate}[label=\roman*), leftmargin=2em]
		\item $T$ es invertible,
		\item $T$ es no singular,
		\item $T$ es suprayectiva,
		\item Si $\base{} = \cv{\alpha}{n}$ es base de $V$, entonces $\{T(\alpha_1), \ldots, T(\alpha_n)\}$ es base de $W$.
	\end{enumerate}
	
	\begin{proof}
		i) $\to$ ii) Como $\tl{T}{V}{W}{\alpha}$ es invertible, en particular es inyectiva, luego $\ker_T = \{0_V\}$, luego $\tl{T}{V}{W}{\alpha}$ es no singular.\\
		
		ii) $\to$ iii) Como $\tl{T}{V}{W}{\alpha}$ es no singular, entonces $\nul{T} = 0$, así, $\ran{T} = n$, entonces $T(V) = W$, luego $\tl{T}{V}{W}{\alpha}$ es suprayectiva.\\
		
		iii) $\to$ iv) Suponga que $T$ es suprayectiva. Si $\cv{\alpha}{n}$ es cualquier base ordenada de $V$, entonces $\Legen{\{T(\alpha_1), \ldots, T(\alpha_n)\}} = W$, luego entonces si $\{T(\alpha_1), \ldots, T(\alpha_n)\}$ es \ld, como $\Legen{\{T(\alpha_1), \ldots, T(\alpha_n)\}} = W$, entonces podemos encontrar una base de la forma $\{T(\alpha_1), \ldots, T(\alpha_{i-1}), T(\alpha_{i+1}), \ldots, T(\alpha_n)\}$ con $\{T(\alpha_1), \ldots, T(\alpha_{i-1}), T(\alpha_{i+1}), \ldots, T(\alpha_n)\}$ \li, entonces tendriamos que $\dim(W) = n-l$, $n \geq l$. Por lo tanto $\{T(\alpha_1), \ldots, T(\alpha_n)\}$ es \li.
		
		Luego $\{T(\alpha_1), \ldots, T(\alpha_n)\}$ es base.\\
		
		iv) $\to$ i) Suponga que existe $\cv{\alpha}{n}$ base de $V$ tal que $\{T(\alpha_1), \ldots, T(\alpha_n)\}$ es base $W$. Como $\{T(\alpha_1), \ldots, T(\alpha_n)\}$ es base de $W$ entonces $\Legen{\{T(\alpha_1), \ldots, T(\alpha_n)\}} = W$. Así $\tl{T}{V}{W}{\alpha}$ es suprayectiva. Por hipótesis general, $\tl{T}{V}{W}{\alpha}$ es \tlc. Sea $\alpha \in \ker_T$, existen $c_1, \ldots, c_n \in F$ tales que $\sum\limits_{k=1}^{n} c_k \alpha_k$, luego $T \left( \sum\limits_{k=1}^{n} c_k \alpha_k \right) = T(\alpha) = 0_W $. Luego $\sum\limits_{k=1}^{n} c_k T(\alpha_k) = T(\alpha) = 0_W$.
		
		Luego, como $\{T(\alpha_1), \ldots, T(\alpha_n)\}$ es base de $W$, $c_k = 0_F$, $\forall k \in \inte{1}{n}$, i.e. $\{T(\alpha_k)\}_{k \in \inte{1}{n}}$ es \li
		
		Luego, $\ker_T = \{0_V\}$, así $\tl{T}{V}{W}{\alpha}$ es inyectiva. Por lo tanto se tiene lo deseado.\\
	\end{proof}
\end{teo}

% Ejemplo comentado
\iffalse
\begin{ejem}
	Lugar de trabajo: $\mathbb{C}^{3}$ como $\Fev{\mathbb{C}}$.
	
	Sean $\alpha_1 = (1,0,i), \alpha_2 = (0,1,1)$ y $\alpha_3 = (i,1,0)$.
\end{ejem}
\fi

\subsection{Isomorfismos}

\begin{dfn}[Morfismos]
	Si $V$ y $W$ son \Fsevs{F} y $\tl{T}{V}{W}{\alpha}$ una función entonces:
	\begin{enumerate}[label=\arabic*)]
		\item $T$ es un homomorfismo si es $\tlc$
		\item $T$ es un endomorfismo si es $\tlc$ y $W=V$.
		\item $T$ es un monomorfismo si es $\tlc$ inyectiva.
		\item $T$ es un epimorfismo si es $\tlc$ suprayectiva.
		\item $T$ es un isomorfismo si es $\tlc$ biyectiva.
		\item $T$ es un automorfismo si es $\tlc$ biyectiva y $W=V$.
	\end{enumerate}
\end{dfn}

Si existe $\tl{T}{V}{W}{\alpha}$ isomorfismo, entonces se dice que $V$ es \textbf{isomorfo} a $W$ y se denota por $V \eqsim W$.

\begin{teo}
	Todo \Fev{F} $V$ de dimensión $n$ es isomorfo a $F^n$.
	
	\begin{proof}
		Sea $V$ un \Fev{F} de dimensión finita $n \in \mathbb{N} \setminus \{0\}$, y $\base{} = \cv{\alpha}{n}$ base ordenada para $V$. Se define $\fdefc{T}{V}{F^n}{\alpha}{T(\alpha):=(x_1, \ldots, x_n)}$ donde $(x_1, \ldots, x_n)$ es el vector de coordenadas de $\alpha$ respecto a la base ordenada $\base{}$. $T$ es inyectiva y suprayectiva (¿por qué?). Luego $T$ es un isomorfismo.
	\end{proof}
\end{teo}

\subsection{Representación matricial de una transformación lineal}

Si $V$ es un \Fev{F} de dimensión finita $n$ y $W$ es un \Fev{F} de dimensión finita $m$, si $\base{_V} = \cv{\alpha}{n}$ es base ordenada de $V$ y $\base{_W} = \cv{\beta}{m}$ es base ordenada de $W$, $\tl{T}{V}{W}{\alpha}$ es una \tlc. Entonces la transformación lineal está determinada por la aplicación de los elementos de $\base{_V}$ de manera única como una combinación lineal
$$T(\alpha_j) = \sum\limits_{i=1}^{m} A_{ij}{\beta_i}$$
de los $\beta_i$, los elementos de $F$ $A_{1j}, \ldots, A_{mj}$ son las coordenadas de $T(\alpha_j)$ en la base $\base{_W}$.

A la función $\fdefc{A}{\inte{1}{m} \times \inte{1}{n}}{F}{(i,j)}{A_{ij}}$ se le llama la \textbf{matriz representante de la transformación lineal $\tl{T}{V}{W}{\alpha}$ respecto a las bases $\base{_V}$ y $\base{_W}$}.

\begin{teo}
	Sea $V$ un \Fev{F} de dimensión finita $n$ y $W$ un $F$-espacio vectorial de dimensión finita $m$. Sea $\base{_V}$ una base ordenada de $V$ y $\base{_W}$ una base ordenada de $W$. Para cada transformación lineal $\tl{T}{V}{W}{\alpha}$ hay una matriz $A \in \Mmnc{m}{n}{F}$ tal que
	$$[T(\alpha)]_{\base{_W}} = A[\alpha]_{\base{_V}}$$
	$\forall \alpha \in V$. Además, $\fdefc{\phi}{\Legen{V,W}}{\Mmnc{m}{n}{F}}{\tl{T}{V}{W}{\alpha}}{\phi\left(\tl{T}{V}{W}{\alpha}\right) = A_{\tl{T}{V}{W}{\alpha}}}$ es un homomorfismo.
	
	\begin{proof}
		Sea $\base{_V} = \cv{\alpha}{n}$ base de $V$ y $\base{_W} = \cv{\beta}{m}$. Considere $\forall j \in \inte{1}{n}$, $T(\alpha_j) \in W$, así, por ser una base, existen elementos del campo, únicos, $A_{1j}, \ldots, A_{mj} \in F$ tales que $T(\alpha_j) = \sum\limits_{i=1}^{m} A_{ij} \beta_i$.
		
		Ahora bien, note que $\forall j \in \inte{1}{n}$ $A_{1j}, \ldots, A_{mj}$ son las coordenadas de $T(\alpha_j)$ respecto a la base $\base{_W}$. Así
		
		$$A = \begin{bmatrix}
			A_{11} & \cdots & A_{1n} \\
			\vdots & \ddots & \vdots \\
			A_{m1} & \cdots & A_{mn} \\
		\end{bmatrix}$$
		
		está bien definida (existe y es única) y si $\begin{bmatrix}
			c_1 \\
			\vdots \\
			c_n
		\end{bmatrix}$ es la matriz de coordenadas del vector $\alpha$ en la base $\base{_V}$, entonces se tiene que $\alpha = \sum\limits_{j=1}^{n} c_j \alpha_j$, luego $T(\alpha) = \sum\limits_{j=1}^{n} c_j T(\alpha_j) = \sum\limits_{j=1}^{n} c_j \left(\sum\limits_{i=1}^{m} A_{ij} \beta_i \right) = \sum\limits_{j=1}^{n} \left(\sum\limits_{i=1}^{m} A_{ij} c_j \beta_i \right) = \sum\limits_{i=1}^{m} \left(\sum\limits_{j=1}^{n} A_{ij} c_j \beta_i \right) = \sum\limits_{i=1}^{m} \left(\sum\limits_{j=1}^{n} A_{ij} c_j\right) \beta_i$, i.e.
		$$[T(\alpha)]_{\base{_W}} = A[\alpha]_{\base{_V}}$$
		
		Por otro lado, seguimos considerando las bases fijas $\base{_V}, \base{_W}$. Considere $$\fdefc{\phi}{\Legen{V,W}}{\Mmnc{m}{n}{F}}{\tl{T}{V}{W}{\alpha}}{\phi\left(\tl{T}{V}{W}{\alpha}\right) = A_{\tl{T}{V}{W}{\alpha}}}$$ donde $A_{\tl{T}{V}{W}{\alpha}}$ representa a la matriz dada anteriormente.
		
		Considere $c \in F, \tl{T}{V}{W}{\alpha}, \tl{U}{V}{W}{\alpha}$. Es fácil mostrar que $$A_{\tlcomb{cT}{V}{W}{\alpha}} = cA_{\tl{T}{V}{W}{\alpha}} \text{ y } A_{\tlcomb{T+U}{V}{W}{\alpha}} = A_{\tl{T}{V}{W}{\alpha}} + A_{\tl{U}{V}{W}{\alpha}}$$, entonces
		
		\begin{align*}
			\phi\left(c\tl{T}{V}{W}{\alpha} + \tl{U}{V}{W}{\alpha}\right) &= A_{\tlcomb{cT + U}{V}{W}{\alpha}} \\
			&= cA_{\tl{T}{V}{W}{\alpha}} + A_{\tl{U}{V}{W}{\alpha}}.
		\end{align*}
		Recuerde que $\forall (i,j) \in \inte{1}{m} \times \inte{1}{n}$
		
		\begin{align*}
			\left[cA_{\tl{T}{V}{W}{\alpha}} + A_{\tl{U}{V}{W}{\alpha}} \right] ((i,j)) &= c\left[A_{\tl{T}{V}{W}{\alpha}}\right]((i,j)) + \left[A_{\tl{U}{V}{W}{\alpha}}\right]((i,j)) \\
			&= c\phi\left(\tl{T}{V}{W}{\alpha}\right) + \phi\left(\tl{U}{V}{W}{\alpha}\right)
		\end{align*}
		
		Luego, $\fdefc{\phi}{\Legen{V,W}}{\Mmnc{m}{n}{F}}{\tl{T}{V}{W}{\alpha}}{\phi\left(\tl{T}{V}{W}{\alpha}\right) = A_{\tl{T}{V}{W}{\alpha}}}$ es un homomorfismo.
		
	\end{proof}
\end{teo}

\begin{teo}
	Sean $V$ y $W$ \Fevs{F} de dimensión finita $n$ y $m$, respectivamente. Sea $\base{_V}$ una base ordenada de $V$ y $\base{_W}$ una base ordenada de $W$. La función $\fdefc{\phi}{\Legen{V,W}}{\Mmnc{m}{n}{F}}{\tl{T}{V}{W}{\alpha}}{\phi\left(\tl{T}{V}{W}{\alpha}\right) = A_{\tl{T}{V}{W}{\alpha}}}$ es un isomorfismo.
	
	\begin{proof}
		Ejercicio.
	\end{proof}
\end{teo}

Es conveniente denotar la matriz representante de una \tlc $\tl{T}{V}{V}{\alpha}$, i.e. un operador lineal, respecto a una misma base $\base{}$ como $[T]_{\base{}}$.

\begin{teo} \label{teo:RepresentacionDeComposicionDeTL}
	Sean $V, W, Z$ \Fevs{F} de dimensión finita, $\tl{T}{V}{W}{\alpha}$, $\tl{U}{W}{Z}{\beta}$ \tlc, $\base{_V}, \base{_W}, \base{_Z}$ bases ordenadas de lo respectivos espacios $V,W,Z$. Si $A$ es la matriz representante de la \tlc $\tl{T}{V}{W}{\alpha}$ respecto a las bases $\base{_V}, \base{_W}$ y $B$ es la matriz representante de la \tlc $\tl{U}{W}{Z}{\beta}$ respecto a las bases $\base{_W}, \base{_Z}$, entonces la matriz que representa a la composición $\tlcomb{UT}{V}{Z}{\alpha}$ respecto a las bases $\base{_V}, \base{_Z}$ es $C = BA$.
	
	\begin{proof}
		Ejercicio.
	\end{proof}
\end{teo}

Como consecuencia inmediata un operador lineal $\tl{T}{V}{V}{\alpha}$ es invertible si y sólo si $[T]_{\base{_V}}$ es invertible.

Entonces existe $\tl{U}{V}{V}{\beta}$ operador lineal tal que $$\tlcomb{UT}{V}{V}{\alpha} = \fdefc{I}{V}{V}{\alpha}{I(V)} = \tlcomb{TU}{V}{V}{\beta}$$. Además observe que
$$[T^{-1}]_{\base{_V}} = [T]_{\base{_V}}^{-1}.$$

Se quiere investigar ahora lo que le sucede a la matriz representante cuando se cambia la base ordenada. Por simplicidad, se considera sólo el caso de operadores lineales sobre un \Fev{F} $V$.

\begin{teo} \label{teo:CambioDeBasesTL}
	Si $V$ es un \Fev{F} de dimensión finita y $\base{} = \cv{\alpha}{n}$, $\base{'} = \cv{\alpha'}{n}$ son bases ordenadas de $V$ y $\tl{T}{V}{V}{\alpha}$ es un operador lineal.
	
	Si $P = [P_1, \ldots, P_n] \in \Mnc{n}{F}$ cuyas columnas $P_j = [\alpha_j']_{\base{}}$ entonces
	$$[T]_{\base{'}} = P^{-1}[T]_{\base{}}P$$
	es decir, $\tl{U}{V}{V}{\alpha}$ es un operador lineal tal que $U(\alpha_j) = \alpha'_j \; \forall j \in \inte{1}{n}$, entonces $[T]_{\base{'}} = [U]^{-1}_{\base{}}[T]_{\base{}} [U]_{\base{}}$.
	
	\begin{proof}
		Ejercicio.
	\end{proof}
\end{teo}

\begin{dfn}[Matrices similares]
	Sean $A,B \in \Mnc{n}{F}$. Se dice que $B$ es \textbf{similar} a $A$ sobre $F$ si existe $P \in \Mnc{n}{F}$ invertible tal que $B = P^{-1}AP$.
\end{dfn}

\section{Funcionales lineales}

\begin{dfn}
	Si $V$ es un \Fev{F} y $\tl{f}{V}{F}{\alpha}$ es una \tlc se dice que $\tl{f}{V}{F}{\alpha}$ es un \textbf{funcional lineal} sobre $V$.
\end{dfn}

\begin{ejem}
	Sea $F$ un campo y $a_1, \ldots, a_n \in F$. Sea $\fdefc{f}{F^n}{F}{(x_1, \ldots, x_n)}{\sum\limits_{i=1}^{n} a_i x_i}$.\\
	
	Sean $A = (a_1, \ldots, a_n)$, $B = (b_1, \ldots, b_n)$ y $c \in F$. Entonces
	\begin{align*}
		f(cA + B) &= f((ca_1 + b_1, \ldots, ca_n + b_n)) \\
		&= \sum\limits_{i=1}^{n} (ca_i + b_i) x_i \\
		&= \sum\limits_{i=1}^{n} (ca_ix_i + b_ix_i) \\
		&= \sum\limits_{i=1}^{n} ca_ix_i + \sum\limits_{i=1}^{n} b_ix_i \\
		&= c\sum\limits_{i=1}^{n}a_ix_i + \sum\limits_{i=1}^{n} b_ix_i \\
		&= cf(A) + f(B).
	\end{align*}
	entonces $\fdefc{f}{F^n}{F}{(x_1, \ldots, x_n)}{\sum\limits_{i=1}^{n} a_i x_i}$ es un funcional lineal.
\end{ejem}

\begin{ejem}
	Lugar de trabajo: $\Mnc{n}{F}$ como $\Fev{F}$.
	
	Sean $A,B \in \Mnc{n}{F}$, $\fdefc{Tr}{\Mnc{n}{F}}{F}{A}{\sum\limits_{i=1}^{n} A((i,i))}$ y $c \in F$.
	\begin{align*}
		Tr(cA + B) &= \sum\limits_{i=1}^{n} [cA((i,i)) + B((i,i))] \\
		&= \sum\limits_{i=1}^{n} cA((i,i)) + \sum\limits_{i=1}^{n} B((i,i)) \\
		&= c\sum\limits_{i=1}^{n} A((i,i)) + \sum\limits_{i=1}^{n} B((i,i)) \\
		&= cTr(A) + Tr(B)
	\end{align*}
	entonces $\fdefc{Tr}{\Mnc{n}{F}}{F}{A}{\sum\limits_{i=1}^{n} A((i,i))}$ es un funcional lineal. A este funcional lineal se le llama \textbf{traza}.
\end{ejem}

\begin{obs}
	Si $V$ es un \Fev{F}, la colección de funcionales lineales sobre $V$ forma un \Fev{F}.
\end{obs}

Si $\base{} = \cv{\alpha}{n}$ es base de un \Fev{F} $V$, $\forall i \in \inte{1}{n}$ se define $$\fdefc{f_i}{V}{F}{\alpha}{f_i(\alpha)}$$ funcional lineal, donde $f_j(\alpha_i) = \delta_{ij}$\\

Se tiene que $\{f_i\}_{i \in \inte{1}{n}}$ es base de $\Legen{V,F}$, en efecto:\\
Sean $c_1, \ldots, c_n \in F$ tales que $\sum\limits_{k=1}^{n} c_kf_k = 0_{\Legen{V,F}}$, en particular $\forall k \in \inte{1}{n}$
\begin{align*}
	0_V = 0_{\Legen{V,F}}(\alpha_i) &= \sum\limits_{k=1}^{n} e_kf_k(\alpha_i) \\
	&= \sum\limits_{k=1}^{n} c_k\delta_{ik} \\
	&= c_i
\end{align*}
entonces $c_i = 0$ $\forall i \in \inte{1}{n}$. Luego $\{f_i\}_{i\in\inte{1}{n}}$ es \li. Como $\card{\{f_i\}_{i\in \inte{1}{n}}} = \dim(V) = dim(\Legen{V,F}) = n$, entonces $\Legen{\{f_i\}_{i \in \inte{1}{n}}} = \Legen{V,W}$. Luego $\{f_i\}_{i \in \inte{1}{n}}$ es base de $\Legen{V,F}$.\\

Con las condiciones anteriores se denota por $\dual{\base{}}$ a $\{f_i\}_{i \in \inte{1}{n}}$ y se llama \textbf{base dual} de $\base{}$ y a $\Legen{V,F}$ se le denota por $\dual{V}$ y se le llama \textbf{espacio dual} de $V$.

\begin{ejem}
	Lugar de trabajo: $\mathbb{R}^{2}$ como $\Fev{\mathbb{R}}$.
	
	Sean $\alpha_1 = (1,2)$, $\alpha_2 = (1,1)$, $\beta = (5,7)$ y $\base{} = \{\alpha_1, \alpha_2\}$. $\base{}$ es base de $\mathbb{R}^{2}$.\\
	
	Si $c_1\alpha_1 + c_2\alpha_2 = \beta$, entonces encontramos que\\
	
	$\begin{matrizau}{2}
		1 & 1 & 5 \\
		2 & 1 & 7
	\end{matrizau}$$\begin{array}{c}
		R_1 \to R_1 - R_2
	\end{array}$
	$\begin{matrizau}{2}
		-1 & 0 & -2 \\
		2 & 1 & 7
	\end{matrizau}$$\begin{array}{c}
		R_2 \to R_2 + 2R_1 \\
		R_1 \to (-1)R_1
	\end{array}$
	$\begin{matrizau}{2}
		1 & 0 & 2 \\
		0 & 1 & 3
	\end{matrizau}$.\\
	
	y entonces $c_1 = 2$ y $c_2 = 3$. Luego si $\dual{\base{}} = \{f_1, f_2\}$ es la base dual de $\base{}$, entonces se tiene que $f_1(\alpha_1) = 1$, $f_1(\alpha_2) = 0$, $f_2(\alpha_1) = 0$ y $f_2(\alpha_2) = 1$.\\
	Entonces $$f_1(\beta) = f_1(2\alpha_1 + 4\alpha_2) = 2f_1(\alpha_1) + 3f_1(\alpha_2) = 2$$ y $$f_2(\beta) = f_2(2\alpha_1 + 4\alpha_2) = 2f_2(\alpha_1) + 3f_2(\alpha_2) = 3.$$
\end{ejem}

Cuando $V$ es de dimensión finita, entonces tenemos que $\dim(V^*) = \dim(V)$.

\begin{teo} \label{teo:ExisteUnaUnicaBaseDual}
	Sea $V$ un \Fev{F} de dimensión finita y $\base{} = \cv{\alpha}{n}$ una base de $V$. Entonces existe una única base dual $\base{^*} = \cv{f}{n}$ de $V^*$ tal que $f_i(\alpha_j) = \delta_{ij}$. Para cada funcional lineal $\tl{f}{V}{F}{\alpha}$ sobre $V$ se tiene
	$$f = \sum\limits_{i=1}^{n} f(\alpha_i) f_i$$
	y para cada $\alpha \in V$, se tiene
	$$\alpha = \sum\limits_{i=1}^{n} f_i(\alpha)\alpha_i.$$
	
	\begin{proof}
		Ejercicio.
	\end{proof}
\end{teo}

\begin{dfn}[Anulador de un subconjunto de un \Fev{F}]
	Si $V$ es un $F$-espacio vectorial y $S \subseteq V$, el \textbf{anulador} de $S$ (denotado por $S^0$) es el conjunto de funcionales lineales $\tl{f}{V}{F}{\alpha}$ tales que $f(\alpha) = 0$ $\forall \alpha \in S$. O bien,
	$$S^{0} = \left\{ \tl{f}{V}{F}{\alpha} \in V^* \;\Big|\; f(\alpha) = 0, \; \forall \alpha \in S \right\}.$$
\end{dfn}

\begin{ejem}
	Sea $V$ un \Fev{F}. Entonces
	\begin{itemize}
		\item $\varnothing^0 = \dual{V}$
		\item $\{0_V\}^0 = \dual{V}$
		\item $V^0 = 0_{\Legen{V,F}}$.
	\end{itemize}
	
	el anulador de un conjunto $S$ es un \Fsev{F} de $\dual{V}$.
\end{ejem}

\begin{teo} \label{teo:TeoremaDeLaDimensionAnuladores}
	Sea $V$ un \Fev{F} de dimensión finita y $W$ un $F$-subespacio vectorial de $V$. Entonces
	$$\dim(W) + \dim(W^{0}) = \dim(V).$$
	
	\begin{proof}
		Ejercicio.
	\end{proof}
\end{teo}

Para \Fevs{F} de dimensión finita, introducimos el concepto de \textbf{hiperespacio}. Si $V$ es un \Fev{F} de dimensión finita $n$, un \Fsev{F} de $V$ se dice hiperespacio si tiene dimensión $n-1$.

\begin{col} \label{col:DimensionAnuladores1}
	Si $W$ es un \Fsev{F} de dimensión finita $k$ de un $F$-espacio vectorial $V$ de dimensión finita $n$, entonces $W$ es la intersección de $n-k$ hiperespacios en $V$.
	
	\begin{proof}
		Ejercicio.
	\end{proof}
\end{col}

\begin{col} \label{col:DimensionAnuladores2}
	Si $W_1$ y $W_2$ son \Fsevs{F} de un \Fev{F} de dimensión finita, entonces $W_1 = W_2$ si y sólo si $W_1^{0} = W_2^{0}$.
	
	\begin{proof}
		Ejercicio.
	\end{proof}
\end{col}

\section{El doble dual}

\begin{teo} \label{teo:327}
	Sea $V$ un \Fev{F} de dimensión finita. Para cada $\alpha \in V$ se define
	$$\fdefc{L_\alpha}{V^*}{F}{f}{f(\alpha)}$$
	Entonces la función $\fdefc{\psi}{V}{V^{**}}{\alpha}{L_{\alpha}}$ es un isomorfismo.

	En términos categóricos, si $V$ y $W$ son $F$-espacios vectoriales de dimensión finita y si $\fdef{T}{V}{W}$ es t.l. y $\fdef{\psi_V}{V}{V^{**}}$ con $\psi_V(\alpha) = L_\alpha$ entonces el diagrama:

	\[
		\begin{tikzcd}[column sep=large, row sep=huge]
			V \arrow[d, "T"] \arrow[r, "\psi_V"] & V^{**} \arrow[d, "T^{**}"] \\
			W \arrow[r, "\psi_W"] & W^{**} 
		\end{tikzcd}
	\]

	conmuta. Luego $\psi_V$ y $\psi_W$ componen un isomorfismo natural.
	\begin{proof}
		Es claro que $L_\alpha$ es lineal. Sean $\alpha,\beta \in V$, $c \in F$ y $\gamma = c\alpha + \beta$. Entonces para cada $f \in V^{*}$
		\begin{align*}
			\psi(\gamma) = L_{\gamma} &= f(\gamma) \\
			&= f(c\alpha + \beta) \\
			&= cf(\alpha) + f(\beta) \\
			&= cL_\alpha(f) + L_\beta(f) \\
		\end{align*}
		Luego $\psi(\gamma) = \psi(c\alpha + \beta) = c\psi(\alpha) + \psi(\beta)$. Por lo que $\fdefc{\psi}{V}{V^{**}}{\alpha}{L_{\alpha}}$ es una transformación lineal.\\
		Ahora supongamos que $\alpha \not= 0$. Entonces $L_\alpha \not= 0_{V**}$. En efecto, existe $\fdefc{f}{V}{F}{\alpha}{c_1}$ donde $\alpha = \sum\limits_{i=1}^{n} c_i \alpha_i$ para la base ordenada $\base{} = \{\alpha, \alpha_2, \ldots, \alpha_n\}$, por lo que $f(\alpha) = 1 \not= 0$. Luego $L_\alpha \not= 0_{V^{**}}$, por lo que si $L_\alpha = 0_{V^{**}}$ entonces $\alpha = 0$. Ahora si $\alpha = 0$ entonces para $f \in \dual{V}$ se tiene que $f(\alpha) = f(0) = 0$, luego $L_\alpha = 0_{V^{**}}$.\\
		Así $\alpha = 0$ si y sólo si $L_\alpha = 0_{V^{**}}$. Luego $\fdefc{\psi}{V}{V^{**}}{\alpha}{L_{\alpha}}$ es no singular, por lo tanto inyectiva. Finalmente como $\dim(V^{**}) = \dim(V^{*}) = \dim(V)$, por el Teorema \ref{teo:313}, $\fdefc{\psi}{V}{V^{**}}{\alpha}{L_{\alpha}}$ es suprayectiva. Luego $\fdefc{\psi}{V}{V^{**}}{\alpha}{L_{\alpha}}$ es un isomorfismo.
	\end{proof}
\end{teo}

\begin{col} \label{col:IsomorfismoDual1}
	Sea $V$ un \Fev{F} de dimensión finita. Si $\tl{L}{V}{F}{\alpha} \in V^*$ es un funcional lineal, entonces hay un único $\alpha \in V$ tal que $L(f) = f(\alpha)$ $\forall \tl{f}{V}{F}{\alpha} \in V^*$.
	
	\begin{proof}
		Ejercicio.
	\end{proof}
\end{col}

\begin{col} \label{col:IsomorfismoDual2}
	Sea $V$ un \Fev{F} de dimensión finita. Cada base de $V^*$ es el dual de alguna base de $V$.
	
	\begin{proof}
		Ejercicio.
	\end{proof}
\end{col}

De Teorema \ref{teo:327} podemos identificar a un vector $\alpha \in V$ con $\fdefc{L_\alpha}{V^*}{F}{f}{f(\alpha)}$ y del mismo modo a $V$ con $V^{**}$. De modo que, abusando de la notación, decimos que $V$ es el dual de $V^{*}$ o que $V$ y $V^{*}$ están en dualidad mutua de forma natural.\\

Considere ahora el conjunto $E \subseteq V^{*}$. Entonces $E^{0}$ es un subconjunto de $V^{**}$. Así, si identificamos a $V^{**}$ con $V$ entonces $E^{0}$ es un subespacio de $V$, a saber, el conjunto de los $\alpha \in V$ tales que $F(\alpha) = 0$ para todo $f \in E$. Luego, enunciamos el siguiente teorema

\begin{teo}
	Si $S$ es cualquier subconjunto de un \Fev{F} $V$ de dimensión finita, entonces $(S^0)^0 = \Legen{S}$.
	
	\begin{proof}
		Sea $W = \Legen{S}$. Claramente $W^{0} = S^{0}$. Por el Teorema \ref{teo:TeoremaDeLaDimensionAnuladores} se tiene que
		\begin{align*}
			\dim(W) + \dim(W^{0}) &= \dim(V) \\
			\dim(W^{0}) + \dim(W^{0})^{0} &= \dim(V^{*}).
		\end{align*}
		y como $\dim(V) = \dim(V^{*})$ se tiene que
		$$\dim(W) = dim((W^{0})^0).$$
		Como $W$ es subespacio de $(W^0)^0$ se tiene entonces que $W = (W^0)^0$ y luego $(S^0)^0 = \Legen{S}$.
	\end{proof}
\end{teo}

Los resultados de esta sección son válidos para espacios vectoriales arbitrarios, con la necesidad de utilizar el \textbf{Axioma de Elección}.\\
Ahora consideremos $V$ un \Fev{F}. Es claro que no podemos construir hiperespacios para $V$ si $dim(V) \not\in \mathbb{N}$. Por esto damos la siguiente definición:

\begin{dfn}[Hiperespacio en un \Fev{F}]
	Si $V$ es un \Fev{F}, un \textbf{hiperespacio} en $V$ es un \Fsev{F} $N$, distinto de $V$ tal que si $W$ es un \Fsev{F} de $V$ con $N \subseteq W$ entonces $W = N$ ó $W = V$.
\end{dfn}

donde expresamos la idea de que $N$ \Fsev{F} de $V$ tiene "una dimensión menos'' que $V$. Es decir, no existe un \Fsev{F} propio de $V$ más grande que $N$. Para sintetizar esta idea decimos que $N$ es maximal en $V$.

\begin{teo}
	Si $\tl{f}{V}{F}{\alpha}$ es un funcional lineal en un \Fev{F} $V$ distinto del funcional lineal cero, entonces $\ker_f$ es un hiperespacio en $V$. Recíprocamente, todo hiperespacio de $V$ es el kernel de un (no único) funcional lineal $\tl{f}{V}{F}{\alpha}$ distinto del funcional lineal cero sobre $V$.
	
	\begin{proof}
		Sea $f \in V^{*}$ con $f \not= 0_{V^*}$ y $N_f$ su kernel. Sea $\alpha \in V \setminus N_f$. Si $\beta \in V$ entonces $\beta \in \Legen{N_f \cup \{\alpha\}}$. En efecto.
		
		Defínase $c = \frac{f(\beta)}{f(\alpha)}$. Entonces $\gamma := \beta - c\alpha \in N_f$, pues $f(\gamma) = f(\beta - c\alpha) = f(\beta) - cf(\alpha) = 0$. Así $\beta \in \Legen{N_f \cup \{\alpha\}}$.\\
		
		Ahora, sea $N$ un hiperespacio de $V$. Consideremos a $\alpha \in V \setminus N$ fijo. Como $N$ es maximal, $\Legen{N \cap \{\alpha\}} = V$. Luego cada vector $\beta \in V$ tiene la forma $\beta = \gamma + c\alpha$ para $\gamma \in N, c \in F$.\\
		
		$\gamma$ y $c$ son determinados de manera única por $\beta$. En efecto, si tenemos $\beta = \gamma' + c'\alpha$, $\gamma' \in N, c' \in F$ entonces $(c' - c)\alpha = \gamma - \gamma'$. Si $c' - c \not= 0$ entonces $\alpha \in N$, una contradicción. Por lo que $c' = c$ y $\gamma = \gamma'$. \\
		
		Luego entonces, si $\beta \in V$ existe un único $c_{\beta} \in F$ tal que $\beta - c_\beta \alpha \in N$. Sea $g(\beta)$ el mismo $c$. Entonces $g(\beta)$ define un funcional lineal $\fdefc{g}{V}{F}{\beta}{c}$.\\
		
		En efecto, sean $\beta, \theta \in V$ y $d \in F$. Entonces para $d\beta$ $f(d\beta) = c_{d\beta}$. Luego, $\beta - \frac{c_{d\beta}}{d} \in N$, por lo que $\frac{c_{d\beta}}{d} = c_\beta$ y entonces $f(d\beta) = df(\beta)$.
		
		Ahora, $f(\beta + \theta) = c_{\beta + \theta}$. Así, si $\beta = \gamma_{\beta} + c_{\beta}\alpha$ y $\theta = \gamma_{\theta} + c_{\theta}\alpha$, 
		\begin{align*}
			\gamma_{\beta} + c_{\beta}\alpha + \gamma_{\theta} + c_{\theta}\alpha &= \gamma_{\beta + \theta} + c_{\beta + \theta}\alpha \\
			(\gamma_{\beta} + \gamma_{\theta}) - \gamma_{\beta + \theta} &= (c_{\beta + \theta} - (c_{\beta} + c_{\theta}))\alpha \\
		\end{align*}
		
		Y si $(\gamma_{\beta} + \gamma_{\theta}) - \gamma_{\beta + \theta} \not= 0$ entonces $\alpha \in N$. De modo que $c_{\beta + \theta} = c_{\alpha} + c_{\theta}$ y entonces $f(\beta + \theta) = f(\beta) + f(\theta)$.\\
		
		Finalmente, como $f(\beta - c_{\beta}\alpha) = f(\beta) - c_{\beta}f(\alpha) = c_{\beta} - c_{\beta} = 0$ se tiene que $$N = \ker_{f}.$$
	\end{proof}
\end{teo}

\begin{lema} \label{lema:gIgualcfSSIKerfIgualKerg}
	Si $\tl{f}{V}{F}{\alpha}$ y $\tl{g}{V}{F}{\alpha}$ son funcionales lineales en un $F$-espacio vectorial $V$, $c \in F$, entonces $g = cf$ si y sólo si $\ker_f \subseteq \ker_g$, i.e.
	$$f(\alpha) = 0 \implies g(\alpha) = 0.$$
	
	\begin{proof}
		Ejercicio.
	\end{proof}
\end{lema}

\begin{teo} \label{teo:332}
	Sean $\tl{g}{V}{F}{\alpha},\tl{f_1}{V}{F}{\alpha},\ldots,\tl{f_n}{V}{F}{\alpha}$ funcionales lineales sobre un \Fev{F} $V$ y $c_1, \ldots, c_n \in F$. Entonces $$g = \sum\limits_{i=1}^{n} c_i f_i$$ si y sólo si $\ker_{f_1} \cap \cdots \cap \ker_{f_r} \subseteq \ker_g$.
	
	\begin{proof}
		Si $g = \sum\limits_{i=1}^{n} c_i f_i$ y $f_i(\alpha) = 0$ para cada $i \in \inte{1}{n}$ entonces $g(\alpha) = 0$. Así, $\bigcap\limits_{i=1}^{n} \ker_{f_i} \subseteq \ker_g$. \\
		
		Para probar el regreso procedemos por inducción sobre n.
		
		Por el Lema \ref{lema:gIgualcfSSIKerfIgualKerg} se tiene probado el caso base $n = 1$. Supongamos ahora que el teorema es cierto para $n = k-1$. Sean $\tl{g'}{V}{F}{\alpha},\tl{f_1'}{V}{F}{\alpha},\ldots,\tl{f_n'}{V}{F}{\alpha}$ con $g' = g|_{\ker_{f_k}}$ y $f_j' = f_j |_{\ker_{f_k}}$, $j \in \inte{1}{k-1}$. Así, si $\alpha \in \ker_{f_k}$ y $f_j(\alpha) = 0$, entonces $\alpha \in \bigcap\limits_{i=1}^{k} \ker_{f_i}$. En particular $\alpha \in \bigcap\limits_{i=1}^{k-1} \ker_{f_i}$ y entonces, por hipótesis de inducción, existen $c_1, \ldots, c_{k-1} \in F$ tales que
		$$g' = \sum\limits_{j=1}^{k-1} c_j f_j' \text{ y entonces } g'(\alpha) = 0.$$
		Sea $h := g - \sum\limits_{j=1}^{k-1}c_j f_j$. Entonces $h$ es un funcional lineal en $V$ y $h(\alpha) = 0$ $\forall \alpha \in \ker_{f_k}$. Luego, por el Lema \ref{lema:gIgualcfSSIKerfIgualKerg} $h = cf_k$, $c \in F$. Si $h = c_kf_k$ entonces
		$$g = \sum\limits_{i=1}^{k} c_i f_i.$$
	\end{proof}
\end{teo}

\section{Ejercicios}
\begin{enumerate}[label=\arabic*.]
	\item Demostrar los casos restantes del Teorema \ref{teo:TeoremaDeLaDimensión}.
	\item Comprobar que la función $T$ en el Teorema \ref{teo:ComprobarTL} es una \tlc
	\item Demostrar el Teorema \ref{teo:EspacioDeTLEsUnEV}.
	\item Mostrar que $\mathscr{A}$, en el Teorema \ref{teo:DimensionDeLVWMostrarAEsLI}, es \li
	\item Demostrar el Lema \ref{lema:OperacionesEnLVW}.
	\item Demostrar el Teorema \ref{teo:TNoSingularSSITMandaBasesABases}.
	\item Demostrar el Teorema \ref{teo:RepresentacionDeComposicionDeTL}.
	\item Demostrar el Teorema \ref{teo:CambioDeBasesTL}.
	\item Demostrar el Teorema \ref{teo:ExisteUnaUnicaBaseDual}.
	\item Demostrar el Teorema \ref{teo:TeoremaDeLaDimensionAnuladores}.
	\item Demostrar el Corolario \ref{col:DimensionAnuladores1}
	\item Demostrar el Corolario \ref{col:DimensionAnuladores2}
	\item Demostrar el Corolario \ref{col:IsomorfismoDual1}
	\item Demostrar el Corolario \ref{col:IsomorfismoDual2}
	\item Demostrar el Lema \ref{lema:gIgualcfSSIKerfIgualKerg}
	\item Sea $V$ un $\Fev{\mathbb{C}}$, suponga que $\tl{T}{V}{\mathbb{C}^{3}}{\alpha}$ es un isomorfismo. Sean $\alpha_1, \alpha_2, \alpha_3, \alpha_4 \in V$ tal que
	$$\begin{array}{c c}
		T(\alpha_1) = (1,0,i) & T(\alpha_2) = (-2,1+i,0) \\
		T(\alpha_3) = (-1,1,1) & T(\alpha_4) = (\sqrt{2},i,3)
	\end{array}$$
	\begin{enumerate}[label=\alph*), leftmargin=2em]
		\item ¿$\alpha_1 \in \Legen{\{\alpha_2, \alpha_3\}}$?
		\item Sea $W_1 = \Legen{\{\alpha_1, \alpha_2\}}$ y $W_2 = \Legen{\{\alpha_3, \alpha_4\}}$. Determine $W_1 \cap W_2$.
		\item Encuentre una base para el $\Fsev{\mathbb{C}}$ $\Legen{\{\alpha_1, \alpha_2, \alpha_3, \alpha_4\}}$ de $V$.
	\end{enumerate}
	\item ¿Es la función $\fdefc{T}{\conmodr{n} \times \conmodr{n}}{\conmodr{n} \times \conmodr{n}}{(\clamodr{x_1}{n}, \clamodr{x_2}{n})}{(\clamodr{x_1}{n}^{2}, \clamodr{x_2}{n})}$ una \tlc? Considere los casos
	\begin{enumerate}[label=\alph*), leftmargin=2em]
		\item $n \in \mathbb{N}$.
		\item $n \in \mathbb{N}, n=p$ número primo.
	\end{enumerate}
	\item Si $C([a,b])$ es el conjunto de funciones continuas en el intervalo $[a,b] \subseteq \mathbb{R}$, demostrar que $\fdefc{I}{C([a,b])}{\mathbb{R}}{f}{\int\limits_{a}^{b}f(t)dt}$ es un funcional lineal.
	
	\item Sea $V$ el \Fev{\mathbb{R}} de las funciones polinomiales $\fdefc{p}{\mathbb{R}}{\mathbb{R}}{x}{p(x)}$ tal que $\text{grad}(p) \leq 2$. Defínase los funcionales lineales
	$$ \fdefc{f_1}{V}{\mathbb{R}}{p}{\int_{0}^{1}p(x)dx}, \hspace{0.5cm} \fdefc{f_2}{V}{\mathbb{R}}{p}{\int_{0}^{2}p(x)dx}, \hspace{0.5cm} \fdefc{f_3}{V}{\mathbb{R}}{p}{\int_{0}^{-1}p(x)dx}.$$
	
	Demuestra que $\{f_1, f_2, f_3\}$ es una base de $V^*$ mostrando la base de $V$ de la cual es el dual.
	\item Utiliza el Teorema \ref{teo:332} para probar lo siguiente. Si $W$ es un \Fsev{F} de un \Fev{F} de dimensión finita y si $\{g_1, \ldots, g_r\}$ es cualquier base de $W^{0}$, entonces
	$$W = \bigcap_{i=1}^r \ker_{g_i}.$$
\end{enumerate}
