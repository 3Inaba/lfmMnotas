\setcounter{chapter}{-1}
%\justifying
\chapter{Preliminares}
\pagenumbering{arabic}
\setcounter{page}{1}

\section{Nociones sobre conjuntos}
\subsection{El lenguaje de la teoría de conjuntos$^\dagger$}
Esta sección se indica opcional mediante el superíndice $^\dagger$. Está aquí para clarificar lo que es una propiedad en la teoría de conjuntos y cómo construirlas a partir de interpretaciones.

La teoría desarrollada en el curso hace uso de la teoría de conjuntos de Zermelo-Frankel con el axioma de elección. Un estudio más detallado se puede encontrar en \cite{settheoryhernandez}, \cite{settheoryraymond} y \cite{tecoloteII}. Para consultar sobre la lógica que de la que se hace uso aquí, \cite{alfredologica} y \cite{metamathematics} son fuentes suficientes para entender más a detalle lo que aquí se presenta. 

Se presenta la teoría de conjuntos Zermelo-Fraenkel (ZF) con el Axioma de Elección (ZFE) de manera axiomática a través de un lenguaje formal. Un lenguaje formal consta de un alfabeto, reglas sintácticas y una interpretación. A grandes rasgos, un lenguaje formal es un lenguaje con estructura de \com{cálculo}. Se supone que el lector ya tiene nociones de lógica, aún cuando estas no son del todo formales, se entiende que sabe cómo formar proposiciones yuxtaponiendo símbolos como $p \cond q$. De manera similar, se dará el lenguaje de la teoría de conjuntos con el cual se formarán cadenas de símbolos bajo ciertas reglas y las cuales interpretaremos como dichos o declaraciones sobre los conjuntos. 

Decimos que el lenguaje formal de la teoría de conjuntos (o simplemente el lenguaje TC) consiste de los siguientes símbolos \cite{breton}:
\begin{enumerate}[label=\arabic*.]
	\item Funtores o conectivos lógicos: $\land, \lor, \cond, \bicond$.
	\item Cuantificadores: $\forall, \exists$.
	\item Variables: $a, b, c, \ldots, A, B, C, \ldots, \alpha, \beta, \gamma, \ldots, a_1, a_2, a_3, \ldots$.
	\item Paréntesis: $(,)$.
	\item Relatores: $=, \in$.
\end{enumerate}

Una cadena de símbolos es la yuxtaposición de símbolos del lenguaje formal. En este caso, $\cond \forall A \exists = \alpha \in \land \beta$ es una cadena de símbolos que el lector percibirá que no tiene sentido. Y con toda razón. Lo que buscamos es construir cadenas de símbolos con cierta estructura para poder dotarlas de un significado sin ambigüedades. Así entonces, damos reglas sintácticas para construir lo que llamaremos {\bfseries fórmulas}, primero en noción y luego dando reglas de formación explícitas..

Una fórmula del lenguaje TC es una cadena finita de símbolos (finita en su sentido común: que termina) de forma que \com{tenga sentido}, lo cual se puede lograr de varias formas\footnote{Para evitar ambigüedad, libros que introducen el lenguaje formal de la lógica proposicional definen fórmulas mediante el uso de la notación polaca inversa. Por ejemplo, $\cond p q$ (note la estructura $f(a,b)$) tiene dicha estructura y tiene la interpretación usual, $p$ entonces $q$. Sin embargo esta notación evita la ambigüedad de escribir, por ejemplo, $p \cond q \cond r$, supliéndola por una notación de lectura única ${\cond} \cond p q r$ sin la necesidad de paréntesis.}. Nos decantaremos por evitar esta notación y en su lugar utilizar los paréntesis para remover ambigüedades. Primero, queremos expresar declaraciones conjuntistas mediante la lógica. Por esto es necesario que dentro de nuestro lenguaje esté contenido el lenguaje de la lógica de primer orden. Esto es, que si cadenas de símbolos del lenguaje TC, $\varphi$ y $\psi$, son fórmulas, entonces estas suplen el lugar de las variables proposicionales $\neg \varphi$, $\varphi \cond \psi$, $\varphi \land \psi$, $\varphi \lor \psi$ y $\varphi \bicond \psi$, haciendo de estas nuevas proposiciones, fórmulas del lenguaje TC. En cuanto a los cuantificadores, las fórmulas toman el lugar de las proposiciones para formar predicados como $\forall x (\varphi)$ o $\exists y (\psi)$ donde $x$ y $y$ son variables.

Ahora, las reglas de formación explícitas son:
\begin{enumerate}[label=RF\arabic*., leftmargin=4em]
	\item Para cualesquiera variables $x$ y $y$, $x \in y$ es una fórmula.
	\item Para cualesquiera variables $x$ y $y$, $x = y$ es una fórmula.
	\item Para cualquier fórmula $\varphi$, $\neg \varphi$ es una fórmula.
	\item Para cualesquiera fórmulas $\varphi$ y $\psi$, $\varphi \cond \psi$ es una fórmula.
	\item Para cualesquiera fórmulas $\varphi$ y $\psi$, $\varphi \land \psi$ es una fórmula.
	\item Para cualesquiera fórmulas $\varphi$ y $\psi$, $\varphi \lor \psi$ es una fórmula.
	\item Para cualquier fórmula $\varphi$ y cualquier variable $x$, $\forall x (\varphi)$ es una fórmula.
	\item Para cualquier fórmula $\varphi$ y cualquier variable $x$, $\exists x (\varphi)$ es una fórmula.
	\item Estas son las únicas maneras de construir fórmulas.
\end{enumerate}

Estas reglas de formación nos permiten saber si una cadena es una fórmula del lenguaje TC mediante la examinación parte por parte de la cadena. Por ejemplo \[\forall x \exists z (x \in y \cond \neg x = z)\] se construye a partir de las cadenas $x \in y$ y $x = z$ las cuales son fórmulas por RF1 y RF2. Entonces $\neg x = z$ es una fórmula por RF3 y $x \in y \cond \neg x = z$ es una fórmula por RF4. Luego $\exists z ( x\in y \cond \neg x = z )$ es una fórmula por RF8 y entonces $\forall x \exists z (x \in y \cond \neg x = z)$ es una fórmula por RF7.

De este modo, otras cadenas como $\forall x \land \in z \neg$ no son fórmulas y esto encajará con el sentido del que dotaremos a las fórmulas.

Los símbolos lógicos se interpretan de la misma manera que se hace en su propio lenguaje formal, e.g. \com{entonces}, \com{y}, \com{o} (en su sentido inclusivo), etc. Mientras que los símbolos del lenguaje TC $\in$ y $=$ se interpretan como \com{estar en} (o simplemente \com{en}) y \com{es igual} o simplemente \com{igual}, respectivamente. Esto nos permite interpretar predicados más complejos, como \[\exists y \forall x (\neg x \in y)\] como \com{existe $y$ para todo $x$ tal que no $x$ está en $y$}, que mejor interpretado significaría \com{existe $y$ para todo $x$ tal que ningún $x$ está en $y$}. Aquí una distinción recurrente que llega a causar problemas es que $\forall x (\varphi)$ no significa que existan elementos $x$ para los cuales se declare $\varphi$. Significa que, sin importar si existen o no existen elementos $x$, de ellos se dice $\varphi$. Así entonces, una traducción final a la fórmula anterior sería \com{existe un $y$ tal que, de existir $x$, entonces $x$ no está en $y$}, la cual deja claro este rol del cuantificador $\forall$, pero quedará implícita en la mayoría de declaraciones sobre conjuntos que se hagan. Y, sin abordar demasiado el tema, los cuantificadores trabajan sobre un {\itshape universo} definido. Esto es, cuando se dice algo sobre las variables $x$ mediante la fórmula $\forall x (\varphi)$, esas variables tienen que salir de algún lugar, del universo de trabajo. Esto permite, intuitivamente, pensar que cuando escribimos $\forall x (\varphi)$, nos es posible determinar un asignamiento de verdad (verdadero o falso) a la fórmula mediante la interpretación de las variables $x$ en dicho universo\footnote{Algo que no exploramos aquí es lo que se entiende por verdad de las fórmulas. Para ese tipo de estudio se requiere introducir la noción de {\itshape modelo}, lo cual se desvía de los propósitos de la sección. Por esto, se deja al lector con otra noción intuitiva de las fórmulas que son verdaderas y las que no dependiendo del universo que contiene a sus variables}.

Para un tratamiento más riguroso, es necesario cubrir por completo las reglas de transformación. Sin embargo, se evitará esto, en su lugar dejando al lector con la manipulación lógica que ha practicado en el primer semestre o con otros textos. Las reglas de transformación son reglas que nos dicen cómo pasar sintácticamente de una fórmula a otra. Estas vienen, usualmente, de la intuición lógica usual, es decir, de manipular los predicados o proposiciones para obtener otros equivalentes. Por ejemplo, pasar de $\neg (\forall x (\varphi))$ a $\exists x (\neg \varphi)$ o de $\neg(\varphi \land \psi)$ a $(\neg \varphi) \lor (\neg \psi)$. La cuestión sintáctica y deductiva requiere de métodos como el axiomático o el de deducción natural para establecer formalmente las reglas. Para el lector, de momento, será suficiente con manejar intuitivamente estos conceptos, guiándose por la intuición o de algún libro.

Abreviamos $\neg x = y$ por $x \not= y$, $\neg x \in y$ por $x \not\in y$, $\forall z (z \in x \cond z \in y)$ por $x \subseteq y$ y $\neg x \subseteq y$ por $x \not\subseteq y$. De maneras similares, podemos introducir nuevos símbolos para representar fórmulas de nuestro lenguaje formal mientras exploramos la teoría.

Ahora, el principal fin de esta sección es dejar claro qué constituye una propiedad en nuestro lenguaje formal. Para esto, se da la siguiente definición.

\begin{dfn}
	Si $\varphi$ es una fórmula del lenguaje TC, decimos que $x$ es una {\bfseries variable libre} en $\varphi$ si $x$ no está afectada por ningún cuantificador. Denotamos que $x$ es libre en $\varphi$ por $\varphi(x)$. Si al menos un cuantificador afecta a $x$ entonces decimos que la variable es ligada en $\varphi$.
\end{dfn}

Por ejemplo, la fórmula $\exists y (x \not= y)$ tiene variable libre $x$ y variable ligada $y$. Su interpretación es \com{existe un $y$ tal que $x$ no es igual a $y$}, de donde se debe notar que se está diciendo algo sobre $x$, o que se está dictando una {\bfseries propiedad} de $x$ mientras que se está utilizando a $y$ para decir algo sobre $x$. Otro ejemplo, en $\exists y \forall x (x \not\in y)$ no hay variables libres. En este caso, no estamos predicando algo sobre algún $y$, estamos afirmando que existe uno que cumple con $x \not\in y$.

Intuitivamente, una fórmula con variable libre $x$ se interpreta como una \com{propiedad} de $x$. Esto nos permite hablar de propiedadades mientras podamos llevarlas al lenguaje TC. Así, establecemos la siguiente definición.

\begin{dfn}
	Introduciendo los símbolos $\{,\}$ y $\mid$, si $\varphi$ es una fórmula del lenguaje TC con variable libre $x$, definimos un \textbf{término clase} como la cadena de símbolos de la forma $\{x \mid \varphi(x)\}$.
\end{dfn}

Vale la pena interpretar la definición anterior. Un término clase o simplemente una \textbf{clase} es un objeto que no es parte del lenguaje TC pero es de utilidad para hablar sobre la teoría que se construye a traves de él. Las clases se cuantifican sobre conjuntos, informalmente, son colecciones de conjuntos. Aquí $\mid$ se interpreta como \com{tal que} de modo que la definición nos dice que la clase $A$ es la colección de los conjuntos $x$ tales que $x$ es libre en $\varphi$, o bien, tales que $\varphi(x)$ (que se interpreta como ``aquellos $x$ que cumplen\footnote{Note que un término clase junto con la fórmula de variable libre $x$ es una definición meramente sintáctica. No exploramos lo que significa, por ejemplo, sustituir un $a$ del universo en la fórmula o qué significa que $\varphi$ sea verdadera en $x$.} la propiedad $\varphi$").

Si le es posible definir lo que significa \com{ser un perro rabioso} en el lenguaje TC entonces le es posible construir la colección de los perros rabiosos a través de los términos clases. Esto último, en pocas palabras, se refiere a lo que significa \com{formalizar} un concepto. En medida de lo posible, la formalización trata de evitar elecciones arbitrarias de propiedades, por ejemplo, \com{ser pequeño} o \com{estar gordo} son propiedades subjetivas que pueden ser formalizadas\footnote{e.g. decimos que $x$ es pequeño si $\exists x_1 \exists x_2 \forall y (y \in x \bicond (y = x_1 \lor y = x_2))$. Pero estas nociones de pequeño y grande pueden ser de interés para teorías como la de categorías.} pero que carecen de importancia para la teoría, por ejemplo, de los números reales.

Podemos describir la \textbf{clase de todos los conjuntos} por $\uclass := \{x \mid x=x\}$ (la colección de todos los $x$ que son iguales a sí mismos, esto es, todos los objetos de la teoría de conjuntos: los conjuntos) y entonces convenir en que $x \in \uclass$ significa que $x$ es un conjunto, y en general, decir que si $x \in A$ entonces $\varphi(x)$, i.e., $x$ cumple la propiedad que define al término clase $A$.

Si se tienen clases $A$ y $B$ se escribe $A \cap B$ por $\{x \mid x \in A \land x \in B\}$ y se escribe $A \subseteq B$ por $\forall x (x \in A \cond x \in B)$.

A continuación, utilizando el lenguaje que acabamos de definir (de manera poco formal) se crea una TC-teoría, i.e. una colección de fórmulas del lenguaje TC que se siguen de la manipulación lógica de fórmulas dadas, que llamamos {\itshape axiomas}. La teoría de conjuntos ZFE tiene los siguientes axiomas.

\begin{enumerate}[label=A.\arabic*]
	\item $\exists y \forall x (\neg x \in y)$,
	\item $\forall a \forall b \forall x (x \in a \bicond x \in b) \cond a=b$,
	\item Para cada término clase $A$, $x \cap A \in \mathcal{V}$,
	\item $\forall a \forall b \exists y \forall x (x \in y \bicond x = a \lor x = b)$,
	\item $\forall a \exists y \forall z (z \in y \bicond \exists x \in a(z \in x))$,
	\item $\forall a \exists y \forall z (z \in y \bicond z \subseteq a)$,
	\item $\forall x (\varnothing \not = x \cond \exists f(f \text{ es una función de elección para } x))$.
\end{enumerate}

Algunos detalles sobre el Axioma 7 se encuentran en el apéndice B. Este es utilizado de forma implícita en muchas demostraciones en matemáticas.

\subsection{Algunos axiomas y resultados}
Ahora es conveniente interpretar los axiomas para trabajar con ellos de forma más entendible en el lenguaje natural.

\begin{axiom}[de Existencia]
	Existe un conjunto que no tiene elementos.
\end{axiom}

\begin{axiom}[de Extensión]
	Si todo elemento de $a$ es un elemento de $b$ entonces $a=b$.
\end{axiom}

De estos dos primeros axiomas se puede demostrar que el conjunto que no tiene elementos es único y por lo tanto podemos denotarlo por $\varnothing$. 

\begin{axiom}[de Especificación/Esquema de Comprensión]
	Para cualquier propiedad $\varphi(z)$ (fórmula del lenguaje TC con variable libre $z$) y conjunto $x$, la clase \[\{z \mid z \in x \land \varphi(z)\}\] que escribimos como \[\{z \in x \mid \varphi(z)\}\] es un conjunto.
\end{axiom}

Note que en realidad el axioma Esquema de Comprensión adquiere su nombre por ser un esquema para crear, a lo más, tantos axiomas como clases.

\begin{axiom}[del Par]
	Para cualesquiera conjuntos $a$ y $b$ existe un conjunto $y$ tal que $x \in y$ si y sólo si $x = a$ o $x = b$.
\end{axiom}

\begin{axiom}[de Unión]
	Para cualquier conjunto $a$ existe un conjunto $y$ tal que $z \in y$ si y sólo si existe un $x \in a$ tal que $z \in x$.
\end{axiom}

\begin{axiom}[del Conjunto Potencia]
	Para cualquier conjunto $a$ existe un conjunto $y$ tal que $z \in y$ si y sólo si $z \subseteq a$.
\end{axiom}

Los conjuntos determinados por el axioma Esquema de Comprensión, axioma del Par, el axioma de Unión y el axioma del Conjunto Potencia son únicos. Por esto podemos denotar sin ambigüedad al conjunto determinado por A.6 por $\mathcal{P}(a)$, al conjunto determinado por A.5 por $\bigcup a$ y si $a$ consiste de dos elementos (i.e. $\forall z (z \in a \bicond z = x \lor z = y)$ con $x \not= y$), denotando $a = \{x,y\}$, entonces escribimos $x \cup y := \bigcup \{x,y\}$. En general, si $a$ es el conjunto definido por $\forall x (x \in a \bicond x = x_1 \lor \cdots \lor x = x_n)$ con $x_1, \ldots, x_n$ distintos entre sí entonces podemos denotar a través de las clases $a = \{x_1, \ldots, x_n\}$.

No se extenderá el estudio de conjuntos en estas notas a los términos clases, así que de aquí en adelante se denota a los conjuntos con letras mayúsculas o letras caligráficas, góticas, etc. Para un estudio desde otras teorías o extensiones de la utilizada consultar \cite{settheoryraymond} y \cite{quine}.

\begin{dfn}[Par ordenado]
	Se define el \textbf{par ordenado} de elementos $a$ y $b$ como
	$$ (a,b) = \{ \{a\},\{a,b\} \} $$
\end{dfn}

\begin{dfn}[Producto cartesiano]
	Sean $A$ y $B$ conjuntos cualesquiera. El \textbf{producto cartesiano} de $A$ y de $B$ es el conjunto $A \times B$ consistente de todos aquellos pares ordenados $(a,b)$ tales que $a \in A$ y $b \in B$, esto es,
	$$ A \times B = \{(a,b) \mid a \in A \wedge b \in B\}.$$
\end{dfn}

Se puede demostrar que para cualesquiera $A$ y $B$ el conjunto $A \times B$ existe y está dado de la siguiente manera:
$$ A \times B = \{ (a,b) \in \mathcal{P}(\mathcal{P}(A \cup B)) \mid a \in A \wedge b \in B \} $$

\begin{dfn}[Relación (binaria)]
	Si $A$ y $B$ son conjuntos, se dice que el conjunto $R$ es una \textbf{relación (binaria) de $A$ en $B$} si 
	$$ R \subseteq A \times B.$$
	Si $R \subseteq A \times A$ diremos simplemente que $R$ es una relación en $A$.
	Y denotamos $(x,y) \in R$ como $x \mathrel{R} y$. En caso contrario, escribimos $x \not\mathrel{R} y$ 
\end{dfn}

\begin{ejem}
	\begin{enumerate}[label=\roman*)]
		\item $\varnothing$ y $A \times B$ son relaciones en $A \times B$ (de $A$ en $B$).
		\item $\{(1,1), (2,2), (1,4), (3,4)\}$ no es una relación en $\{1,2,3\}$ pero sí en $\{1,2,3,4\}$.
	\end{enumerate}
\end{ejem}

\begin{dfn}[Relación de asignación total]
	Si $A$ y $B$ son conjuntos y $R$ es una \textbf{relación de $A$ en $B$}, se dice que $R$ es relación de \textbf{asignación total} si
	$$ \forall a \in A \; \exists \; b_a \in B \text{ tal que } (a, b_a) \in R$$
\end{dfn}

\begin{dfn}[Relación de asignación única]
	Si $A$ y $B$ son conjuntos y $R$ es una relación de $A$ en $B$, se dice que $R$ es relación de \textbf{asignación única} si satisface que
	$$ \forall a,b,c \; \text{ si } (a,b) \in R \text{ y } (a,c) \in R \text{ entonces } b = c.$$
\end{dfn}

\begin{dfn}[Función]
	Si $A$ y $B$ son conjuntos y $R$ es una relación de $A$ en $B$, se dice que $R$ es una \textbf{función}, si $R$ es una relación de asignación total y de asignación única.
\end{dfn}

\begin{ejem}
	\begin{enumerate}[label=\roman*)]
		\item $\{(1,3), (2,5)\}$ no es relación de asignación total de $\{1,2,3\}$ en $\{1,2,3,4,5\}$. Pero sí desde $\{1,2\}$.
		\item $\{(1,1),(1,2),(2,1)\}$ no es relación de asignamiento único en $\{1,2\}$. Pero $\{(1,2)$, $ (2,1)\}$ sí lo es.
		\item $\fdef{f}{\real \setminus \{-2,-3\}}{\real}$ con $x \mapsto 1/(x^2-5x+6)$ no es función.
		\item $\fdef{g}{\real \setminus \{2,3\}}{\real}$ con $x \mapsto 1/(x^2-5x+6)$ es función.
	\end{enumerate}
\end{ejem}

\begin{dfn}[Relación reflexiva]
	Si $A$ es un conjunto y $R$ es una relación en $A$ se dice que $R$ es \textbf{reflexiva} si $\forall a \in A$ $(a,a) \in R$.
\end{dfn}

\begin{ejem}
	\begin{enumerate}[label=\roman*)]
		\item La relación de perpendicularidad $\perp$ no es reflexiva.
		\item $\not=$ no es reflexiva.
		\item $\parallel$ depende de la definición.
		\item $a \sim b \bicond n \mid b - a$ es reflexiva.
	\end{enumerate}
\end{ejem}

\begin{dfn}[Relación simétrica]
	Si $A$ es un conjunto y $R$ es una relación en $A$ se dice que $R$ es \textbf{simétrica} si $\forall a,b \in A \text{ si } (a,b) \in R \text{ entonces } (b,a) \in R$.
\end{dfn}

\begin{dfn}[Relación transitiva]
	Si $A$ es un conjunto y $R$ es una relación en $A$ se dice que $R$ es \textbf{transitiva} si $\forall a,b,c \in A \text{ si } (a,b) \in R \text{ y } (b,c) \in R \text{ entonces } (a,c) \in R$.
\end{dfn}

\begin{dfn}[Relación antisimétrica]
	Si $A$ es un conjunto y $R$ es una relación en $A$ se dice que $R$ es \textbf{antisimétrica} si $\forall a,b \in A \text{ si } (a,b) \in R \text{ y } (b,a) \in R$, entonces $a = b$.
\end{dfn}

\begin{dfn}[Relación de orden no estricto]
	Si $A$ es un conjunto y $R$ es una relación en $A$ se dice que $R$ es una \textbf{relación de orden no estricto} si satisface que $\mathrel{R}$
	\begin{enumerate}[label=\roman*), leftmargin=2em]
		\item es una relación reflexiva.
		\item es una relación antisimétrica.
		\item es una relación transitiva.
	\end{enumerate}
\end{dfn}

Para facilitar la escritura se escribe $a \mathrel{R} b$ para $(a,b) \in R$.

\begin{dfn}[Relación de equivalencia]
	Si $A$ es un conjunto y $R$ es una relación en $A, A \not= \varnothing$, se dice que $R$ es una \textbf{relación de equivalencia} si $\mathrel{R}$
	\begin{enumerate}[label=\roman*), leftmargin=2em]
		\item es una relación reflexiva.
		\item es una relación simétrica.
		\item es una relación transitiva.
	\end{enumerate}
\end{dfn}

\begin{dfn}[Clase de equivalencia módulo $R$]
	Si $A$ es un conjunto no vacío y $R$ es una relación de equivalencia en $A$ y $a \in A$, se define la \textbf{clase de equivalencia del elemento $a$ módulo $R$} (denotado $[\,a\,]_R$) como
	$$ [\,a\,]_R = \{x \in A \mid a \mathrel{R} x\}.$$
\end{dfn}

\begin{prop}[Representación múltiple de una clase]
	Si $A$ es un conjunto no vacío, $R$ es una relación de equivalencia en $A$, $a,b \in A$ y $a \mathrel{R} b$, entonces $[\,a\,]_R = [\,b\,]_R$.
	
	\begin{proof}
		$\boxed{\subseteq}$ $[\,a\,]_R \subseteq [\,b\,]_R$.
		Suponga que $a \mathrel{R} b$, por simetría $b \mathrel{R} a$, además si $x \in [\,a\,]_R$ entonces $a \mathrel{R} x$, luego por transitividad $b \mathrel{R} x$. Así $x \in [\,b\,]_R$\\
		
		$\boxed{\supseteq}$ $[\,b\,]_R \subseteq [\,a\,]_R$ se demuestra de fórma análoga.
	\end{proof}
\end{prop}

\begin{prop}[Clases disjuntas]
	Si $A$ es un conjunto no vacío, $R$ una relación de equivalencia en $A$, $a,b \in A$ y $a \not \mathrel{R} b$, entonces $[\,a\,]_R \cap [\,b\,]_R = \varnothing$.
	
	\begin{proof}
		Procedemos por contrapositiva. \\
		Suponga que $[\,a\,]_R \cap [\,b\,]_R \not= \varnothing$, así $\exists x_0 \in [\,a\,]_R, x_0 \in [\,b\,]_R$, así $a \mathrel{R} x_0$ y $b \mathrel{R} x_0$, en particular por simetría $x_0 \mathrel{R} b$, luego por transitividad $a \mathrel{R} b$. Por lo anterior se tiene lo deseado.
	\end{proof}
\end{prop}

\begin{dfn}[Partición y probable partición]
	Si $A$ es un conjunto no vacío y $\{ A_{\alpha} \}_{\alpha \in \Omega}$ una familia no vacía de subconjuntos de $A$, se dice que la colección $\{ A_{\alpha} \}_{\alpha \in \Omega}$ es una \textbf{posible partición} de $A$ si se satisface:
	\begin{enumerate}[label=\roman*)]
		\item $\bigcup\limits_{\alpha \in \Omega} A_{\alpha} = A$.
		\item Si $\alpha,\beta \in \Omega \text{ con } \alpha \not= \beta \text{, entonces } A_{\alpha} \cap A_{\beta} = \varnothing$
		\item Si además se satisface, $\forall \alpha \in \Omega, A_{\alpha} \not= \varnothing$, entonces se dice que $\{ A_{\alpha} \}_{\alpha \in \Omega}$ es una \textbf{partición}.
	\end{enumerate}
\end{dfn}

\begin{prop}[Particiones de conjuntos inducen relaciones de equivalencia]
	Si $A$ es un conjunto no vacío y $\{ A_{\alpha} \}_{\alpha \in \Omega}$ una familia de conjuntos la cual es una partición de $A$, entonces la siguiente relación es una relación de equivalencia.
	$$ x \mathrel{\sim_{\Omega}} y \bicond \exists a \in \Omega \text{ tal que } x,y \in A_{\alpha} $$
	
	\begin{proof}
		\boxed{Reflexiva} Como $\{ A_{\alpha} \}_{\alpha \in \Omega}$ es una partición de $A$, en particular $\bigcup_{\alpha \in \Omega} A_\alpha$, así, $\forall a \in A$ existe $\alpha_a \in \Omega$ tal que $a \in A_{\alpha_a}$, en particular $a \in A_{\alpha_a} \wedge a \in A_{\alpha_a}$. Así $a \mathrel{\sim_{\Omega} a}$. \\
		
		\boxed{\text{\textit{Sim\'etrica}}} Suponga que $x \mathrel{\sim_{\Omega}} y$, entonces existe $\alpha_{xy} \in \Omega$ tal que $x,y \in A_{\alpha_{xy}}$, así $y,x \in A_{\alpha_{xy}}$, luego $y \mathrel{\sim_{\Omega}} x$. \\
		
		\boxed{Transitiva} Suponga que $x \mathrel{\sim_\Omega} y \wedge y \mathrel{\sim_\Omega}$ z, entonces existen $\alpha_{xy}, \alpha_{yz} \in \Omega$ tales que $x,y \in A_{\alpha_{xy}}$ y $y,z \in A_{\alpha_{yz}}$, en particular $y \in A_{\alpha{xy}}, y \in A_{\alpha_{yz}}$.
		
		Como $\{ A_{\alpha} \}_{\alpha \in \Omega}$ es partición, entonces $A_{\alpha_{xy}} = A_{\alpha{yz}}$, así $x,y,z \in A_{\alpha{xy}}$ en particular $x,z \in A_{\alpha_{xy}}$, así $x \mathrel{\sim_\Omega} z$.
	\end{proof}
\end{prop}

\subsection{Definiciones y resultados específicos para nuestro estudio}

\begin{dfn}[Elementos comparables bajo un orden parcial]
	Sea $A$ un conjunto y $\leq$ un orden parcial en $A$. Se dice que $a,b \in A$ son \textbf{comparables} bajo $\leq$ si $a \leq b$ o $b \leq a$.
\end{dfn}

\begin{dfn}[Cadena]
	Sea $A$ un conjunto y $\leq$ un orden en $A$. Se dice que $\mathscr{C}$ es una \textbf{cadena} en $A$ si para cualesquiera $a,b \in A$ los elementos en $\mathscr{C}$ son comparables.
\end{dfn}

\begin{dfn}[Orden total]
	Si $(A, \leq)$ es un conjunto parcialmente ordenado, se dice \textbf{orden total} si $\forall a,b \in A$ $a$ y $b$ son comparables. 
\end{dfn}

\begin{obs}
	Una cadena es un orden total.
\end{obs}

\begin{dfn}[Elemento máximo de un subconjunto de un conjunto parcialmente ordenado]
	Sea $(A, \leq)$ un conjunto parcialmente ordenado, $B \subseteq A$ y $b \in A$, $M \in B$ se dice \textbf{elemento máximo} del conjunto $B$ en el orden parcial $\leq$ si para todo $d \in B$, $d \leq M$.
\end{dfn}

\begin{dfn}[Supremo]
	Si $(A, \leq)$ es un conjunto parcialmente ordenado, $S$ se dice \textbf{supremo} de $B \subseteq A$ en el conjunto parcialmente ordenado $(A, \leq)$ si $S$ es el mínimo del conjunto de cotas superiores.
\end{dfn}

\begin{dfn}[Elemento maximal de un subconjunto de un conjunto parcialmente ordenado]
	Sea $(A, \leq)$ es un conjunto parcialmente ordenado y $B \subseteq A$. $M' \in B$ se dice \textbf{elemento maximal} de $B$ en el orden parcial $\leq$ si no existe $d \in B$ tal que $M' \leq d$, $M' \not= d$.
\end{dfn}

\begin{dfn}[Conjuntos equipotentes]
	Si $A$ y $B$ son conjuntos, se dice que $A$ y $B$ son \textbf{equipotentes} o que tienen la misma cardinalidad (denotando $A \sim B$ o $\card{A} = \card{B}$) si existe $\underset{a \mapsto \phi(b)}{\phi : A \longrightarrow B}$ tal que $\phi$ es biyectiva.
\end{dfn}

\begin{ejem}
	$\mathbb{N} \sim \mathbb{Z}$. En efecto, la función
	$$\fdefc{\phi}{\mathbb{N}}{\mathbb{Z}}{n}{\left\{\begin{array}{ccc}
			n, \text{ si } n = 0 \\
			-\frac{n+1}{2}, \text{ si } n \text{ es impar} \\
			\frac{n}{2}, \text{ si } n \text{ es par} \\
		\end{array}\right.}$$
	es una función biyectiva.
\end{ejem}

\begin{prop} \label{prop:equipotenciaEnV}
	La equipotencia en la clase de todos los conjuntos es una relación de equivalencia de clases (definiendo un análogo para estas).
	\begin{proof}
		Ejercicio.
	\end{proof}
\end{prop}

\begin{lema}[Lema de Zorn] \label{lema:Zorn}
	Sea $A$ es un conjunto no vacío parcialmente ordenado por $\leq$. Para toda cadena $\mathscr{C}$ en $A$, si $\mathscr{C}$ está acotada superiormente entonces $A$ tiene elementos maximales.
	\begin{proof}
		\cite{settheoryhernandez}.
	\end{proof}
\end{lema}

\begin{teo}[Cantor-Bernstein] \label{teo:CantorBernstein}
	Sean $A$ y $B$ conjuntos. Si existen
	$ \underset{a \mapsto \phi_1(a)}{\phi_1 : A \longrightarrow B} $
	función inyectiva y
	$ \underset{b \mapsto \phi_2(b)}{\phi_2 : B \longrightarrow A} $
	función inyectiva, entonces $A \sim B$.
	\begin{proof}
		\cite{settheoryhernandez}.
	\end{proof}
\end{teo}

\begin{teo}[Buen orden]
	Todo conjunto puede ser bien ordenado.
	\begin{proof}
		\cite{settheoryhernandez}.
	\end{proof}
\end{teo}

\begin{dfn}[Conjunto ordenado por un conjunto de índices bien ordenado]
	Si $A$ y $\Omega$ son conjuntos, $\underset{\alpha \mapsto \phi(\alpha) = \alpha_k}{\phi : \Omega \longrightarrow A}$ y $\leq_\omega$ un buen orden en $\Omega$, se dice que el conjunto $A$ está \textbf{ordenado bajo el orden inducido} por $\leq_\omega$ si el conjunto se ordena bajo $\phi(\Omega)$, es decir $\{ \phi(\alpha) \}_{\alpha \in (\Omega, \leq_\omega)}$ es un conjunto bien ordenado.
\end{dfn}

\section{Algunas estructuras algebraicas.}

\begin{dfn}[Operación binaria]
	Una operación binaria $*$ en un conjunto no vacío $S$ es una función $\fdefc{*}{S \times S}{S}{(s,t)}{*((s,t))}$ (y denotamos $*((s,t))$ como $s*t$).
	\begin{itemize}
		\item La operación se dice que es \textbf{asociativa} si $(s \mathrel{*} t) \mathrel{*} r = s \mathrel{*} (t \mathrel{*} r)$ $\forall s,t,r \in S$.
		\item La operación se dice que es \textbf{conmutativa} si $s \mathrel{*} t = t \mathrel{*} s$ $\forall s,t \in S$.
		\item Un elemento $e$ en $S$ es llamado una \textbf{identidad} de $*$ si $s \mathrel{*} e = e \mathrel{*} s = s$ $\forall s \in S$.
		\item Si $*$ tiene un elemento identidad $e$, y $s \in S$, entonces $l \in S$ se dice un \textbf{inverso} para $s$ si $s \mathrel{*} l = l \mathrel{*} s = e$.
	\end{itemize}
\end{dfn}

\begin{dfn}[Grupo]
	Una estructura $(G, *)$ se dice grupo si satisface
	\begin{enumerate}[label=\roman*), leftmargin=2em]
		\item $G$ es un conjunto no vacío.
		\item $\fdefc{*}{G \times G}{G}{(g,h)}{*((g,h)):=g\mathrel{*}h}$ es una operación binaria la cual satisface
		\begin{enumerate}[label=\alph*), leftmargin=2em]
			\item \textbf{Asociatividad.} $\forall a,b,c \in G$ $(a \mathrel{*} b) \mathrel{*} c = a \mathrel{*} (b \mathrel{*} c)$.
			\item \textbf{Identidad.} Existe un \textbf{elemento identidad} $e \in G$, esto es,
			
			$\forall a \in G$ $e \mathrel{*} a = a \mathrel{*} e = a$.
			\item \textbf{Inversos.} Para cada $a \in G$ existe un elemento \textbf{inverso} $a^{-1} \in G$, esto es, un elemento $a^{-1} \in G$ tal que
			
			$a \mathrel{*} a^{-1} = a^{-1} \mathrel{*} a = e$.
		\end{enumerate}
		
		Si además cumple
		\begin{enumerate}[label=\alph*), leftmargin=2em]
			\setcounter{enumii}{3}
			\item \textbf{Conmutatividad.} $\forall a,b \in G$ $a \mathrel{*} b = b \mathrel{*} a$.
		\end{enumerate}
		
		Se dice \textbf{grupo abeliano}.
	\end{enumerate}
\end{dfn}
%\end{tcolorbox}

%\begin{tcolorbox}[title=Anillo, colback=backanillo, boxrule=0.0mm, arc=0.2mm, breakable, enhanced, colframe=backanillo, sharp corners, coltitle=white, colbacktitle=colanillo, fonttitle=\scshape, borderline={0.3mm}{0.5mm}{colanillo}, attach boxed title to top center={xshift=0mm, yshift=-2mm}]
\begin{dfn}[Anillo]
	Una estructura $(R, +, \cdot)$ se dice anillo si satisface
	\begin{enumerate}[label=\roman*), leftmargin=2em]
		\item $(R, +)$ es un grupo abeliano.
		\item $\fdefc{\cdot}{R \times R}{R}{(a,b)}{\cdot((a,b)):=a\mathrel{\cdot}b}$ es una operación binaria asociativa.
		\item $+$ y $\cdot$ satisfacen las propiedades distributivas, es decir
		\begin{enumerate}[label=\alph*), leftmargin=2em]
			\item $a \cdot (b + c) = a \cdot b + a \cdot c$ $\forall a,b,c \in R$.
			\item $(a + b)\cdot c = a \cdot c + b \cdot c$ $\forall a,b,c \in R$.
		\end{enumerate}
		\begin{itemize}
			\item El anillo se dice \textbf{conmutativo} si $\forall a,b \in R$ $a \cdot b = b \cdot a$.
			\item Se dice \textbf{anillo con identidad} si existe un $1_R \in R$ tal que $\forall a \in R$ $1_R \cdot a = a \cdot 1_R = a$.
			\item Se dice \textbf{anillo conmutativo con identidad} si $R$ es un anillo conmutativo y es un anillo con identidad.
			\item El anillo se dice \textbf{dominio entero} si es anillo conmutativo con identidad y satisface la propiedad
			
			$\forall a,b \in R$ si $a \cdot b = 0_R$ entonces $a = 0_R$ o $b = 0_R$.
			\item El anillo se dice \textbf{campo} si, además de ser anillo conmutativo con identidad, satisface
			
			$\forall a \in R \setminus \{0_R\}$ existe un $a^{-1} \in R$ tal que $a \cdot a^{-1} = a^{-1} \cdot a = 1_R$.
		\end{itemize}
	\end{enumerate}
\end{dfn}
%	\end{tcolorbox}

%\begin{tcolorbox}[title=Subanillo, colback=backsubanillo, boxrule=0.0mm, arc=0.2mm, breakable, enhanced, colframe=backsubanillo, sharp corners, coltitle=white, colbacktitle=colsubanillo, fonttitle=\scshape, borderline={0.3mm}{0.5mm}{colsubanillo}, attach boxed title to top center={xshift=0mm, yshift=-2mm}]
\begin{dfn}[Subanillo]
Si $R$ es un anillo, un conjunto $S$ se dice \textbf{subanillo} de $R$ si $S \subseteq R$ y $S$ es un anillo.
\end{dfn}
%\end{tcolorbox}
Los subanillos $S \subseteq R$ $(S, +_S, \cdot_S)$ tienen como operaciones las respectivas restricciones de las operaciones del anillo $(R, +, \cdot)$ a $S$.

\begin{dfn}[Grupo de unidades]
Si $A$ es una estructura algebraica, a 
$$U_R = \{u \in A \mid u \text{ tiene inverso } \}$$
se le llama \textbf{grupo de unidades} de $A$.
\end{dfn}

%\begin{tcolorbox}[title=Ideal, colback=backideal, boxrule=0.0mm, arc=0.2mm, breakable, enhanced, colframe=backideal, sharp corners, coltitle=white, colbacktitle=colideal, fonttitle=\scshape, borderline={0.3mm}{0.5mm}{colideal}, attach boxed title to top center={xshift=0mm, yshift=-2mm}]
\begin{dfn}[Ideal]
Si $R$ es un anillo, un conjunto $I$ se dice \textbf{ideal} si $I$ es un subanillo de $R$ tal que $\forall a \in R$ y $\forall i \in I$, $a \cdot i \in I$ y $i \cdot a \in I$.
\end{dfn}
%\end{tcolorbox}

La siguiente es la generalización de la estructura espacio vectorial.

\begin{dfn}[Módulo izquierdo]
	Sea $R$ un anillo. Una estructura $(M, +_M, \cdot_M, R)$ se dice \textbf{módulo izquierdo} si satisface
	\begin{enumerate}[label=\roman*), leftmargin=2em]
		\item $(M, +_M)$ es un grupo abeliano.
		\item $\fdefc{\cdot_M}{R \times M}{M}{(a,m)}{\cdot_M((a,m)):=a\mathrel{\cdot_M}m}$ es una función que satisface
		\begin{enumerate}[label=\alph*), leftmargin=2em]
			\item $\forall a,b \in R$ y $\forall m \in M$
			$(a \cdot b) \cdot_M m = a \cdot_M (b \cdot_M m)$.
			\item $\forall a, b \in R$ y $\forall m \in M$ $(a + b) \cdot_M m = a \cdot_M m +_M b \cdot_M m$.
			\item $\forall a \in R$ y $\forall m,n \in M$ $a \cdot_M (m +_M n) = a \cdot_M m +_M a \cdot_M n$.
			\item $\forall m \in M$ existe un $1_M \in R$ tal que $1_M \cdot_M m = m$.
		\end{enumerate}
	\end{enumerate}
\end{dfn}
%\end{tcolorbox}

También se puede definir el \textbf{módulo derecho}, la razón por la que se enuncia la definición de módulo izquierdo es porque usualmente se conviene realizar la operación $\cdot_M$ desde la izquierda.

\section{Ejercicios}
\begin{enumerate}[label=\arabic*.]
	\item Demostrar que los conjuntos determinados por A.1, A.3, A.4, A.5 y A.6 son únicos.
	\item Demostrar la Proposición \ref{prop:equipotenciaEnV}.
	\item Encontrar todas las relaciones en $\{1,2,3\}$ y determinar de qué tipo son.
	\item Determine cuáles de las siguientes son relaciones de equivalencia:
	\begin{enumerate}[label=\alph*)]
		\item $\in_{X} := \{(x,y) \in X \times X \mid x \in y\}$ donde $X = \{\varnothing, \{\varnothing\}, \{\{\varnothing\}\}\}$.
		\item $\{(1,1), (2,2), (3,3), (1,2), (2,3)\}$ en $\{1,2,3\}$.
		\item $\ls$ en $\real$.
		\item $\not=$ en $\real$.
	\end{enumerate}
	\item ¿Existe algún conjunto $X$ donde $\in_X$ sea una relación de equivalencia? Para esto considere los casos i) afirmamos el axioma de fundación: $$\text{A.8) }A \not= \varnothing \cond \exists x \in A \forall y \in A (y \not\in x)$$ y ii) cuando lo negamos.
	\item Demostrar que el conjunto $\mathcal{P}_g = \{y \in \mathcal{F} \mid \frac{d^2}{dx^2}y(x) + a_1\frac{d}{dx}y(x) + a_0 y(x) = 0\}$ con la operación usual de $+$ forma un grupo abeliano.
	\item Demostrar que el conjunto $\mathcal{F}_{\mathscr{I}} = \{f \in \mathcal{F} \mid f \text{ es integrable en } [a,b]\}$ junto con la operación usual de $+$ forma un grupo abeliano.
	\item \begin{enumerate}[label=\alph*), leftmargin=2em]
		\item Utilizar el teorema fundamental de la aritmética para encontrar una inyección de $\mathbb{Q}^+$ a $\mathbb{N}$.
		\item Demostrar que $\mathbb{N} \sim \mathbb{Q}$
	\end{enumerate}
	\item Denótese la clase $\gconmodr{\lp}{\equiv}$ a la clase de todas las fórmulas del lenguaje formal de la lógica proposicional reducida bajo la relación de equivalencia de fórmulas. Defínase $\varphi \oplus \psi \equiv \neg (\varphi \bicond \psi)$ y denótese una tautología por $\top$ y una contradicción por $\bot$.
		
		Demuestre que $(\gconmodr{\lp}{\equiv}, \oplus, \land)$ es un anillo.
	\item Demuestre que $\uclass$ no es un conjunto.
	\item Demuestre que $\bigcap \varnothing = \uclass$.
	\item Si $y \not= \varnothing$ es un conjunto, demuestre que la clase $\{x \mid x \sim y \}$ no es un conjunto.
	\item Demuestre que la clase $\{x \mid x \text{ es función}\}$ no es un conjunto.
\end{enumerate}
