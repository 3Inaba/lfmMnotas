\setcounter{chapter}{-1}
%\justifying
\chapter{Preliminares}
\pagenumbering{arabic}
\setcounter{page}{1}

\iffalse
\section{Lógica de predicados de primer orden}

Se introduce un tratamiento de la lógica como lenguaje formal. A quien tenga la curiosidad de saber más sobre lógica lea los libros \cite{tecoloteI}, \cite{mathlogicmonk}, \cite{logicivorra}, \cite{setlogicivorra}, o con un enfoque más filosófico lea \cite{alfredologica} y \cite{metamath}.

Considérese la noción de colección de objetos. No se dará como tal una definición de esto así que se tendrá que pensar con la idea intuitiva durante el tratamiento que se dará de la lógica en estos apuntes. De manera similar a la noción de conjunto que dio Cántor, consideramos una colección un objeto por sí mismo, con un nombre y elementos que están en relación con la colección por la relación de pertenencia. Por ejemplo, se dirá que $\difsim$ está en $\mathfrak{O}$.

\begin{dfn}
	Un lenguaje formal de primer orden es una colección de colecciones de símbolos que contiene:
	\begin{enumerate}[label=\roman*)]
		\item Una colección de símbolos $\Clf$ que llamamos constantes.
		\item Una colección de símbolos $\Vlf$ que llamamos variables.
		\item Una colección de símbolos $\Flf$ que llamamos funtores.
		\item Una colección de símbolos $\Rlf$ que llamamos relatores.
		\item Una colección de símbolos $\Slf$ que llamamos conectores.
		\item Una colección de símbolos $\Qlf$ que llamamos cuantificadores.
	\end{enumerate}
\end{dfn}

El lenguaje formal sobre el que trabajaremos será:

\begin{dfn}
	El lenguaje formal de la lógica de primer orden $\lpo$ consiste de:
	\begin{enumerate}[label=\roman*)]
		\item $\Clfpo$ contiene letras minúsculas indizadas sobre números naturales $c_i$, en abstracto $c$.
		\item $\Vlfpo$ contiene letras minúsculas indizadas sobre números naturales $x_i$, en abstracto $x$.
		\item $\Flfpo$ contiene letras minúsculas indizadas sobre números naturales $f_i$, en abstracto $f$. Cada funtor tiene asociado una \com{$n$-aridad} que se denota por $f^n$.
		\item $\Rlfpo$ contiene el símbolo $=$ y contiene letras mayúsculas indizadas sobre números naturales $R_i$, en abstracto $R$. Cada relator tiene asociado una \com{$n$-aridad} que se denota por $R^n$.
		\item $\Slfpo$ contiene los símbolos $\cond$ y $\neg$ y ningún otro. A $\cond$ se le llama condicional o implicador y a $\neg$ se le llama negador.
		\item $\Qlfpo$ contiene el símbolo $\gen$ y ningún otro. A $\gen$ se le llama cuantificador universal o generalizador.
	\end{enumerate}
\end{dfn}

\begin{dfn}
	Una cadena de símbolos de $\lpo$ es una sucesión finita de símbolos de colecciones de $\lpo$. Si $a_1, \ldots, a_n$ son símbolos de $\Clfpo, \Vlfpo, \Flfpo$, $\Rlfpo$, $\Slfpo$ o $\Qlfpo$, se denota la sucesión finita formada por dichos elementos como $\cad{a_1 a_2 \cdots a_n}$ o simplemente $a_1 a_2 \cdots a_n$. Si $\cdA_1, \ldots, \cdA_k$ son cadenas de símbolos entonces la concatenación de las cadenas es la yuxtaposición de una delante de la otra, e.g., $\cad{\cdA_1 \cdots \cdA_k}$ o simplemente $\cdA_1 \cdots \cdA_k$. 
	
	(Note que para cualquier permutación $\sigma \in S_n$, $\cdA_{\sigma(1)} \cdots \cdA_{\sigma(n)}$ es una cadena.)
\end{dfn}

Decimos que dos cadenas $\cdA, \cdB$ son equivalentes si para cada término $a_i$ de $\cdA$ que corresponda a $b_i$ de $\cdB$ se tiene que $a_i$ es el mismo símbolo que $b_i$. Denotamos $\cdA \equiv \cdB$.

Así, por ejemplo, las siguientes son cadenas de símbolos de $\lpo$:
\begin{enumerate}[label=\roman*)]
	\item $\cad{{\cond} {\cond} R^n \neg a_2 {=} \alpha} \equiv \cad{{\cond}} \cad{{\cond} R^n} \cad{\neg a_2 } \cad{ {=} \alpha}$
	\item ${\cond}{\cond}{\cond}{\cond}{\cond}{\cond}{\cond}a$
\end{enumerate}

\begin{dfn}
	Una cadena de $\lpo$ se dice término si el primer símbolo es una variable, una constante o un funtor, y se dice fórmula si su primer símbolo es un relator, un conector o un cuantificador universal.
\end{dfn}

\begin{dfn}
	Una expresión de $\lpo$ es una cadena de símbolos $\phi$ si y sólo si existe una sucesión finita $\cad{\psi_1,\ldots,\psi_k}$ de cadenas de símbolos de $\lpo$ de modo que $\psi_k = \phi$ y cada $\psi_l$ satisface alguna de las siguientes condiciones:
	\begin{enumerate}[label=\roman*)]
		\item $\psi_l \equiv \cad{c}$ donde $c$ es una constante de $\lpo$,
		\item $\psi_l \equiv \cad{x}$ donde $\alpha$ es una variable de $\lpo$,
		\item $\psi_l \equiv f \psi_{j_1} \cdots \psi_{j_n}$ donde $f$ es un funtor $n$-ádico, $0 \le j_1, \ldots, j_n \ls l$ y $\psi_{j_1}, \ldots, \psi_{j_n}$ son términos.
		\item $\psi_l \equiv R \psi_{j_1} \cdots \psi_{j_n}$ donde $R$ es un relator $n$-ádico, $0 \le j_1, \ldots, j_n \ls l$ y $\psi_{j_1}, \ldots, \psi_{j_n}$ son términos.
		\item $\psi_l \equiv \neg \psi_j$, donde $0 \le j \ls l$ y $\psi_j$ es una fórmula.
		\item $\psi_l \equiv \cond \psi_i \psi_j$, donde $0 \le j,i \ls l$ y $\psi_j$ y $\psi_i$ son fórmulas.
		\item $\psi_l \equiv \forall x \psi_j$, donde $x$ es una variable de $\lpo$, $0 \le j \ls l$ y $\psi_j$ es una fórmula.
	\end{enumerate}
\end{dfn}

\begin{ejem}
	La cadena de símbolos $\phi \equiv \cad{{\cond} \neg \neg {=} x c {=} x c}$ es una expresión de $\lpo$. En efecto, consideremos $\cad{\psi_1,\psi_2,\psi_3,\psi_4,\psi_5,\psi_6}$:
	\begin{enumerate}[label=\arabic*)]
		\item $\psi_1 \equiv x$,
		\item $\psi_2 \equiv c$,
		\item $\psi_3 \equiv {=}xc$,
		\item $\psi_4 \equiv \neg{=}xc $,
		\item $\psi_5 \equiv \neg\neg{=}xc$,
		\item $\psi_6 \equiv {\cond}\neg\neg{=}xc{=}xc \equiv \phi$.
	\end{enumerate}
\end{ejem}

Note que cualquier expresión de $\lpo$ se lee de manera única. Una demostración de esto se hace mediante la formalización desde la matemática en \cite{mathlogicmonk}. Esta notación (usualmente llamada polaca) nos es de utilidad pues no necesita de paréntesis. Acordamos, de manera más informal, escribir expresiones en la notación usual haciendo uso de paréntesis teniendo en cuenta la notación con la que construimos las expresiones del lenguaje. Así, una expresión de $\lpo$ como ${\cond}{\cond}{=}c_1c_2{=}c_2c_1{\cond}\neg{=}c_2c_1\neg{=}c_1c_2$ se escribirá como $(c_1=c_2 \cond c_2=c_1) \cond (\neg c_2=c_1 \cond \neg c_1 = c_2)$.

Si $R$ es un relator a las expresiones del tipo $R^n t_1 \cdots t_n$ donde $t_1, \ldots, t_n$ son términos se les llamará fórmulas atómicas.

\begin{dfn}
	Un modelo $M$ para $\lpo$ consiste en:
	\begin{enumerate}[label=\roman*)]
		\item Una colección $U$ que llamaremos el universo de $M$.
		\item A cada $c$ en $\Clfpo$ se le asigna un único valor en $U$ denotado por $M(c)$ o $\olc$.
		\item A cada relator $n$-ádico $R^n$ en $\Rlfpo$ se le asigna una única relación $n$-ádica en $U$ denotada $M(R^n)$ o $\olR$. A ${=}$ se le asigna la identidad en $U$ ${\equiv}$.
		\item A cada funtor $n$-ádico $f^n$ en $\Flfpo$ se le asigna una única función $n$-ádica en $U$ denotada $M(f^n)$ o $\olf$.
	\end{enumerate}
\end{dfn}

Formalizando, un modelo es una función de la colección de expresiones de un lenguaje formal a la colección (el conjunto) $\{0,1\}$. La noción puede ser más general, consultar \cite{setlogicivorra}.

\begin{dfn}
	Una valoración en $\lpo$ en un modelo $M$ es una relación $v$ que asigna a cada variable $x$ de $\lpo$ un único objeto $v(x)$ del universo de $M$.
\end{dfn}

Si $v$ es una valoración en $\lpo$ en un modelo $M$, $a$ está en el universo de $M$ y $x$ es una variable, entonces denotamos $v_x^a$ a la valoración donde $v_x^a(y) \equiv a$ si $y = x$ y $v_x^a (y) \equiv v(y)$ si $y \not \equiv x$.

\begin{dfn}
	Si $v$ es una valoración de $\lpo$ en un modelo $M$, decimos que $M$ satisface la fórmula $\phi$ respecto a la valoración $v$ (denotado $M \sfl \phi[v]$) y cuándo un término $t$ denota en $M$, respecto a la valoración $v$, a un objeto denotado por $M(t)[v]$ o $\olt$, si se satisface:
	\begin{enumerate}[label=\roman*)]
		\item $M(x_i)[v] \equiv v(x_i)$,
		\item $M(c_i)[v] \equiv \olc_i$,
		\item $M \sfl R^n t_1 \cdots t_n$ si y sólo si $\olR^n(\olt_1 \cdots \olt_n)$,
		\item $M(f^n t_1 \cdots t_n)[v] \equiv \olf^n(\olt_1 \cdots \olt_n)$,
		\item $M \sfl \neg \phi[v]$ si y sólo si no $M \sfl \phi[v]$,
		\item $M \sfl (\phi_1 \cond \phi_2)$ si y sólo si no $M \sfl \phi_1[v]$ o $M \sfl \phi_2[v]$,
		\item $M \sfl \gen x_i \phi [v]$ si y sólo si para todo objeto $a$ del universo de $M$ se cumple que $M \sfl \phi[v_x^a]$.
	\end{enumerate}
\end{dfn}

\fi


\section{Nociones sobre conjuntos}
\subsection{Algunos axiomas y resultados}
La teoría desarrollada en el curso hace uso de la teoría de conjuntos de Zermelo-Frankel con el axioma de elección (ZFE). Un estudio más detallado se puede encontrar en \cite{settheoryhernandez}, \cite{settheoryraymond} y \cite{tecoloteII}. Para consultar sobre lógica, los primeros capítulos de dichos libros además de \cite{alfredologica} y \cite{metamathematics} son suficientes para tener una idea.

Se presenta la teoría de manera axiomática, donde se utilizan los conectores lógicos $\neg, \cond, \bicond, \lor, \land$ y los cuantificadores $\forall, \exists$. Las fórmulas atómicas serán aquellas formadas con los relatores $=$ y $\in$, donde $\in$ se interpreta como \com{está en} o simplemente \com{en}.

\begin{dfn}
	Introduciendo los símbolos $\{\}$ y $\mid$, un \textbf{término clase} es una cadena de símbolos de la forma $\{x \mid \varphi\}$ tal que $\varphi$ es una fórmula de la teoría de conjuntos.
\end{dfn}

Vale la pena interpretar la definición anterior. Un término clase o simplemente una \textbf{clase} es un objeto que no es parte de la teoría de conjuntos ZFE pero es de utilidad para hablar sobre ella. Las clases se cuantifican sobre conjuntos, informalmente, son colecciones de conjuntos.

Podemos describir la \textbf{clase de todos los conjuntos} por $\uclass := \{x \mid x=x\}$ y entonces convenir en que $x \in \uclass$ significa que $x$ es un conjunto.

Para clases $A$ y $B$ se define $A \cap B \equiv \{x \mid x \in A \land x \in B\}$ y $A \subseteq B \equiv \forall x (x \in A \cond x \in B)$.

A continuación, algunos axiomas de la teoría ZFE:

\begin{enumerate}[label=A.\arabic*]
	\item $\exists y \forall x (\neg x \in y)$,
	\item $\forall a \forall b \forall x (x \in a \bicond x \in b) \cond a=b$,
	\item Para cada término clase $A$ $x \cap A \in \mathcal{V}$,
	\item $\forall a \forall b \exists y \forall x (x \in y \bicond x = a \lor x = b)$,
	\item $\forall a \exists y \forall z (z \in y \cond \exists x \in a(z \in b))$,
	\item $\forall a \exists y \forall z (z \in y \bicond z \subseteq a)$,
	\item $\forall x (\varnothing \not = x \cond \exists f(f \text{ es una función de elección para } x))$.
\end{enumerate}

Algunos detalles sobre el Axioma 7 se encuentran en el apéndice B. Este es utilizado de forma implícita en muchas demostraciones en matemáticas.

Ahora es conveniente interpretar los axiomas:

\begin{axiom}[de Existencia]
	Existe un conjunto que no tiene elementos.
\end{axiom}

\begin{axiom}[de Extensión]
	Si todo elemento de $a$ es un elemento de $b$ entonces $a=b$.
\end{axiom}

De estos dos primeros axiomas se puede demostrar que el conjunto que no tiene elementos es único y por lo tanto podemos denotarlo por $\varnothing$. 

\begin{axiom}[de Especificación / Esquema de Comprensión]
	Si $\varphi$ es una fórmula de la teoría ZFE, diremos que $\varphi$ es una propiedad y denotaremos $\varphi(x)$ si la propiedad se evalúa como verdadera en $x$. Entonces para cada propiedad $\varphi$ y conjunto $x$, $\{y \in x \mid \varphi(y) \}$ es un conjunto.
\end{axiom}

\begin{axiom}[del Par]
	Para cualesquiera conjuntos $a$ y $b$ existe un conjunto $y$ tal que $x \in y$ si y sólo si $x = a$ o $x = b$.
\end{axiom}

\begin{axiom}[de Unión]
	Para cualquier conjunto $a$ existe un conjunto $y$ tal que $z \in y$ si y sólo si existe un $x \in a$ tal que $z \in x$.
\end{axiom}

\begin{axiom}[del Conjunto Potencia]
	Para cualquier conjunto $a$ existe un conjunto $y$ tal que $z \in y$ si y sólo si $z \subseteq a$.
\end{axiom}

Los conjuntos determinados por el axioma Esquema de Comprensión, axioma del Par, el axioma de Unión y el axioma del Conjunto Potencia son únicos.

No se extenderá el estudio de conjuntos en estas notas a los términos clases, así que de aquí en adelante se denota a los conjuntos con letras mayúsculas o letras caligráficas, góticas, etc. Para un estudio desde otras teorías o extensiones de la utilizada consultar \cite{settheoryraymond} y \cite{quine}.

\begin{dfn}[Par ordenado]
	Se define el \textbf{par ordenado} de elementos $a$ y $b$ como
	$$ (a,b) = \{ \{a\},\{a,b\} \} $$
\end{dfn}

\begin{dfn}[Producto cartesiano]
	Sean $A$ y $B$ conjuntos cualesquiera. El \textbf{producto cartesiano} de $A$ y de $B$ es el conjunto $A \times B$ consistente de todos aquellos pares ordenados $(a,b)$ tales que $a \in A$ y $b \in B$, esto es,
	$$ A \times B = \{(a,b) \mid a \in A \wedge b \in B\}.$$
\end{dfn}

Se puede demostrar que para cualesquiera $A$ y $B$ el conjunto $A \times B$ existe y está dado de la siguiente manera:
$$ A \times B = \{ (a,b) \in \mathcal{P}(\mathcal{P}(A \cup B)) \mid a \in A \wedge b \in B \} $$

\begin{dfn}[Relación (binaria)]
	Si $A$ y $B$ son conjuntos, se dice que el conjunto $R$ es una \textbf{relación (binaria) de $A$ en $B$} si 
	$$ R \subseteq A \times B.$$
	Si $R \subseteq A \times A$ diremos simplemente que $R$ es una relación en $A$.
	Y denotamos $(x,y) \in R$ como $x \mathrel{R} y$. En caso contrario, escribimos $x \not\mathrel{R} y$ 
\end{dfn}

\begin{ejem}
	\begin{enumerate}[label=\roman*)]
		\item $\varnothing$ y $A \times B$ son relaciones en $A \times B$ (de $A$ en $B$).
		\item $\{(1,1), (2,2), (1,4), (3,4)\}$ no es una relación en $\{1,2,3\}$ pero sí en $\{1,2,3,4\}$.
	\end{enumerate}
\end{ejem}

\begin{dfn}[Relación de asignación total]
	Si $A$ y $B$ son conjuntos y $R$ es una \textbf{relación de $A$ en $B$}, se dice que $R$ es relación de \textbf{asignación total} si
	$$ \forall a \in A \; \exists \; b_a \in B \text{ tal que } (a, b_a) \in R$$
\end{dfn}

\begin{dfn}[Relación de asignación única]
	Si $A$ y $B$ son conjuntos y $R$ es una relación de $A$ en $B$, se dice que $R$ es relación de \textbf{asignación única} si satisface que
	$$ \forall a,b,c \; \text{ si } (a,b) \in R \text{ y } (a,c) \in R \text{ entonces } b = c.$$
\end{dfn}

\begin{dfn}[Función]
	Si $A$ y $B$ son conjuntos y $R$ es una relación de $A$ en $B$, se dice que $R$ es una \textbf{función}, si $R$ es una relación de asignación total y de asignación única.
\end{dfn}

\begin{ejem}
	\begin{enumerate}[label=\roman*)]
		\item $\{(1,3), (2,5)\}$ no es relación de asignación total de $\{1,2,3\}$ en $\{1,2,3,4,5\}$. Pero sí desde $\{1,2\}$.
		\item $\{(1,1),(1,2),(2,1)\}$ no es relación de asignamiento único en $\{1,2\}$. Pero $\{(1,2)$, $ (2,1)\}$ sí lo es.
		\item $\fdef{f}{\real \setminus \{-2,-3\}}{\real}$ con $x \mapsto 1/(x^2-5x+6)$ no es función.
		\item $\fdef{g}{\real \setminus \{2,3\}}{\real}$ con $x \mapsto 1/(x^2-5x+6)$ es función.
	\end{enumerate}
\end{ejem}

\begin{dfn}[Relación reflexiva]
	Si $A$ es un conjunto y $R$ es una relación en $A$ se dice que $R$ es \textbf{reflexiva} si $\forall a \in A$ $(a,a) \in R$.
\end{dfn}

\begin{ejem}
	\begin{enumerate}[label=\roman*)]
		\item La relación de perpendicularidad $\perp$ no es reflexiva.
		\item $\not=$ no es reflexiva.
		\item $\parallel$ depende de la definición.
		\item $a \sim b \bicond n \mid b - a$ es reflexiva.
	\end{enumerate}
\end{ejem}

\begin{dfn}[Relación simétrica]
	Si $A$ es un conjunto y $R$ es una relación en $A$ se dice que $R$ es \textbf{simétrica} si $\forall a,b \in A \text{ si } (a,b) \in R \text{ entonces } (b,a) \in R$.
\end{dfn}

\begin{dfn}[Relación transitiva]
	Si $A$ es un conjunto y $R$ es una relación en $A$ se dice que $R$ es \textbf{transitiva} si $\forall a,b,c \in A \text{ si } (a,b) \in R \text{ y } (b,c) \in R \text{ entonces } (a,c) \in R$.
\end{dfn}

\begin{dfn}[Relación antisimétrica]
	Si $A$ es un conjunto y $R$ es una relación en $A$ se dice que $R$ es \textbf{antisimétrica} si $\forall a,b \in A \text{ si } (a,b) \in R \text{ y } (b,a) \in R$, entonces $a = b$.
\end{dfn}

\begin{dfn}[Relación de orden no estricto]
	Si $A$ es un conjunto y $R$ es una relación en $A$ se dice que $R$ es una \textbf{relación de orden no estricto} si satisface que $\mathrel{R}$
	\begin{enumerate}[label=\roman*), leftmargin=2em]
		\item es una relación reflexiva.
		\item es una relación antisimétrica.
		\item es una relación transitiva.
	\end{enumerate}
\end{dfn}

Para facilitar la escritura se escribe $a \mathrel{R} b$ para $(a,b) \in R$.

\begin{dfn}[Relación de equivalencia]
	Si $A$ es un conjunto y $R$ es una relación en $A, A \not= \varnothing$, se dice que $R$ es una \textbf{relación de equivalencia} si $\mathrel{R}$
	\begin{enumerate}[label=\roman*), leftmargin=2em]
		\item es una relación reflexiva.
		\item es una relación simétrica.
		\item es una relación transitiva.
	\end{enumerate}
\end{dfn}

\begin{dfn}[Clase de equivalencia módulo $R$]
	Si $A$ es un conjunto no vacío y $R$ es una relación de equivalencia en $A$ y $a \in A$, se define la \textbf{clase de equivalencia del elemento $a$ módulo $R$} (denotado $[\,a\,]_R$) como
	$$ [\,a\,]_R = \{x \in A \mid a \mathrel{R} x\}.$$
\end{dfn}

\begin{prop}[Representación múltiple de una clase]
	Si $A$ es un conjunto no vacío, $R$ es una relación de equivalencia en $A$, $a,b \in A$ y $a \mathrel{R} b$, entonces $[\,a\,]_R = [\,b\,]_R$.
	
	\begin{proof}
		$\boxed{\subseteq}$ $[\,a\,]_R \subseteq [\,b\,]_R$.
		Suponga que $a \mathrel{R} b$, por simetría $b \mathrel{R} a$, además si $x \in [\,a\,]_R$ entonces $a \mathrel{R} x$, luego por transitividad $b \mathrel{R} x$. Así $x \in [\,b\,]_R$\\
		
		$\boxed{\supseteq}$ $[\,b\,]_R \subseteq [\,a\,]_R$ se demuestra de fórma análoga.
	\end{proof}
\end{prop}

\begin{prop}[Clases disjuntas]
	Si $A$ es un conjunto no vacío, $R$ una relación de equivalencia en $A$, $a,b \in A$ y $a \not \mathrel{R} b$, entonces $[\,a\,]_R \cap [\,b\,]_R = \varnothing$.
	
	\begin{proof}
		Procedemos por contrapositiva. \\
		Suponga que $[\,a\,]_R \cap [\,b\,]_R \not= \varnothing$, así $\exists x_0 \in [\,a\,]_R, x_0 \in [\,b\,]_R$, así $a \mathrel{R} x_0$ y $b \mathrel{R} x_0$, en particular por simetría $x_0 \mathrel{R} b$, luego por transitividad $a \mathrel{R} b$. Por lo anterior se tiene lo deseado.
	\end{proof}
\end{prop}

\begin{dfn}[Partición y probable partición]
	Si $A$ es un conjunto no vacío y $\{ A_{\alpha} \}_{\alpha \in \Omega}$ una familia no vacía de subconjuntos de $A$, se dice que la colección $\{ A_{\alpha} \}_{\alpha \in \Omega}$ es una \textbf{posible partición} de $A$ si se satisface:
	\begin{enumerate}[label=\roman*)]
		\item $\bigcup\limits_{\alpha \in \Omega} A_{\alpha} = A$.
		\item Si $\alpha,\beta \in \Omega \text{ con } \alpha \not= \beta \text{, entonces } A_{\alpha} \cap A_{\beta} = \varnothing$
		\item Si además se satisface, $\forall \alpha \in \Omega, A_{\alpha} \not= \varnothing$, entonces se dice que $\{ A_{\alpha} \}_{\alpha \in \Omega}$ es una \textbf{partición}.
	\end{enumerate}
\end{dfn}

\begin{prop}[Particiones de conjuntos inducen relaciones de equivalencia]
	Si $A$ es un conjunto no vacío y $\{ A_{\alpha} \}_{\alpha \in \Omega}$ una familia de conjuntos la cual es una partición de $A$, entonces la siguiente relación es una relación de equivalencia.
	$$ x \mathrel{\sim_{\Omega}} y \bicond \exists a \in \Omega \text{ tal que } x,y \in A_{\alpha} $$
	
	\begin{proof}
		\boxed{Reflexiva} Como $\{ A_{\alpha} \}_{\alpha \in \Omega}$ es una partición de $A$, en particular $\bigcup_{\alpha \in \Omega} A_\alpha$, así, $\forall a \in A$ existe $\alpha_a \in \Omega$ tal que $a \in A_{\alpha_a}$, en particular $a \in A_{\alpha_a} \wedge a \in A_{\alpha_a}$. Así $a \mathrel{\sim_{\Omega} a}$. \\
		
		\boxed{\text{\textit{Sim\'etrica}}} Suponga que $x \mathrel{\sim_{\Omega}} y$, entonces existe $\alpha_{xy} \in \Omega$ tal que $x,y \in A_{\alpha_{xy}}$, así $y,x \in A_{\alpha_{xy}}$, luego $y \mathrel{\sim_{\Omega}} x$. \\
		
		\boxed{Transitiva} Suponga que $x \mathrel{\sim_\Omega} y \wedge y \mathrel{\sim_\Omega}$ z, entonces existen $\alpha_{xy}, \alpha_{yz} \in \Omega$ tales que $x,y \in A_{\alpha_{xy}}$ y $y,z \in A_{\alpha_{yz}}$, en particular $y \in A_{\alpha{xy}}, y \in A_{\alpha_{yz}}$.
		
		Como $\{ A_{\alpha} \}_{\alpha \in \Omega}$ es partición, entonces $A_{\alpha_{xy}} = A_{\alpha{yz}}$, así $x,y,z \in A_{\alpha{xy}}$ en particular $x,z \in A_{\alpha_{xy}}$, así $x \mathrel{\sim_\Omega} z$.
	\end{proof}
\end{prop}

\subsection{Definiciones y resultados específicos para nuestro estudio}

\begin{dfn}[Elementos comparables bajo un orden parcial]
	Sea $A$ un conjunto y $\leq$ un orden parcial en $A$. Se dice que $a,b \in A$ son \textbf{comparables} bajo $\leq$ si $a \leq b$ o $b \leq a$.
\end{dfn}

\begin{dfn}[Cadena]
	Sea $A$ un conjunto y $\leq$ un orden en $A$. Se dice que $\mathscr{C}$ es una \textbf{cadena} en $A$ si para cualesquiera $a,b \in A$ los elementos en $\mathscr{C}$ son comparables.
\end{dfn}

\begin{dfn}[Orden total]
	Si $(A, \leq)$ es un conjunto parcialmente ordenado, se dice \textbf{orden total} si $\forall a,b \in A$ $a$ y $b$ son comparables. 
\end{dfn}

\begin{obs}
	Una cadena es un orden total.
\end{obs}

\begin{dfn}[Elemento máximo de un subconjunto de un conjunto parcialmente ordenado]
	Sea $(A, \leq)$ un conjunto parcialmente ordenado, $B \subseteq A$ y $b \in A$, $M \in B$ se dice \textbf{elemento máximo} del conjunto $B$ en el orden parcial $\leq$ si para todo $d \in B$, $d \leq M$.
\end{dfn}

\begin{dfn}[Supremo]
	Si $(A, \leq)$ es un conjunto parcialmente ordenado, $S$ se dice \textbf{supremo} de $B \subseteq A$ en el conjunto parcialmente ordenado $(A, \leq)$ si $S$ es el mínimo del conjunto de cotas superiores.
\end{dfn}

\begin{dfn}[Elemento maximal de un subconjunto de un conjunto parcialmente ordenado]
	Sea $(A, \leq)$ es un conjunto parcialmente ordenado y $B \subseteq A$. $M' \in B$ se dice \textbf{elemento maximal} de $B$ en el orden parcial $\leq$ si no existe $d \in B$ tal que $M' \leq d$, $M' \not= d$.
\end{dfn}

\begin{dfn}[Conjuntos equipotentes]
	Si $A$ y $B$ son conjuntos, se dice que $A$ y $B$ son \textbf{equipotentes} o que tienen la misma cardinalidad (denotando $A \sim B$ o $\card{A} = \card{B}$) si existe $\underset{a \mapsto \phi(b)}{\phi : A \longrightarrow B}$ tal que $\phi$ es biyectiva.
\end{dfn}

\begin{ejem}
	$\mathbb{N} \sim \mathbb{Z}$. En efecto, la función
	$$\fdefc{\phi}{\mathbb{N}}{\mathbb{Z}}{n}{\left\{\begin{array}{ccc}
			n, \text{ si } n = 0 \\
			-\frac{n+1}{2}, \text{ si } n \text{ es impar} \\
			\frac{n}{2}, \text{ si } n \text{ es par} \\
		\end{array}\right.}$$
	es una función biyectiva.
\end{ejem}

\begin{prop} \label{prop:equipotenciaEnV}
	La equipotencia en la clase de todos los conjuntos es una relación de equivalencia.
	\begin{proof}
		Ejercicio.
	\end{proof}
\end{prop}

\begin{lema}[Lema de Zorn] \label{lema:Zorn}
	Sea $A$ es un conjunto no vacío parcialmente ordenado por $\leq$. Para toda cadena $\mathscr{C}$ en $A$, si $\mathscr{C}$ está acotada superiormente entonces $A$ tiene elementos maximales.
	\begin{proof}
		\cite{settheoryhernandez}.
	\end{proof}
\end{lema}

\begin{teo}[Cantor-Bernstein] \label{teo:CantorBernstein}
	Sean $A$ y $B$ conjuntos. Si existen
	$ \underset{a \mapsto \phi_1(a)}{\phi_1 : A \longrightarrow B} $
	función inyectiva y
	$ \underset{b \mapsto \phi_2(b)}{\phi_2 : B \longrightarrow A} $
	función inyectiva, entonces $A \sim B$.
	\begin{proof}
		\cite{settheoryhernandez}.
	\end{proof}
\end{teo}

\begin{teo}[Buen orden]
	Todo conjunto puede ser bien ordenado.
	\begin{proof}
		\cite{settheoryhernandez}.
	\end{proof}
\end{teo}

\begin{dfn}[Conjunto ordenado por un conjunto de índices bien ordenado]
	Si $A$ y $\Omega$ son conjuntos, $\underset{\alpha \mapsto \phi(\alpha) = \alpha_k}{\phi : \Omega \longrightarrow A}$ y $\leq_\omega$ un buen orden en $\Omega$, se dice que el conjunto $A$ está \textbf{ordenado bajo el orden inducido} por $\leq_\omega$ si el conjunto se ordena bajo $\phi(\Omega)$, es decir $\{ \phi(\alpha) \}_{\alpha \in (\Omega, \leq_\omega)}$ es un conjunto bien ordenado.
\end{dfn}

\section{Algunas estructuras algebraicas.}

\begin{dfn}[Operación binaria]
	Una operación binaria $*$ en un conjunto no vacío $S$ es una función $\fdefc{*}{S \times S}{S}{(s,t)}{*((s,t))}$ (y denotamos $*((s,t))$ como $s*t$).
	\begin{itemize}
		\item La operación se dice que es \textbf{asociativa} si $(s \mathrel{*} t) \mathrel{*} r = s \mathrel{*} (t \mathrel{*} r)$ $\forall s,t,r \in S$.
		\item La operación se dice que es \textbf{conmutativa} si $s \mathrel{*} t = t \mathrel{*} s$ $\forall s,t \in S$.
		\item Un elemento $e$ en $S$ es llamado una \textbf{identidad} de $*$ si $s \mathrel{*} e = e \mathrel{*} s = s$ $\forall s \in S$.
		\item Si $*$ tiene un elemento identidad $e$, y $s \in S$, entonces $l \in S$ se dice un \textbf{inverso} para $s$ si $s \mathrel{*} l = l \mathrel{*} s = e$.
	\end{itemize}
\end{dfn}

\begin{dfn}[Grupo]
	Una estructura $(G, *)$ se dice grupo si satisface
	\begin{enumerate}[label=\roman*), leftmargin=2em]
		\item $G$ es un conjunto no vacío.
		\item $\fdefc{*}{G \times G}{G}{(g,h)}{*((g,h)):=g\mathrel{*}h}$ es una operación binaria la cual satisface
		\begin{enumerate}[label=\alph*), leftmargin=2em]
			\item \textbf{Asociatividad.} $\forall a,b,c \in G$ $(a \mathrel{*} b) \mathrel{*} c = a \mathrel{*} (b \mathrel{*} c)$.
			\item \textbf{Identidad.} Existe un \textbf{elemento identidad} $e \in G$, esto es,
			
			$\forall a \in G$ $e \mathrel{*} a = a \mathrel{*} e = a$.
			\item \textbf{Inversos.} Para cada $a \in G$ existe un elemento \textbf{inverso} $a^{-1} \in G$, esto es, un elemento $a^{-1} \in G$ tal que
			
			$a \mathrel{*} a^{-1} = a^{-1} \mathrel{*} a = e$.
		\end{enumerate}
		
		Si además cumple
		\begin{enumerate}[label=\alph*), leftmargin=2em]
			\setcounter{enumii}{3}
			\item \textbf{Conmutatividad.} $\forall a,b \in G$ $a \mathrel{*} b = b \mathrel{*} a$.
		\end{enumerate}
		
		Se dice \textbf{grupo abeliano}.
	\end{enumerate}
\end{dfn}
%\end{tcolorbox}

%\begin{tcolorbox}[title=Anillo, colback=backanillo, boxrule=0.0mm, arc=0.2mm, breakable, enhanced, colframe=backanillo, sharp corners, coltitle=white, colbacktitle=colanillo, fonttitle=\scshape, borderline={0.3mm}{0.5mm}{colanillo}, attach boxed title to top center={xshift=0mm, yshift=-2mm}]
\begin{dfn}[Anillo]
	Una estructura $(R, +, \cdot)$ se dice anillo si satisface
	\begin{enumerate}[label=\roman*), leftmargin=2em]
		\item $(R, +)$ es un grupo abeliano.
		\item $\fdefc{\cdot}{R \times R}{R}{(a,b)}{\cdot((a,b)):=a\mathrel{\cdot}b}$ es una operación binaria asociativa.
		\item $+$ y $\cdot$ satisfacen las propiedades distributivas, es decir
		\begin{enumerate}[label=\alph*), leftmargin=2em]
			\item $a \cdot (b + c) = a \cdot b + a \cdot c$ $\forall a,b,c \in R$.
			\item $(a + b)\cdot c = a \cdot c + b \cdot c$ $\forall a,b,c \in R$.
		\end{enumerate}
		\begin{itemize}
			\item El anillo se dice \textbf{conmutativo} si $\forall a,b \in R$ $a \cdot b = b \cdot a$.
			\item Se dice \textbf{anillo con identidad} si existe un $1_R \in R$ tal que $\forall a \in R$ $1_R \cdot a = a \cdot 1_R = a$.
			\item Se dice \textbf{anillo conmutativo con identidad} si $R$ es un anillo conmutativo y es un anillo con identidad.
			\item El anillo se dice \textbf{dominio entero} si es anillo conmutativo con identidad y satisface la propiedad
			
			$\forall a,b \in R$ si $a \cdot b = 0_R$ entonces $a = 0_R$ o $b = 0_R$.
			\item El anillo se dice \textbf{campo} si, además de ser anillo conmutativo con identidad, satisface
			
			$\forall a \in R \setminus \{0_R\}$ existe un $a^{-1} \in R$ tal que $a \cdot a^{-1} = a^{-1} \cdot a = 1_R$.
		\end{itemize}
	\end{enumerate}
\end{dfn}
%	\end{tcolorbox}

%\begin{tcolorbox}[title=Subanillo, colback=backsubanillo, boxrule=0.0mm, arc=0.2mm, breakable, enhanced, colframe=backsubanillo, sharp corners, coltitle=white, colbacktitle=colsubanillo, fonttitle=\scshape, borderline={0.3mm}{0.5mm}{colsubanillo}, attach boxed title to top center={xshift=0mm, yshift=-2mm}]
\begin{dfn}[Subanillo]
Si $R$ es un anillo, un conjunto $S$ se dice \textbf{subanillo} de $R$ si $S \subseteq R$ y $S$ es un anillo.
\end{dfn}
%\end{tcolorbox}
Los subanillos $S \subseteq R$ $(S, +_S, \cdot_S)$ tienen como operaciones las respectivas restricciones de las operaciones del anillo $(R, +, \cdot)$ a $S$.

\begin{dfn}[Grupo de unidades]
Si $A$ es una estructura algebraica, a 
$$U_R = \{u \in A \mid u \text{ tiene inverso } \}$$
se le llama \textbf{grupo de unidades} de $A$.
\end{dfn}

%\begin{tcolorbox}[title=Ideal, colback=backideal, boxrule=0.0mm, arc=0.2mm, breakable, enhanced, colframe=backideal, sharp corners, coltitle=white, colbacktitle=colideal, fonttitle=\scshape, borderline={0.3mm}{0.5mm}{colideal}, attach boxed title to top center={xshift=0mm, yshift=-2mm}]
\begin{dfn}[Ideal]
Si $R$ es un anillo, un conjunto $I$ se dice \textbf{ideal} si $I$ es un subanillo de $R$ tal que $\forall a \in R$ y $\forall i \in I$, $a \cdot i \in I$ y $i \cdot a \in I$.
\end{dfn}
%\end{tcolorbox}

La siguiente es la generalización de la estructura espacio vectorial.

\begin{dfn}[Módulo izquierdo]
	Sea $R$ un anillo. Una estructura $(M, +_M, \cdot_M, R)$ se dice \textbf{módulo izquierdo} si satisface
	\begin{enumerate}[label=\roman*), leftmargin=2em]
		\item $(M, +_M)$ es un grupo abeliano.
		\item $\fdefc{\cdot_M}{R \times M}{M}{(a,m)}{\cdot_M((a,m)):=a\mathrel{\cdot_M}m}$ es una función que satisface
		\begin{enumerate}[label=\alph*), leftmargin=2em]
			\item $\forall a,b \in R$ y $\forall m \in M$
			$(a \cdot b) \cdot_M m = a \cdot_M (b \cdot_M m)$.
			\item $\forall a, b \in R$ y $\forall m \in M$ $(a + b) \cdot_M m = a \cdot_M m +_M b \cdot_M m$.
			\item $\forall a \in R$ y $\forall m,n \in M$ $a \cdot_M (m +_M n) = a \cdot_M m +_M a \cdot_M n$.
			\item $\forall m \in M$ existe un $1_M \in R$ tal que $1_M \cdot_M m = m$.
		\end{enumerate}
	\end{enumerate}
\end{dfn}
%\end{tcolorbox}

También se puede definir el \textbf{módulo derecho}, la razón por la que se enuncia la definición de módulo izquierdo es porque usualmente se conviene realizar la operación $\cdot_M$ desde la izquierda.

\section{Ejercicios}
\begin{enumerate}[label=\arabic*.]
	\item Demostrar que los conjuntos determinados por A.1, A.3, A.4, A.5 y A.6 son únicos.
	\item Demostrar la Proposición \ref{prop:equipotenciaEnV}.
	\item Encontrar todas las relaciones en $\{1,2,3\}$ y determinar de qué tipo son.
	\item Determine cuáles de las siguientes son relaciones de equivalencia:
	\begin{enumerate}[label=\alph*)]
		\item $\in_{X} := \{(x,y) \in X \times X \mid x \in y\}$ donde $X = \{\varnothing, \{\varnothing\}, \{\{\varnothing\}\}\}$.
		\item $\{(1,1), (2,2), (3,3), (1,2), (2,3)\}$ en $\{1,2,3\}$.
		\item $\ls$ en $\real$.
		\item $\not=$ en $\real$.
	\end{enumerate}
	\item ¿Existe algún conjunto $X$ donde $\in_X$ sea una relación de equivalencia? Para esto considere los casos i) afirmamos el axioma de fundación: $$\text{A.8) }A \not= \varnothing \cond \exists x \in A \forall y \in A (y \not\in x)$$ y ii) cuando lo negamos.
	\item Demostrar que el conjunto $\mathcal{P}_g = \{y \in \mathcal{F} \mid \frac{d^2}{dx^2}y(x) + a_1\frac{d}{dx}y(x) + a_0 y(x) = 0\}$ con la operación usual de $+$ forma un grupo abeliano.
	\item Demostrar que el conjunto $\mathcal{F}_{\mathscr{I}} = \{f \in \mathcal{F} \mid f \text{ es integrable en } [a,b]\}$ junto con la operación usual de $+$ forma un grupo abeliano.
	\item \begin{enumerate}[label=\alph*), leftmargin=2em]
		\item Utilizar el teorema fundamental de la aritmética para encontrar una inyección de $\mathbb{N}$ a $\mathbb{Q}^{+}$ y de $\mathbb{Q}$ a $\mathbb{N}$.
		\item Demostrar que $\mathbb{N} \sim \mathbb{Q}$
	\end{enumerate}
	
\end{enumerate}