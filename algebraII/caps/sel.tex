\chapter{Sistemas de ecuaciones lineales}

\section{Sistemas de ecuaciones lineales}

Nos quedamos con la idea intuitiva de incógnita, y denotamos por
$$ E_{CR_{x_1,\ldots,x_n}} $$
a la colección de ecuaciones lineales con coeficientes en el anillo $R$ y con incógnitas ordenadas $(x_1, \ldots, x_n)$.

Las expresiones de las incógnitas tienen potencia 1 y no se multiplican entre sí.

Siendo el caso de que los coeficientes estén en un campo $F$, entonces escribimos $ E_{CF_{x_1,\ldots,x_n}} $ y se hacen las sustituciones adecuadas en las definiciones.

Se identifica la colección de ecuaciones lineales con coeficientes en el anillo $R$ y con incógnitas ordenadas $(x_1, \ldots, x_n)$ como
$$R^n \times R.$$
Convenimos en escribir la operación $\cdot$ de un anillo $R$ de manera que para $a,b \in R$ $a \cdot b = ab$.
Luego, una ecuación en el anillo $R$ de la forma $a_1 x_1 + a_2 x_2 + \cdots + a_n x_n = b$ se representa como $((a_1,a_2, \ldots, a_n),b)$

\begin{dfn}[Sistema de $m$ ecuaciones lineales con $n$ incógnitas]
	Un sistema $A$ de $m$ ecuaciones lineales con $n$ incógnitas ordenadas $(x_1, \ldots, x_n)$ con coeficientes en un anillo $R$ es una función
	$$\fdefc{A}{\inte{1}{m}}{E_{CR_{x_1, \ldots, x_n}}}{i}{a_{i1}x_1 + \cdots + a_{in}x_n = y_i}$$
	denotado por
	\sisecn{A}{x}
\end{dfn} 

\begin{dfn}[Solución de un sistema de ecuaciones lineales] Sea $A$ un sistema de $m$ ecuaciones lineales con $n$ incógnitas ordenadas $(x_1, \ldots, x_n)$ con coeficientes un anillo $R$, se dice que $\mathcal{S}_A \subseteq R^n$ es una \textbf{solución del sistema} $A$ si
	$$ \forall (\alpha_1, \ldots, \alpha_n) \in \mathcal{S}_A $$
	\sisec{\alpha}
	
	es decir, $\mathcal{S}_A = \{ (\alpha_1, \ldots, \alpha_n) \in R^n \mid (\alpha_1, \ldots, \alpha_n) \text{ satisface cada ecuación de } A \}$
\end{dfn}

\begin{dfn}[Sistema de ecuaciones lineales homogéneo]
	Si $A$ es un sistema de $m$ ecuaciones lineales con $n$ incógnitas ordenadas $x_1, \ldots, x_n$ con coeficientes en un anillo $R$
	
	\sisecn{A}{x}
	
	se dice \textbf{homogéneo} si $\forall i \in \inte{1}{m} \;\; y_i = 0$.
\end{dfn}

Si $A$ es un sistema de $m$ ecuaciones lineales con $n$ incógnitas ordenadas $x_1, \ldots, x_n$ con coeficientes en un anillo $R$. Denotamos por
$$ E_{CR_{x_1, \ldots, x_n}A} \text{ a } A(\inte{1}{m})$$

\begin{lema} \label{lema:solucionDeAEsLaInterseccoinDeSoluciones}
	Sea $A$ un sistema de ecuaciones lineales con $n$-incógnitas ordenadas $(x_1,\ldots$, $x_n)$ con coeficientes en un anillo $R$, entonces
	$$\mathcal{S}_A = \bigcap\limits_{i = 1}^{m} \mathcal{S}_{A(i)}$$
	
	donde $\mathcal{S}_{A(i)}$ denota la solución de la $i$-ésima ecuación de $A$.
	
	\begin{proof}
		Ejercicio.
	\end{proof}
\end{lema}

\begin{lema} \label{lema:AContBEntSolBContenSolA}
	Si $A$ y $B$ son dos sistemas de ecuaciones lineales con $n$-incógnitas ordenadas $(x_1, \ldots, x_n)$; $m_A$ el número de ecuaciones del sistema $A$; $m_B$ el número de ecuaciones del sistema $B$, con coeficientes en un anillo $R$ tales que $E_{CR_{x_1, \ldots, x_n}A} \subseteq E_{CR_{x_1, \ldots, x_n}B}$, entonces
	$$ \mathcal{S}_B \subseteq \mathcal{S}_A$$
	
	\begin{proof}
		Ejercicio.
	\end{proof}
\end{lema}

\begin{prop} \label{prop:IgualdadDeSoluciones}
	Si $A$ y $B$ son dos sistemas de ecuaciones lineales con $n$-incógnitas ordendas $(x_1, \ldots, x_n)$; $m_A$ el número de ecuaciones del sistema $A$; $m_B$ el número de ecuaciones del sistema $B$, con coeficientes en un anillo $R$ tales que $E_{CR_{x_1, \ldots, x_n}A} = E_{CR_{x_1, \ldots, x_n}B}$, entonces
	$$ \mathcal{S}_B = \mathcal{S}_A$$
	
	\begin{proof}
		Ejercicio.
	\end{proof}
\end{prop}

\begin{dfn}[Combinación lineal de ecuaciones lineales]
	Sea $A$ un sistema de $m$ ecuaciones lineales con $n$-incógnitas ordendas $(x_1, \ldots, x_n)$ con coeficientes en un anillo $R$, se dice que la ecuación
	$d_1x_1 + \ldots + d_nx_n = y$ es \textbf{combinación lineal} de las ecuaciones del sistema:
	\sisecn{A}{x}
	si existen $c_1, \ldots, c_n \in R$ tales que
	$$ d_i = \sum_{j = 1}^{m} c_ja_{ji} \text{ y } y = \sum_{j=1}^{m} c_jy_{j} $$
\end{dfn}

\begin{lema} \label{lema:SolANoCambiaBajoIntercambio}
	Sean $A, A_{i \leftrightarrow j}$ sistemas de $m$ ecuaciones lineales de $n$-incógnitas ordenadas $(x_1, \ldots, x_n)$ en un anillo $R$ tales que $A(i) = A_{i \leftrightarrow j}(j)$ y $A(j) = A_{i \leftrightarrow j}(i)$ y $A(k) = A_{i \leftrightarrow j}(k) \; \forall k \in \inte{1}{m} \setminus \{i,j\}$, entonces
	$$ S_A = S_{A_{i \leftrightarrow j}} $$
	
	\begin{proof}
		Ejercicio.
	\end{proof}
\end{lema}

\begin{dfn}(Operaciones de ecuaciones lineales)
	Sea $R$ un anillo. Entonces definimos lo siguiente:
	
	$$\fdef{+}{E_{CR_{x_1, \ldots, x_n}} \times E_{CR_{x_1, \ldots, x_n}}}{E_{CR_{x_1, \ldots, x_n}}}$$
	{\small$$\asign{(a_{i1}x_1 + \ldots + a_{in}x_n = y_i, b_{j1}x_1 + \ldots + b_{jn}x_n = y_j)}{((a_{i1} + b_{j1})x_1 + \ldots + (a_{in} + b_{jn})x_n = y_i + y_j)}$$}
	
	$$\fdef{\cdot}{R \times E_{CR_{x_1, \ldots, x_n}}}{E_{CR_{x_1, \ldots, x_n}}}$$
	$$\asign{(c, a_{i1}x_1 + \ldots + a_{in}x_n = y_j)}{ca_{i1}x_1 + \ldots + ca_{in}x_n = cy_i}$$
	
	Entendiéndose que la operación aditiva en la asignación es parte del anillo $(R, +, \cdot)$.
\end{dfn}

Se entenderá desde ahora al mencionar un sistema de $m$ ecuaciones lineales con $n$-incógnitas que esté tiene incógnitas ordenadas (dígase $(x_1, \ldots, x_n)$) y tiene coeficientes en un anillo $R$ o un campo $F$.

\begin{prop}
	Sean $A, A_{i \rightarrow c \cdot i}$ sistemas de ecuaciones lineales con $n$-incógnitas tales que
	$$ \forall j \in \inte{1}{m} \setminus \{i\} \;\;\; A(j) = A_{i \rightarrow c \cdot i}(j) \text{ y } c \cdot A(i) = A_{i \rightarrow{c} \cdot i}(i), c \in \mathcal{U}_R $$
	$$ \text{entonces } \mathcal{S}_A = \mathcal{S}_{A_{i \rightarrow c \cdot i}} $$
	
	Basta demostrar que $\mathcal{S}_{A(i)} = \mathcal{S}_{A_{i \to c \cdot i}}$, en efecto. \\
	
	\begin{proof}
		\boxed{\subseteq} Suponga que la $n$-ada $(d_1, \ldots, d_n) \in R^n$ sea solución de $A(i)$. Es decir,
		$$ \sum_{j=1}^n a_{ij}d_j = y_i, \text{ luego } c \cdot \sum_{j=1}^n a_{ij}d_j = c \cdot y_i$$
		$$ \therefore (d_1, \ldots, d_n) \in \mathcal{S}_{A_{i \to c \cdot i}(i)}$$
		$$\text{luego, } \mathcal{S}_{A(i)} \subseteq \mathcal{S}_{A_{i \to c \cdot i}(i)}$$
		
		\boxed{\supseteq} Suponga que la $n$-ada $(d_1, \ldots, d_n) \in S_{A_{i \to c \cdot i}(i)}$, entonces
		$$ c \cdot \sum_{j=1}^n a_{ij}d_j = c \cdot y_i $$
		$$ \text{como } c \in \mathcal{U}_R \text{, entonces } c^{-1}c \cdot \sum_{j=1}^n a_{ij}d_j = c^{-1}c \cdot y_i, \text{ así} $$
		$$ \sum_{j = 1}^n a_{ij}d_j = y_i \text{ es decir } (d_1, \ldots, d_n) \in \mathcal{S}_{A(i)} $$
		$$ \text{luego } \mathcal{S}_{A_{i \to c \cdot i}(i)} \subseteq \mathcal{S}_{A(i)}$$
		
		De la doble contención, $\mathcal{S}_{A(i)} = \mathcal{S}_{A_{i \to c \cdot i}(i)}$
	\end{proof}
\end{prop}

\begin{prop} \label{prop:LaOtraContencion}
	Sean $A$, $A_{i \to i + c \cdot j}(i)$ sistemas de ecuaciones lineales con $n$-incógnitas tales que:
	$$ \forall j \in \inte{1}{m} \setminus \{i\} \;\; A(j) = A_{i \to i + c \cdot j}(j) \text{ y } A(i) + c \cdot A(j) = A_{i \to i + c \cdot j}(i), c \in \mathcal{U}_R$$
	$ \text{entonces } \mathcal{S}_{A} = \mathcal{S}_{A_{i \to i + c \cdot j}} $\\
	
	Basta demostrar que $\mathcal{S}_{A(i)} \cap \mathcal{S}_{A(j)} = \mathcal{S}_{A_{i \to i + c \cdot j}(i)} \cap \mathcal{S}_{A_{i \to i + c \cdot j}(j)}$ 
	
	\begin{proof}
		\boxed{\subseteq} Suponga que $(d_1, \ldots, d_n) \in \mathcal{S}_{A(i)} \cap \mathcal{S}_{A(j)}$.
		$$ \sum_{k = 1}^{n} a_{ik}d_k = y_i, c\sum_{k = 1}^{n} a_{jk}d_k = cy_j, \text{ en particular}$$
		$$ \sum_{k = 1}^{n} (a_{ik} + ca_{jk})d_k = y_i + y_j $$
		$$ \text{entonces } (d_1, \ldots, d_n) \in \mathcal{S}_{A_{i \to i + c \cdot j}(i)} \cap \mathcal{S}_{A_{i \to i + c \cdot j}(j)}.$$
		La demostración de \boxed{\supseteq} se queda como ejercicio.
	\end{proof}
\end{prop}

\begin{dfn}[Sistemas equivalentes]
	Dos sistemas $A,B$ de ecuaciones lineales con $n$-incógnitas se dicen sistemas equivalentes ($A \equiv B$) si cada ecuación del sistema $A$ es combinación lineal de las ecuaciones del sistema $B$ y cada ecuación del sistema $B$ es combinación lineal de las ecuaciones del sistema $A$.
\end{dfn}

\begin{lema}
	Si $A$ es un sistema de ecuaciones lineales con $n$-incógnitas y $C_{la}$ una combinación lineal de las ecuaciones del sistema $A$. Entonces $\mathcal{S}_A \subseteq \mathcal{S}_{Cl_A}$
	
	\begin{proof}
		Considere
		\sisecn{A}{x}
		Sea $\sum\limits_{k=1}^{m} (d_k[a_{k1}x_1 + \cdots + a_{kn}x_n]) = \sum\limits_{k=1}^{m} d_k y_k$ una combinación lineal de las ecuaciones del sistema $A$. Si $(\alpha_1, \ldots, \alpha_n) \in \mathcal{S}_A$, entonces $\forall i \in \inte{1}{m}$ $(\alpha_1, \ldots, \alpha_n) \in \bigcap\limits_{l=1}^{m} \mathcal{S}_{A(i)}$, luego en particular $\forall i \in \inte{1}{m}$ $(\alpha_1, \ldots, \alpha_n) \in \mathcal{S}_{A(i)}$. En particular $(\alpha_1, \ldots, \alpha_n) \in \mathcal{S}_{c \cdot A(i)}$ $\forall i \in \inte{1}{m}$, así $(\alpha_1, \ldots, \alpha_n)$ es solución para $Cl_A$. ¨Por tanto $\mathcal{S}_A \subseteq \mathcal{S}_{Cl_A}$.
	\end{proof}
\end{lema}

\begin{teo} \label{teo:IgualdadDeSoluciones}
	Si $A$ y $B$ son sistemas de ecuaciones lineales con $n$-incógnitas y coeficientes en el anillo conmutativo con identidad $R$, tales que $A \equiv B$, entonces
	$$ \mathcal{S}_A = \mathcal{S}_B $$
	
	\begin{proof}
		Ejercicio.
	\end{proof}
\end{teo}

\section{Matrices}

\begin{dfn}[Matriz]
	Sea $R$ un anillo. Una matriz es una función
	$$\fdefc{A}{\inte{1}{m} \times \inte{1}{n}}{R}{(i,j)}{A((i,j)):=a_{ij}}$$
	y denotamos
	\arrmatriz{A}{a}
\end{dfn}

\begin{dfn}[Orden de una matriz]
	Decimos que el orden de una matriz es el número de filas por el número de columnas, esto es:
	$$ m \times n$$
\end{dfn}
Bajo esta definición, una matriz de orden $m \times n$ es de distinto orden a una matriz $n \times m$. Denotamos el conjunto de matrices de $m$ filas con $n$ columnas con coeficientes en el anillo $R$ como
$$ \Mmnc{m}{n}{R} = \{ A \in \mathcal{U} \mid A \text{ es una matriz de orden } m \times n\} .$$
Si la matriz es de orden $n \times n$, escribimos entonces $\Mnc{n}{R}$.

\begin{dfn}[Operaciones con matrices]
	Definimos lo siguiente:
	$$\fdef{+}{\Mmnc{m}{n}{R} \times \Mmnc{m}{n}{R}}{\Mmnc{m}{n}{R}}$$
	$$ (A((i,j)), B((i,j))) \mapsto C((i,j)) := c_{ij} = a_{ij} + b_{ij} \;\; \forall (i,j) \in \inte{1}{m} \times \inte{1}{n} $$
	
	$$\fdef{\cdot_{R}}{R \times \Mmnc{m}{n}{R}}{\Mmnc{m}{n}{R}}$$
	$$ (\alpha, A((i,j))) \mapsto \alpha A((i,j)) := \alpha \cdot_R (a_{ij}) = \alpha \cdot_R a_{ij} \;\; \forall (i,j) \in \inte{1}{m} \times \inte{1}{n} $$
	
	$$\fdef{\cdot}{\Mmnc{m}{l}{R} \times \Mmnc{l}{n}{R}}{\Mmnc{m}{n}{R}}$$
	$$ ((A,B)) \mapsto \cdot ((A,B)) := C, c_{ij} = \sum_{k = 1}^l a_{ik}b_{kj} $$
\end{dfn}

\begin{dfn}[Matriz aumentada]
	Si $A \in \Mmnc{m}{n}{R}$, se dice que $B$ es la \textbf{matriz aumentada} de $A$ con vector resultado $\begin{pmatrix} y_1\\ \vdots \\ y_m \end{pmatrix}$ si
	$$ B \in \Mmnc{m}{n+1}{R} \text{ tal que } \forall (i,j) \in \inte{1}{m} \times \inte{1}{n} $$
	$$ A((i,j)) = B((i,j)) .$$
\end{dfn}

A cada sistema de $m$ ecuaciones lineales con $n$-incógnitas
\sisecn{A}{x}
Se le asigna la matriz aumentada
\arrmatrizau{A}{a}{y}

\subsection{Operaciones elementales}

\begin{dfn}
	Una operación elemental de matrices de orden $m \times n$ con coeficientes en el anillo $R$ es una función
	$$\fdefc{e}{\Mmnc{m}{n}{R}}{\Mmnc{m}{n}{R}}{A}{e(A)}$$
	donde $e(A)$ es la representación de un sistema de equivalente al sistema $A_{EC}$ que es representado por la matriz $A$ el cual proviene de una operación elemental de sistemas.
\end{dfn}

\begin{itemize}
	\item Sea $A \in \Mmnc{m}{n}{R}$ y sea $A_{EC}$ el sistema que representa. Entonces el sistema $A_{EC i \leftrightarrow j}$ lo representa la matriz $e_{1}(A)$ donde se intercambian los renglones $i$ y $j$ de la matriz $A$.
	\item Sea $A \in \Mmnc{m}{n}{R}$ y sea $A_{EC}$ el sistema que representa. Entonces el sistema $A_{EC i \rightarrow c \cdot i}$ lo representa la matriz $e_{2}(A)$ donde la fila $i$ se cambia por la fila que resulta de operar $c \cdot a_{ij}$ $\forall j \in \inte{1}{n}$, $c \in R$.
	\item Sea $A \in \Mmnc{m}{n}{R}$ y sea $A_{EC}$ el sistema que representa. Entonces el sistema $A_{EC i \rightarrow i + c \cdot j}$ lo representa la matriz $e_{3}(A)$ donde la fila $i$ se cambia por la fila que resulta de operar $a_{ik} + c \cdot a_{jk}$ $\forall k \in \inte{1}{n}$, $c \in R$.
\end{itemize}

\begin{obs}
	Los sistemas homogéneos son invariantes bajo operaciones elementales.
\end{obs}

\begin{ejem}
	Lugar de trabajo: $\mathbb{R}$.
	
	Si $A_{EC} = 
	\begin{sde}{4}
		3x &+& y &=& 7 \\
		2x &-& y &=& 8
	\end{sde}
	$\\\\
	
	luego $A = 
	\begin{matrizau}{2}
		3 & 1 & 7 \\
		2 & -1 & 8
	\end{matrizau}
	$, entonces $e_1(A) = 
	\begin{matrizau}{2}
		2 & -1 & 8 \\
		3 & 1 & 7
	\end{matrizau}
	$\\\\
	
	Si $A_{EC 2 \to c \cdot 2} =
	\begin{sde}{4}
		3x &+&´y &=& 7 \\
		2cx &-& cy &=& 8c
	\end{sde}
	$\\\\
	
	entonces $e_2(A) = 
	\begin{matrizau}{2}
		3 & 1 & 7 \\
		2c & -c & 8c
	\end{matrizau}
	$\\\\
	
	Si $A_{EC1 \to 1 + c \cdot 2} =
	\begin{sde}{4}
		(3+2c)x &+& (1-c)y &=& 7 + 8c \\
		2x &-& y &=& 8 \\
	\end{sde}
	$\\\\
	
	entonces, $e_3(A) = 
	\begin{matrizau}{2}
		3 + 2c & 1 - c & 7 + 8c \\
		2 & -1 & 8
	\end{matrizau}
	$
\end{ejem}

Desde ahora nos referiremos a las operaciones elementales $e_1, e_2, e_3$ con el respectivo cambio en el subíndice para facilitar la lectura. Entonces si el sistema de ecuaciones lineales $A_{EC}$ es representado por la matriz $A$, a $A_{EC i \leftrightarrow j}$ se le representa con la matriz $e_{i \leftrightarrow j}(A)$

Sea $e_{i \leftrightarrow j}$ la operación elemental de matrices de orden $m \times n$ con coeficientes en $R$, entonces su operación inversa es la misma operación elemental.\\

Sea $e_{i \to c \cdot i}$ la operación elemental de matrices de orden $m \times n$ con coeficientes en $R$ y $c \in R$, si $c \in \mathcal{U}_R$ entonces su operación inversa es la matriz elemental $e_{i \to c^{-1} \cdot i}$.\\

Sea $e_{i \to i + c \cdot j}$ la operación elemental de matrices de orden $m \times n$ con coeficientes en $R$, entonces su operación inversa es la operación elemental $e_{i \to i + (-c) \cdot j}$.

\begin{obs}
	Si $c \in \mathcal{U}_R$ entonces $-c \in \mathcal{U}_R$
\end{obs}

\subsection{Matrices equivalentes por filas}

\begin{dfn}[Matrices equivalentes por filas]
	Si $A \in \Mmnc{m}{n}{R}$, se dice que $A$ es equivalente por filas a $B \in \Mmnc{m}{n}{R}$ si existe una secuencia finita $e_1, \ldots, e_l$ de operaciones elementales por fila tal que $e_l \circ \cdots \circ e_1 (A) = B$.
\end{dfn}

\begin{obs}
	Si $c \in \mathcal{U}_R$ entonces $-c \in \mathcal{U}_R$
\end{obs}

\begin{prop}\label{prop:EquivMatricesEsRelacionDeEquiv}
	La equivalencia de matrices en $\Mmnc{m}{n}{R}$ es una relación de equivalencia.
	
	\begin{proof}
		Ejercicio.
	\end{proof}
\end{prop}

Las matrices equivalentes tienen el mismo orden, pero lo sistemas equivalentes no necesariamente tienen el mismo orden.

\begin{ejem}
	Lugar de trabajo: $\mathbb{R}$.
	
	Si $A = 
	\begin{sde}{4}
		x &+& y &=& 2 \\
		x &-& y &=& 0
	\end{sde}
	$ y $B = 
	\begin{sde}{4}
		2x &+& 2y &=& 4\\
		5x &-& 5y &=& 0\\
		7x &-& 3y &=& 4
	\end{sde}$,
	$A \equiv B$, entonces tomamos $\text{máx}\{\inte{1}{m_A}, \inte{1}{m_B}\}$ y construimos el sistema $A^{*} = 
	\begin{sde}{4}
		x &+& y &=& 2 \\
		x &-& y &=& 0 \\
		0x &+& 0y &=& 0
	\end{sde}$\\\\
	
	y entonces manejamos las matrices representativas de $A^{*}$ y $B$.
\end{ejem}

\begin{dfn}[Matriz escalón reducida por filas]
	Sea $R$ un anillo, $A \in \Mmnc{m}{n}{R}$ se dice que $A$ es una matriz escalón reducida por filas si satisface
	\begin{itemize}
		\item $A$ es la matriz $\mathbf{0}_{m \times n}(R)$.
		\item Si $A$ no es la matriz $\mathbf{0}_{m \times n}(R)$ y satisface:
		\begin{enumerate}[label=\roman*), leftmargin=2em]
			\item El primer elemento no nulo de cada fila no nula es igual a $1_R$.
			\item Cada columna de $A$ que tiene el primer elemento no nulo de alguna fila tiene todos sus otros elementos en $0_R$.
			\item 
			\begin{enumerate}[label=\alph*), leftmargin=2em]
				\item Toda fila de $A$ que tiene todos sus coeficientes cero está por debajo de toda fila que tenga algún coeficiente no nulo.
				\item Si las filas $1, \ldots, r$ son las filas no nulas de $A$ y si el primer elemento no nulo de la fila $i$ está en la columna $k_i$, entonces $k_1 < \cdots < k_r$.
			\end{enumerate}
		\end{enumerate}
	\end{itemize}
\end{dfn}

\begin{teo}
	Toda matriz $A \in \Mmnc{m}{n}{R}$ es equivalente por filas a una matriz escalón reducida por filas.\\
	
	La demostración de este teorema será "platicada'' ya que no es viable demostrar el teorema de una manera más formal. Entonces usaremos el \textbf{razonamiento inductivo}.
	
	\begin{proof}
		Si $A = \mathbf{0}_{m \times n}(R)$ entonces $A$ es escalón reducida. Luego $A$ es equivalente por filas a $A$.
		\begin{enumerate}[label=\roman*), leftmargin=2em]
			\item Si $A$ tiene filas nulas, al hacer cambios adecuados se pueden llevar, bajo operaciones elementales por fila, a ocupar las últimas filas. Así $A$ es equivalente por filas a esta matriz resultante $A_{Red}$.
			\item Por medio de una cantidad finita de operaciones elementales por fila se puede colocar el primer elemento no nulo de cada fila de $A_{Red}$ en un orden adecuado. Así, esta matriz $A_{Red_{2}}$ es equivalente por filas a $A_{Red}$, luego a $A$.
			\item En el caso de que se pueda realizar la multiplicación por inversos multiplicativos (haciendo un análisis detallado en anillos) se multiplica el primer elemento no nulo de la primera fila de la matriz por su inverso multiplicativo, después, los elementos de la columna donde está ese elemento se hacen cero. Esto se hace con cada elemento no nulo de cada fila siempre y cuando se pueda multiplicar por el inverso.
		\end{enumerate}
	\end{proof}
\end{teo}

\begin{ejem}
	Lugar de trabajo: $\mathbb{R}$.
	
	$ A =
	\begin{matrizau}{2}
		2 & 3 & 8 \\
		4 & 5 & 6
	\end{matrizau}
	$
	$
	\begin{array}{c}
		R_2 \to R_2 - 2R_1 \\
		R_2 \cdot (-1)
	\end{array}
	$ 
	$
	\begin{matrizau}{2}
		2 & 3 & 8 \\
		0 & 1 & 10
	\end{matrizau}
	$
	$
	\begin{array}{c}
		R_1 \to R_1 - R_2  \\
		R_1 \cdot \frac{1}{2} 
	\end{array}
	$
	$
	\begin{matrizau}{2}
		1 & 0 & -1 \\
		0 & 1 & 10
	\end{matrizau}
	$
\end{ejem}

\begin{ejem}
	Lugar de trabajo: $(\mathcal{P}(X), \difsim, \cap)$, $X = \{1,2,3\}$.
	
	$\mathcal{A} = 
	\begin{sde}{4}
		\{1,2\} \cap \alpha &\difsim& \{X\} \cap \beta &=& \{1\} \\
		\varnothing \cap \alpha &\difsim& \{2,3\} \cap \beta &=& \{2\}
	\end{sde}$.
	
	Construimos el sistema $\mathcal{A}^{*} = 
	\begin{sde}{4}
		X \cap \alpha &\difsim& \{1,2\} \cap \beta &=& \{1\} \\
		\{2,3\} \cap \alpha &\difsim& \varnothing \cap \beta &=& \{2\}
	\end{sde}.
	$luego\\\\
	
	$A = 
	\begin{matrizau}{2}
		X & \{1,2\} & \{1\} \\
		\{2,3\} & \varnothing & \{2\}
	\end{matrizau}
	$
	$
	\begin{array}{c}
		R_2 \to R_2 \mathrel{\difsim} \{2,3\} \cap R_1 \\
	\end{array}
	$
	$\begin{matrizau}{2}
		X & \{1,2\} & \{1\} \\
		\varnothing & \{2\} & \{2\}
	\end{matrizau}$\\\\
	
	En este caso, no podemos llevar la matriz a su forma escalón reducida por filas debido al anillo sobre el que trabajamos. Si se busca encontrar la solución del sistema, habría que encontrar la intersección de las soluciones de cada ecuación.
\end{ejem}

\begin{ejem}
	Lugar de trabajo: $\mathbb{Z}[x]$.
	
	$\mathcal{A} = 
	\begin{sde}{4}
		(x^2 - 1)A  &+& (x^2 + x -1)B &=& x \\
		(x)A &+& (x + 1)B &=& 1
	\end{sde}
	$.\\\\
	
	Construimos la matriz $ A = 
	\begin{matrizau}{2}
		x^2 - 1 & x^2 + x - 1 & x \\
		x & x + 1 & 1
	\end{matrizau}
	$
	$\begin{array}{c}
		R_1 \to (-1) \cdot R_1
	\end{array}$ \\\\
	
	$\begin{matrizau}{2}
		1 - x^2 & 1 - x - x^2 & -x \\
		x & x + 1 & 1 \\
	\end{matrizau}$
	$\begin{array}{c}
		R_2 \to R_2 + x \cdot R_1
	\end{array}$
	$\begin{matrizau}{2}
		1 & 1 & 0 \\
		x & x + 1 & 2
	\end{matrizau}$
	$\begin{array}{c}
		R_2 \to R_2 - x \cdot R_1
	\end{array}$\\\\
	
	$\begin{matrizau}{2}
		1 & 1 & 0 \\
		0 & 1 & 2
	\end{matrizau}$
	$\begin{array}{c}
		R_1 \to R_1 - R_2
	\end{array}$
	$\begin{matrizau}{2}
		1 & 0 & -1 \\
		0 & 1 & 2
	\end{matrizau}$.
\end{ejem}

Realizando las operaciones elementales adecuadas obtenemos la matriz escalón reducida por filas que representa un sistema equivalente al original. Este último sistema llega a mostrar las soluciones del sistema o nos da un sistema más simple para aplicar la intersección de soluciones.

\begin{ejem}
	Lugar de trabajo: $\mathbb{R}$.
	
	$\mathcal{A} = \begin{sde}{4}
		x &+& y &=& 4  \\
		2x &+& 2y &=& 7
	\end{sde}$$\begin{array}{c} \longrightarrow\end{array}$
	$\begin{matrizau}{2}
		1 & 1 & 4 \\
		2 & 2 & 7
	\end{matrizau}$
	$\begin{array}{c}
		R_2 \to R_2 - R_1
	\end{array}$
	$\begin{matrizau}{2}
		1 & 1 & 4 \\
		0 & 0 & -1
	\end{matrizau}$ luego, $\mathcal{S}_{\mathcal{A}} = \varnothing $\\\\
	
	$\mathcal{B} = \begin{sde}{4}
		x &+& y &=& 4  \\
		2x &+& 2y &=& 8
	\end{sde}$$\begin{array}{c} \longrightarrow\end{array}$
	$\begin{matrizau}{2}
		1 & 1 & 4 \\
		2 & 2 & 8
	\end{matrizau}$
	$\begin{array}{c}
		R_2 \to R_2 - R_1
	\end{array}$
	$\begin{matrizau}{2}
		1 & 1 & 4 \\
		0 & 0 & 0
	\end{matrizau}$\\\\
	
	luego, $\mathcal{S}_{\mathcal{B}} = \{(x, 4 - x) \in \mathbb{R}^2 \mid x \in \mathbb{R}\} $\\\\
	
	$\mathcal{C} = \begin{sde}{4}
		x &+& y &=& 2  \\
		x &-& y &=& 0
	\end{sde}$$\begin{array}{c} \longrightarrow\end{array}$
	$\begin{matrizau}{2}
		1 & 1 & 2 \\
		1 & -1 & 0
	\end{matrizau}$
	$\begin{array}{c}
		R_2 \to R_2 - R_1
	\end{array}$\\\\
	
	$\begin{matrizau}{2}
		1 & 1 & 2 \\
		0 & -2 & -2
	\end{matrizau}$
	$\begin{array}{c}
		R_2 \to (-\frac{1}{2}) \cdot R_2
	\end{array}$
	$\begin{matrizau}{2}
		1 & 1 & 2 \\
		0 & 1 & 1
	\end{matrizau}$$\begin{array}{c}
		R_1 \to R_1 -  R_2
	\end{array}$
	$\begin{matrizau}{2}
		1 & 0 & 1 \\
		0 & 1 & 1
	\end{matrizau}$\\\\
	luego, $\mathcal{S}_{\mathcal{C}} = \{(1,1)\} $
\end{ejem}

En el siguiente ejemplo se ilustra la importancia de analizar las propiedades del anillo sobre el que trabajamos.

\begin{ejem}[Operaciones en $\conmodr{n}$]
	Lugar de trabajo: $\conmodr{6}$.
	
	$\mathcal{A} = \begin{sde}{4}
		\clamodr{0}{6} x &+& \clamodr{0}{6} y &=& \clamodr{3}{6}
	\end{sde}$
	Note que $\mathcal{S}_{\mathcal{A}} = \varnothing$. Si ahora construimos $\mathcal{B} = \mathcal{A}_{i \to 2 \cdot 1}$, entonces $\mathcal{B} = \begin{sde}{4}
		\clamodr{0}{6} &+& \clamodr{0}{6} &=& \clamodr{0}{6}
	\end{sde}$ entonces $\mathcal{S}_\mathcal{B} = \conmodr{6} \times \conmodr{6}$.
\end{ejem}

\begin{ejem}[Otra forma de obtener la solución de un sistema de ecuaciones lineales]
	Lugar de trabajo: $\conmodr{6}$.
	
	$\mathcal{A}' = \begin{sde}{4}
		\clamodr{3}{6}x &+& \clamodr{2}{6}y &=& \clamodr{2}{6} \\
		\clamodr{2}{6}x &+& \clamodr{4}{6}y &=& \clamodr{0}{6}
	\end{sde}$ $\begin{array}{c} \longrightarrow\end{array}$ 
	$\begin{matrizau}{2}
		\clamodr{3}{6} & \clamodr{2}{6} & \clamodr{2}{6} \\
		\clamodr{2}{6} & \clamodr{4}{6} & \clamodr{0}{6}
	\end{matrizau}$ $\begin{array}{c} \sim_{\subseteq} \end{array}$
	$\begin{array}{c}
		R_2 \to \clamodr{3}{6} \cdot R_2
	\end{array}$ \\\\
	
	$\begin{matrizau}{2}
		\clamodr{3}{6} & \clamodr{2}{6} & \clamodr{2}{6} \\
		\clamodr{0}{6} & \clamodr{0}{6} & \clamodr{0}{6}
	\end{matrizau}$
	$\begin{array}{c}
		R_1 \to \clamodr{2}{6} \cdot R_1
	\end{array}$
	$\begin{matrizau}{2}
		\clamodr{0}{6} & \clamodr{4}{6} & \clamodr{4}{6} \\
		\clamodr{0}{6} & \clamodr{0}{6} & \clamodr{0}{6}
	\end{matrizau}$\\\\
	
	Luego, $\mathcal{S}_{\mathcal{A}'} \subseteq \conmodr{6} \times S_{\clamodr{0}{6} x  + \clamodr{4}{6} y = \clamodr{4}{6}}$. Ahora resolvemos
	
	$$\clamodr{0}{6} + \clamodr{4}{6} = \clamodr{4}{6}$$
	$$\clamodr{4}{6} y - \clamodr{4}{6} = \clamodr{0}{6}$$
	$$\clamodr{4}{6} (y - \clamodr{1}{6}) = \clamodr{0}{6}$$
	
	Así, $y \in \{\clamodr{1}{6}, \clamodr{4}{6}\}$. Luego $\mathcal{S}_{\mathcal{A}'} \subseteq \conmodr{6} \times \{\clamodr{1}{6}, \clamodr{4}{6}\}$. Si ahora evaluamos las ecuaciones del sistema en un elemento de $\conmodr{6} \times \{\clamodr{1}{6}, \clamodr{4}{6}\}$ si algún elemento satisface las ecuaciones de $\mathcal{A}'$, entonces está en $\mathcal{S}_{\mathcal{A}'}$. Así, encontramos que $$\mathcal{S}_{\mathcal{A}'} = \{ (\clamodr{4}{6}, \clamodr{1}{6}), (\clamodr{4}{6}, \clamodr{4}{6}) \}.$$
\end{ejem}

\begin{ejem}
	Lugar de trabajo: $2\mathbb{Z}$.
	
	$\mathcal{A} = \begin{sde}{4}
		2A &+& 4B &=& 16 \\
		12A &+& 26B &=& 100 \\
	\end{sde}$ $\begin{array}{c} \longrightarrow \end{array}$
	$\begin{matrizau}{2}
		2 & 4 & 16 \\
		12 & 26 & 100
	\end{matrizau}$
	$\begin{array}{c}
		R_2 \to R_2 - 6R_1
	\end{array}$\\\\
	
	$\begin{matrizau}{2}
		2 & 4 & 16 \\
		0 & 2 & 4
	\end{matrizau}$
	$\begin{array}{c}
		R_1 \to R_1 - 2R_2
	\end{array}$
	$\begin{matrizau}{2}
		2 & 0 & 8 \\
		0 & 2 & 4
	\end{matrizau}$\\\\
	
	Resolvemos ahora:
	\begin{enumerate}[label=\alph*), leftmargin=2em]
		\item $2A + 0B = 8$ 
		\item $0A + 2B = 4$
	\end{enumerate}
	
	$$ 2A - 8 = 0 $$
	$$ 2(A - 4) = 0 $$
	Como $\mathbb{Z}$ es un dominio entero, entonces los subanillos de $\mathbb{Z}$ son un dominio entero. Como $2 \not= 0$
	$$ A - 4 = 0 $$
	$$ A = 4 $$
	
	Similarmente, para b), obtenemos $B = 2$. Luego $\mathcal{S}_\mathcal{A} = \{(4,2)\}$
	
\end{ejem}

\subsection{Matrices elementales y matrices inversas}

\begin{dfn}[Matriz elemental]
	Una matriz $A \in \Mmnc{m}{n}{R}$ se dice \textbf{matriz elemental} si se puede obtener a partir de la matriz identidad en $\Mnc{m}{R}$ a través de sólo una operación elemental por fila.
\end{dfn}

\begin{ejem}
	Podemos obtener matrices elementales para un Ejemplo anterior aplicando las operaciones elementales por fila a la matriz identidad en $\Mnc{m}{R}$. Entonces\\\\
	
	$I_{\Mnc{2}{\conmodr{6}}} = \begin{bmatrix}
		\clamodr{1}{6} & \clamodr{0}{6} \\
		\clamodr{0}{6} & \clamodr{1}{6}
	\end{bmatrix}$
	$\begin{array}{c}
		R_2 \to \clamodr{3}{6} \cdot R_2
	\end{array}$
	$\begin{bmatrix}
		\clamodr{3}{6} & \clamodr{0}{6} \\
		\clamodr{0}{6} & \clamodr{1}{6}
	\end{bmatrix}$ es una matriz elemental.
\end{ejem}

\begin{teo} \label{teo:eDeAIgualEA}
	Sea $e$ una operación elemental por fila y $E \in \Mnc{m}{R}$ la matriz elemental $e(I)$. Entonces para toda matriz $A \in \Mmnc{m}{n}{R}$
	$$ e(A) = EA $$
	\begin{proof}
		Ejercicio.
	\end{proof}
\end{teo}

\begin{col} \label{col:BIgualAPA}
	Sean $A,B \in \Mmnc{m}{n}{R}$, entonces $B$ es equivalente por filas a $A$ si y sólo si $B = PA$, donde $P$ es el producto de matrices elementales en $\Mnc{m}{R}$.
	\begin{proof}
		Ejercicio.
	\end{proof}
\end{col}

\begin{dfn}[Matriz inversa]
	Sea $A \in \Mnc{n}{R}$. Una matriz $B \in \Mnc{n}{R}$ tal que $BA = I_{\Mnc{n}{R}}$ es llamada una \textbf{matriz inversa izquierda} de $A$; una matriz $B \in \Mnc{n}{R}$ tal que $AB = I_{\Mnc{n}{R}}$ es llamada una \textbf{matriz inversa derecha} de $A$. Si $B \in \Mnc{n}{R}$ es tal que $BA = AB = I_{\Mnc{n}{R}}$, $B$ se dice una \textbf{matriz inversa} de $A$ y decimos que $A$ es invertible.
\end{dfn}

Podemos demostrar que una matriz inversa es única y entonces denotamos $A^{-1}$ a la matriz inversa de $A$.

\begin{teo} \label{teo:CondicionesDeInvertibilidad}
	Si $A \in \Mnc{n}{R}$, las siguientes proposiciones son lógicamente equivalentes.
	\begin{enumerate}[label=\roman*), leftmargin=2em]
		\item $A$ es invertible.
		\item $A$ es equivalente por filas a $I_{\Mnc{n}{R}}$.
		\item $A$ es producto de matrices elementales.
	\end{enumerate}
	\begin{proof}
		Ejercicio.
	\end{proof}
\end{teo}

\begin{ejem}
	Lugar de trabajo: $\conmodr{7}$.
	
	$\mathcal{A} = \begin{sde}{6}
		\clamodr{323}{7} x &+& \clamodr{228}{7} y &+& \clamodr{123}{7} z &=& \clamodr{2096}{7} \\
		\clamodr{512}{7} x &+& \clamodr{238}{7} y &+& \clamodr{311}{7} z &=& \clamodr{9193}{7} \\
		\clamodr{212}{7} x &+& \clamodr{365}{7} y &+& \clamodr{223}{7} z &=& \clamodr{6995}{7}
	\end{sde}$\\\\
	
	\textit{Solución:}\\
	Elegimos representantes más adecuados, y tenemos:
	
	$\mathcal{A} = \begin{sde}{6}
		\clamodr{1}{7} x &+& \clamodr{4}{7} y &+& \clamodr{4}{7} z &=& \clamodr{3}{7} \\
		\clamodr{1}{7} x &+& \clamodr{0}{7} y &+& \clamodr{3}{7} z &=& \clamodr{2}{7} \\
		\clamodr{2}{7} x &+& \clamodr{1}{7} y &+& \clamodr{6}{7} z &=& \clamodr{2}{7}
	\end{sde}$\\\\
	
	$\begin{array}{c} \longrightarrow \end{array}$
	$\begin{matrizau}{3}
		\clamodr{1}{7} & \clamodr{4}{7} & \clamodr{4}{7} & \clamodr{3}{7} \\
		\clamodr{1}{7} & \clamodr{0}{7} & \clamodr{3}{7} & \clamodr{2}{7} \\
		\clamodr{2}{7} & \clamodr{1}{7} & \clamodr{6}{7} & \clamodr{2}{7}
	\end{matrizau}$
	$\begin{array}{c}
		R_2 \to R_2 + \clamodr{6}{7} \cdot R_1 \\
		R_3 \to R_3 + \clamodr{5}{7} \cdot R_1
	\end{array}$\\\\
	
	$\begin{matrizau}{3}
		\clamodr{1}{7} & \clamodr{4}{7} & \clamodr{4}{7} & \clamodr{3}{7} \\
		\clamodr{0}{7} & \clamodr{3}{7} & \clamodr{6}{7} & \clamodr{6}{7} \\
		\clamodr{0}{7} & \clamodr{0}{7} & \clamodr{5}{7} & \clamodr{3}{7}
	\end{matrizau}$
	$\begin{array}{c}
		R_3 \to \clamodr{3}{7} \cdot R_3
	\end{array}$\\\\
	
	$\begin{matrizau}{3}
		\clamodr{1}{7} & \clamodr{4}{7} & \clamodr{4}{7} & \clamodr{3}{7} \\
		\clamodr{0}{7} & \clamodr{3}{7} & \clamodr{6}{7} & \clamodr{6}{7} \\
		\clamodr{0}{7} & \clamodr{0}{7} & \clamodr{1}{7} & \clamodr{2}{7}
	\end{matrizau}$
	$\begin{array}{c}
		R_1 \to R_1 + \clamodr{3}{7} \cdot R_3 \\
		R_2 \to R_2 + \clamodr{1}{7} \cdot R_3
	\end{array}$\\\\
	
	$\begin{matrizau}{3}
		\clamodr{1}{7} & \clamodr{4}{7} & \clamodr{0}{7} & \clamodr{2}{7} \\
		\clamodr{0}{7} & \clamodr{3}{7} & \clamodr{0}{7} & \clamodr{1}{7} \\
		\clamodr{0}{7} & \clamodr{0}{7} & \clamodr{1}{7} & \clamodr{2}{7}
	\end{matrizau}$
	$\begin{array}{c}
		R_2 \to \clamodr{5}{7} \cdot R_2 \\
		R_1 \to R_1 + \clamodr{3}{7} \cdot R_2
	\end{array}$\\\\
	
	$\begin{matrizau}{3}
		\clamodr{1}{7} & \clamodr{0}{7} & \clamodr{0}{7} & \clamodr{3}{7} \\
		\clamodr{0}{7} & \clamodr{1}{7} & \clamodr{0}{7} & \clamodr{5}{7} \\
		\clamodr{0}{7} & \clamodr{0}{7} & \clamodr{1}{7} & \clamodr{2}{7}
	\end{matrizau}$ Entonces, $\mathcal{S}_{\mathcal{A}} = \{(\clamodr{3}{7}, \clamodr{5}{7}, \clamodr{2}{7})\}$.\\
	
	Luego, una vez encontradas las matrices elementales, las multiplicamos de modo que nos den la inversa de la matriz del sistema $\mathcal{A}$.\\
	
	
	$\begin{bmatrix}%1
		\clamodr{1}{7} & \clamodr{3}{7} & \clamodr{0}{7} \\
		\clamodr{0}{7} & \clamodr{1}{7} & \clamodr{0}{7} \\
		\clamodr{0}{7} & \clamodr{0}{7} & \clamodr{1}{7} 
	\end{bmatrix}$ $\cdot$
	$\begin{bmatrix}%2
		\clamodr{1}{7} & \clamodr{0}{7} & \clamodr{0}{7} \\
		\clamodr{0}{7} & \clamodr{5}{7} & \clamodr{0}{7} \\
		\clamodr{0}{7} & \clamodr{0}{7} & \clamodr{1}{7} 
	\end{bmatrix}$ $\cdot$
	$\begin{bmatrix}%3
		\clamodr{1}{7} & \clamodr{0}{7} & \clamodr{0}{7} \\
		\clamodr{0}{7} & \clamodr{1}{7} & \clamodr{1}{7} \\
		\clamodr{0}{7} & \clamodr{0}{7} & \clamodr{1}{7} 
	\end{bmatrix}$ $\cdot$
	$\begin{bmatrix}%4
		\clamodr{1}{7} & \clamodr{0}{7} & \clamodr{3}{7} \\
		\clamodr{0}{7} & \clamodr{1}{7} & \clamodr{0}{7} \\
		\clamodr{0}{7} & \clamodr{0}{7} & \clamodr{1}{7} 
	\end{bmatrix}$ $\cdot$ \\\\
	$\begin{bmatrix}%5
		\clamodr{1}{7} & \clamodr{0}{7} & \clamodr{0}{7} \\
		\clamodr{0}{7} & \clamodr{1}{7} & \clamodr{0}{7} \\
		\clamodr{0}{7} & \clamodr{0}{7} & \clamodr{3}{7} 
	\end{bmatrix}$ $\cdot$
	$\begin{bmatrix}%6
		\clamodr{1}{7} & \clamodr{0}{7} & \clamodr{0}{7} \\
		\clamodr{0}{7} & \clamodr{1}{7} & \clamodr{0}{7} \\
		\clamodr{5}{7} & \clamodr{0}{7} & \clamodr{1}{7} 
	\end{bmatrix}$ $\cdot$
	$\begin{bmatrix}%7
		\clamodr{1}{7} & \clamodr{0}{7} & \clamodr{0}{7} \\
		\clamodr{6}{7} & \clamodr{1}{7} & \clamodr{0}{7} \\
		\clamodr{0}{7} & \clamodr{0}{7} & \clamodr{1}{7} 
	\end{bmatrix}$ $=$
	$\begin{bmatrix}%res
		\clamodr{4}{7} & \clamodr{1}{7} & \clamodr{5}{7} \\
		\clamodr{0}{7} & \clamodr{5}{7} & \clamodr{1}{7} \\
		\clamodr{1}{7} & \clamodr{0}{7} & \clamodr{3}{7} 
	\end{bmatrix}$\\\\
	
	Luego, 
	$\begin{bmatrix}%res
		\clamodr{4}{7} & \clamodr{1}{7} & \clamodr{5}{7} \\
		\clamodr{0}{7} & \clamodr{5}{7} & \clamodr{1}{7} \\
		\clamodr{1}{7} & \clamodr{0}{7} & \clamodr{3}{7} 
	\end{bmatrix}$ $\cdot$
	$\begin{bmatrix}%orig
		\clamodr{1}{7} & \clamodr{4}{7} & \clamodr{4}{7} \\
		\clamodr{1}{7} & \clamodr{0}{7} & \clamodr{3}{7} \\
		\clamodr{2}{7} & \clamodr{1}{7} & \clamodr{6}{7}
	\end{bmatrix}$ $=$
	$\begin{bmatrix}%id
		\clamodr{1}{7} & \clamodr{0}{7} & \clamodr{0}{7} \\
		\clamodr{0}{7} & \clamodr{1}{7} & \clamodr{0}{7} \\
		\clamodr{0}{7} & \clamodr{0}{7} & \clamodr{1}{7} 
	\end{bmatrix}$
\end{ejem}

\section{Ejercicios}
\begin{enumerate}[label=\arabic*.]
	\item Demostrar el Lema \ref{lema:solucionDeAEsLaInterseccoinDeSoluciones}.
	\item Demostrar el Lema \ref{lema:AContBEntSolBContenSolA}.
	\item Demostrar la Proposición \ref{prop:IgualdadDeSoluciones}.
	\item Demostrar el Lema \ref{lema:SolANoCambiaBajoIntercambio}.
	\item Demostrar la Proposición \ref{prop:LaOtraContencion} \boxed{\supseteq}.
	\item Demostrar el Teorema \ref{teo:IgualdadDeSoluciones}.
	\item Demostrar la Proposición \ref{prop:EquivMatricesEsRelacionDeEquiv}.
	\item Demostrar el Teorema \ref{teo:eDeAIgualEA}.
	\item Demostrar el Corolario \ref{col:BIgualAPA}.
	\item Demostrar el Teorema \ref{teo:CondicionesDeInvertibilidad}.
	\item
	\begin{enumerate}[label=\alph*)]
		\item Demostrar que el conjunto $\mathcal{P}(X)$, con $X = \{ 1,2,3,4 \}$ junto con la operación $\difsim$ y $\cap$ forman un anillo $R = (\mathcal{P}(X), \difsim, \cap)$. (Una demostración se puede realizar considerando funciones características sobre $\conmodr{2}$).
		\item Resolver el sistema de ecuaciones en $R$
		$$\mathcal{C} = \begin{sde}{4}
			\{1,2\}\cap A &\difsim& \{1,3,4\}\cap B &=& \{2\} \\
			\{X\}\cap A &\difsim& \{1,2,4\}\cap B &=& \{1\} 
		\end{sde}$$
	\end{enumerate} 
	\item 
	\begin{enumerate}[label=\alph*), leftmargin=2em]
		\item Demostrar que el conjunto $\Mnc{2}{\mathbb{R}}$ junto con las operaciones usuales de matrices forman un anillo.
		\item Resolver y encontrar la matriz inversa del sistema de ecuaciones lineales con coeficientes en $(\Mnc{2}{\mathbb{R}}, +, \cdot)$
		
		$$\mathcal{B} = \begin{sde}{4}
			\begin{bmatrix}
				1 & 1 \\
				0 & 1
			\end{bmatrix}x &+&
			\begin{bmatrix}
				0 & 1 \\
				1 & 1
			\end{bmatrix}y &=&
			\begin{bmatrix}
				1 & 3 \\
				1 & 5
			\end{bmatrix}\\\\
			\begin{bmatrix}
				2 & 3 \\
				1 & 2
			\end{bmatrix}x &+&
			\begin{bmatrix}
				1 & 2 \\
				0 & 1
			\end{bmatrix}y &=&
			\begin{bmatrix}
				3 & 10 \\
				1 & 5
			\end{bmatrix}
		\end{sde}$$
	\end{enumerate}
	\item 
	\begin{enumerate}[label=\alph*), leftmargin=2em]
		\item Demostrar que el conjunto $\mathbb{Z}[\sqrt{2}]$ junto con las operaciones usuales + y $\cdot$ forman un anillo.
		\item Resolver y encontrar la matriz inversa del sistema de ecuaciones lineales con coeficientes en $(\mathbb{Z}[\sqrt{2}], +, \cdot)$
		
		$$\mathcal{A} = \begin{sde}{4}
			3x &+& 8y &=& 30 + \sqrt{288} \\
			5x &+& 14y &=& 52 + \sqrt{512}
		\end{sde}$$
	\end{enumerate}
	\item 
	\begin{enumerate}[label=\alph*), leftmargin=2em]
		\item Demostrar que el conjunto $\mathcal{F} := \{f \in \mathbb{R} \times \mathbb{R} \mid f:D_f \to \mathbb{R} \text{ es función con } D_f = \mathbb{R} \}$ junto con las operaciones usuales $+$ y $\cdot$ de funciones forman un anillo.
		\item Resolver y encontrar la matriz inversa del sistema de ecuaciones lineales con coeficientes en $(\mathcal{F}, +, \cdot)$
		
		$$\mathcal{A} = \begin{sde}{4}
			\frac{x^4 + 1}{\sqrt{x^4 + 1}}A &+& \frac{\sqrt{x^4 + 1}}{3x^4 + 3}B &=& x^4 \\\\
			\sqrt{x^2 + 1}A &+& \frac{1}{\sqrt{x^4 + 1}}B &=& x^2 + 4
		\end{sde}$$
	\end{enumerate}
	\item 
	\begin{enumerate}[label=\alph*), leftmargin=2em]
		\item 
		Dado el siguiente sistema de ecuaciones lineales con coeficientes en $\mathbb{R}$
		$$ \begin{sde}{6}
			ax &+& ay &+& az &=& b \\
			cx &+& && -bz &=& -c \\
			&& by &+& az &=& -a
		\end{sde} $$
		Si se satisface $abc = 0$, $a \not= bc$ y $b \not= c$, resuelva el sistema, calcule la inversa y dé condiciones de invertibilidad.
		\item 
		Realize lo mismo que en a) pero considerando que el sistema tiene coeficientes en $\conmodr{{23}}$.
	\end{enumerate}
	\item 
	\begin{enumerate}[label=\alph*), leftmargin=2em]
		\item Demostrar que el conjunto $\mathbb{Z}_3 = \{\frac{a}{3^n} \mid a \in \mathbb{Z} \text{ y } n \in \mathbb{N}\}$ junto con las operaciones usuales $+$ y $\cdot$ de números racionales forman un anillo.
		\item Resolver y encontrar la matriz inversa del sistema de ecuaciones lineales con coeficientes en $(\mathbb{Z}_3, +, \cdot)$
		
		$$\mathcal{C} = \begin{sde}{4}
			6A &+& \frac{5}{9}B &=& \frac{8}{3} \\
			\frac{1}{3}A &+& \frac{1}{27}B &=& 1
		\end{sde}$$
	\end{enumerate}
	\item 
	\begin{enumerate}[label=\alph*), leftmargin=2em]
		\item Denótese la clase $\gconmodr{\lp}{\equiv}$ a la clase de todas las fórmulas del lenguaje formal de la lógica proposicional reducida bajo la relación de equivalencia de fórmulas. Defínase $\varphi \oplus \psi \equiv \neg (\varphi \bicond \psi)$ y denótese una tautología por $\top$ y una contradicción por $\bot$.
		
		Demuestre que $(\gconmodr{\lp}{\equiv}, \oplus, \land)$ es un anillo.
		
		\item Resolver y encontrar la matriz inversa del sistema de ecuaciones lineales con coeficientes en $(\gconmodr{\lp}{\equiv}, \oplus, \land)$ donde $\varphi, \psi$ y $\zeta$ son fórmulas atómicas de $\lp$ distintas entre sí tales que $\varphi \oplus \psi \oplus \zeta = \top$.
		
		$$\mathscr{L} = \begin{sde}{6}
			\top \land x &\oplus& \top \land y &\oplus& \top \land z &=& \top \\
			\top \land x &\oplus& \top \land y &\oplus& \bot \land z &=& \psi \\
			\top \land x &\oplus& \bot \land y &\oplus& \top \land z &=& \psi \oplus \zeta
		\end{sde}$$
	\end{enumerate}
\end{enumerate}

\newpage
\thispagestyle{empty}