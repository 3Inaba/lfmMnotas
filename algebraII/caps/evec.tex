\chapter{Espacios vectoriales}

\section{Espacios y subespacios vectoriales}
\subsection{Primeras definiciones}

\begin{dfn}[Espacio vectorial]
	Sea $F$ un campo. Una estructura $(V, +_V, \cdot_V, F)$ se dice \textbf{espacio vectorial} si satisface
	\begin{enumerate}[label=\roman*), leftmargin=2em]
		\item $(V, +_V)$ es un grupo abeliano.
		\item $\fdefc{\cdot_V}{F \times V}{V}{(c,\alpha)}{\cdot_V((c,\alpha)):=c\mathrel{\cdot_V}\alpha}$ es una función que satisface:
		\begin{enumerate}[label=\alph*), leftmargin=2em]
			\item $\forall c, d \in F$ y $\forall \alpha \in V$, $(c \cdot d) \cdot_V \alpha = c \cdot_V (d \cdot_V \alpha)$.
			\item $\forall c, d \in F$ y $\forall \alpha \in V$ $(c + d) \cdot_V \alpha = c \cdot_V \alpha +_V d \cdot_V \alpha$.
			\item $\forall c \in F$ y $\forall \alpha,\beta \in V$ $c \cdot_V (\alpha +_V \beta) = c \cdot_V \alpha +_V c \cdot_V \beta$.
			\item $\forall \alpha \in V$ existe un $1_V \in F$ tal que $1_V \cdot_V \alpha = v$.
		\end{enumerate}
	\end{enumerate}
\end{dfn}

%\begin{tcolorbox}[title=Subespacio vectorial, colback=backsev, boxrule=0.0mm, arc=0.2mm, breakable, enhanced, colframe=backsev, sharp corners, coltitle=white, colbacktitle=colsev, fonttitle=\scshape, borderline={0.3mm}{0.5mm}{colsev}, attach boxed title to top center={xshift=0mm, yshift=-2mm}]
\begin{dfn}[Subespacio vectorial]
	Sea $(V, +_V, \cdot_V, F)$ un espacio vectorial. Si $W \subseteq V$, y $+_W, \cdot_W$ las respectivas restricciones de las funciones $+_V, \cdot_V$ se dice que $(W, +_W, \cdot_W, F)$ es un \textbf{subespacio vectorial de $V$} si
	\begin{enumerate}[label=\roman*), leftmargin=2em]
		\item $0_V \in W$,
		\item $\forall c \in F$ y $\forall \alpha, \beta \in W$ $c \cdot_W \alpha +_W \beta \in W$.
	\end{enumerate}
\end{dfn}
%\end{tcolorbox}

\begin{ejem}
	$(\mathbb{R} \times \{0_\mathbb{R}\}, +_{\mathbb{R} \times \{0_\mathbb{R}\}}, \cdot_{\mathbb{R} \times \{0_\mathbb{R}\}}, \mathbb{R})$ es subespacio vectorial de $(\mathbb{C}, +_{\mathbb{C}}, \cdot_{\mathbb{C}}, \mathbb{R})$:
	
	$$0_{\mathbb{R} \times \{0_\mathbb{R}\}} = 0_\mathbb{C} \text{ y, si }\alpha, \beta \in \mathbb{R} \times \{0_\mathbb{R}\}, c \in \mathbb{R}$$
	$$c\alpha +_{\mathbb{R} \times \{0_\mathbb{R}\}} \beta \in \mathbb{R} \times \{0_\mathbb{R}\}$$
	
	$\therefore$ se tiene lo deseado.
\end{ejem}

Abusando de la notación, cuando se escriba $V$ un \Fev{F} o $W$ un $F$-subespacio vectorial se hace referencia al espacio vectorial $(V, +_V, \cdot_V, F)$ y el subespacio vectorial $(W, +_W, \cdot_W, F)$.

\begin{ejem}
	Determinar si $W = \{(x_1, \ldots, x_n) \in F^{n} \mid x_1 \cdot x_2\}$ es un \Fsev{F} de $F^{n}$.\\
	
	\textit{Solución:}
	
	Si tomamos $\alpha = (1, 0, \ldots, 0) \in W$ y $\beta = (0, 1, 0, \ldots, 0) \in W$ entonces $\alpha + \beta = (1,1,0,\ldots,0) \not\in W$.
	
	Entonces $W$ no es un \Fsev{F} de $F^{n}$.
	
\end{ejem}

\begin{dfn}[Combinación lineal]
	$\beta \in V$ se dice combinación lineal de los vectores $\alpha_1, \ldots, \alpha_n$ si $\exists c_1, \ldots, c_n \in F$
	$$ \sum_{i=1}^{n} c_i \alpha_i = \beta .$$
\end{dfn}

\begin{ejem}
	Lugar de trabajo: $(\mathbb{R}^3, +_{\mathbb{R}^3}, \cdot_{\mathbb{R}^3}, \mathbb{R})$
	
	¿$(1,2,3)$ es combinación lineal de $(2,4,8)$ y $(0,0,3)$?
	
	Si $(1,2,3)$ = $c_1 (2,4,8) + c_2 (0,0,3)$, entonces
	$$ (1,2,3) = (2c_1, 4c_1, 8c_1 + 3c_2) $$
	
	Luego,
	$\begin{array}{c c c c c}
		8c_1 &+& 3c_2 &=& 3\\
		4c_1 & & &=& 2 \\
		2c_1 & & &=& 1 \\
	\end{array}$,
	
	$\begin{matrizau}{2}
		8 & 3 & 3 \\
		4 & 0 & 2 \\
		2 & 0 & 1 \\
	\end{matrizau}$
	$\begin{array}{c}
		R_2 \to R_2 - 2R_3 \\
		R_2 \leftrightarrow R_3 \\
		R_1 \leftrightarrow R_2 \\
	\end{array}$
	$\begin{matrizau}{2}
		2 & 0 & 1 \\
		8 & 3 & 3 \\
		0 & 0 & 0 \\
	\end{matrizau}$
	$\begin{array}{c}
		R_2 \to R_2 - 4R_1 \\
		R_1 \to \frac{1}{2}R_1 \\
		R_2 \to \frac{1}{3}R_2 \\
	\end{array}$
	$\begin{matrizau}{2}
		1 & 0 & \frac{1}{2} \\
		0 & 1 & -\frac{1}{3} \\
		0 & 0 & 0 \\
	\end{matrizau}$\\
	
	entonces,
	$(1,2,3) = \frac{1}{2}(2, 4 ,8) - \frac{1}{3}(0, 0, 3)$.
\end{ejem}

\begin{prop}
	Sea $V$ un $F$-espacio vectorial y $\{ V_\alpha \}_{\alpha \in \Omega}$ una familia no vacía de $F$-subespacios de $V$. Entonces $\bigcap\limits_{\alpha \in \Omega} V_\alpha$ es un $F$-subespacio vectorial de $V$.
	
	\begin{proof}
		Se tiene que $\forall \alpha \in \Omega$ $0_{V_\alpha} \in V_\alpha$, pero $\forall \alpha \in \Omega$ $0_{V_\alpha} = 0_V$, así $\bigcap\limits_{\alpha \in \Omega} V_\alpha \not= \varnothing$.
		
		Sean $\zeta_1, \zeta_2 \in \bigcap\limits_{\alpha \in \Omega} V_\alpha$ y $c \in F$, luego, $\forall \alpha \in \Omega$ $c_1 \zeta_1 + \zeta_2 \in V_\alpha$, luego $c_1 \zeta_1 + \zeta_2 \in \bigcap\limits_{\alpha \in \Omega} V_\alpha$, así $\bigcap\limits_{\alpha \in \Omega} V_\alpha$ es un $F$-subespacio vectorial de $V$.
	\end{proof}
\end{prop}

\begin{dfn}[Suma de subconjuntos de un \Fev{F}]
	Sea $V$ un $F$-espacio vectorial. Si $S_i \subset V$ $\forall i \in \inte{1}{k}$, se define la \textbf{suma de subconjuntos} de $V$ como
	$$\sum\limits_{i=1}^{k} S_i = \left\{ \sum\limits_{i=1}^{k} \alpha_i \in V \mid \alpha_i \in S_i \; \forall i \in \inte{1}{k} \right\}.$$
\end{dfn}

\begin{obs}
	En particular, la suma de subespacios es un subespacio.
\end{obs}

\begin{prop} \label{prop:SumaDeSubEspaciosEsSubespacio}
	Si $W_1, \ldots, W_k$ son \Fsevs{F} del \Fev{F} $V$ entonces $\sum\limits_{i=1}^{k} W_i$ es un \Fsev{F} de $V$.
	
	\begin{proof}
		Ejercicio.
	\end{proof}
\end{prop}

\begin{dfn}[Espacio generado por un conjunto]
	Sea $S$ un conjunto de vectores de un \Fev{F}. El subespacio generado por $S$ denotado por $\Legen{S}$ se define como:
	
	$$ \Legen{S} = \{ \alpha \in V \mid \alpha \in W, \; \forall W \Fsev{F} \text{ de V, con } S \subseteq W \} $$
	
	\noindent O bien, $\Legen{S} = \bigcap W$, $\forall W$ \Fsev{F} de $V$ tal que $S \subseteq W$.
\end{dfn}

Cuando $S$ es un conjunto finito no vacío de vectores $S = \{ \alpha_1, \ldots, \alpha_k \}$ se dice simplemente que $\Legen{S}$ es el subespacio generado por $\alpha_1, \ldots, \alpha_k$.

\begin{prop} \label{prop:SiWEsLaSumaDeSubespaciosEntWEsElGeneradoDeLaUnion}
	Sean $W_1, \ldots, W_k$ \Fsevs{F} del \Fev{F} $V$. Si $W = \sum\limits_{i=1}^{k} W_i$ entonces $W = \Legen{\bigcup\limits_{i=1}^{k} W_i}$.
	
	\begin{proof}
		Ejercicio.
	\end{proof}
\end{prop}

\begin{ejem}
	\begin{itemize}
		\item $\Legen{\varnothing} = \{0_V\}$ (¿por qué?).
		\item $\Legen{V} = V$.
		\item $\Legen{\{(\clamodr{0}{2}, \clamodr{1}{2}), (\clamodr{1}{2}, \clamodr{0}{2})\}} = \conmodr{2} \times \conmodr{2}$.
		\item $\Legen{\{ (\clamodr{0}{2}, \clamodr{1}{2}) \}} = \{\clamodr{0}{2}\} \times \conmodr{2}$.
	\end{itemize}
\end{ejem}

\begin{obs}
	Si $W = \sum\limits_{i=1}^{k} W_i$, $W = \Legen{\bigcup\limits_{i=1}^{k} W_i}$.
\end{obs}

\begin{teo}
	Si $S \not= \varnothing$, con $S \subseteq V$, $V$ un \Fev{F}, entonces
	
	$$ \Legen{S} = \{ \alpha \in V \mid \alpha = \sum_{k=1}^{m} a_k s_k \text{ tal que } a_k \in F \wedge s_k \in S \} $$
	
	\begin{proof}
		$C_{l_s} \subseteq \Legen{S}$, en efecto:
		claramente $S \subseteq \Legen{S}$, de acuerdo a la definición, $\Legen{S}$ es un \Fsev{F} de $V$. Entonces cualquier combinación lineal de los elementos de $S$ pertenecen a $\Legen{S}$. Así $C_{l_s} \subseteq \Legen{S}$.
		
		Ahora $\Legen{S} \subseteq C_{l_s}$, en efecto:\\
		observe primero que $S \subseteq C_{l_s}$. Veamos que $C_{l_s}$ es un \Fsev{F} de $V$. $0_V \in C_{l_s}$, claramente. Si $c \in F$, $\sum\limits_{k_1=1}^{m_1} b_{k_1} s_{k_1}, \sum\limits_{k_2=1}^{m_2} b_{k_2} s_{k_2} \in C_{l_s}$, entonces $\sum\limits_{k_1=1}^{m_1} b_{k_1} s_{k_1} + \sum\limits_{k_2=1}^{m_2} b_{k_2} s_{k_2} \in C_{l_s}$.
		
		Entonces $\Legen{S} \subseteq C_{l_s}$.
	\end{proof}
\end{teo}

\begin{dfn}[Vectores fila de una matriz]
	Sea $A \in \Mmnc{m}{n}{F}$. Los \textbf{vectores fila} de $A$ son los vectores en $F^n$ dados por $\alpha_i = (A_{i1}, \ldots, A_{in})$, $i \in \inte{1}{m}$.
\end{dfn}

\begin{dfn}[Espacio de filas de una matriz]
	Sea $A \in \Mmnc{m}{n}{F}$ y $\alpha_1, \ldots, \alpha_n$ vectores fila de $A$. Al \Fsev{F} $\Legen{\cv{\alpha}{n}}$ de $F^n$ se le dice \textbf{espacio de filas} de $A$.
\end{dfn}

\begin{dfn}[Espacio solución de una matriz]
	Sea $A \in \Mmnc{m}{n}{F}$. Entonces el conjunto de todas las matrices $x \in \Mmnc{n}{1}{F}$ tal que $Ax = 0$ es llamado \textbf{espacio solución} de $A$.
\end{dfn}

\subsection{Bases y dimensión}

\begin{dfn}[Conjunto linealmente independiente]
	Si $S$ es un subconjunto no vacío de un \Fev{F} $V$, $S$ se dice linealmente independiente sobre $F$ (abreviamos \li sobre $F$) si
	
	\begin{enumerate}[label=\roman*), leftmargin=2em]
		\item  Caso finito:
		Si $\card{S} = n$ y $S = \cv{\alpha}{n}$, si
		$$ \sum_{k=1}^{n} c_k \alpha_k = 0, \text{ entonces } c_k = 0, \; \forall k \in \inte{1}{n}. $$
		\item Caso infinito:
		Si $\card{S} \not\in \mathbb{N}$, $\forall n \in \mathbb{N}$ y $\forall \cv{\alpha}{n} \subseteq S$ y $\forall c_1, \ldots, c_n \in F$, si
		$$ \sum_{k=1}^{n} c_k \alpha_k = 0, \text{ entonces } c_k = 0, \; \forall k \in \inte{1}{n}. $$
	\end{enumerate}
\end{dfn}

\begin{dfn}[Conjunto linealmente dependiente]
	Si $S$ es un subconjunto no vacío de un $\Fev{F}$ $V$, $S$ se dice \textbf{linealmente dependiente} sobre $F$ (abreviamos \ld sobre $F$) si $S$ no es linealmente independiente sobre $F$.
\end{dfn}

Al mencionar un conjunto \li ó \ld sobre un campo $F$ se conviene solo decir que el conjunto $S$ es \li ó el conjunto $S$ es \ld, entendiéndose que esto es sobre el campo $F$ parte del $\Fev{F}$ $V$ que contiene a $S$, siempre que se mencione a $V$.

\begin{ejem}
	Lugar de trabajo: $\mathbb{R}^{3}$ como $\Fev{\mathbb{R}}$.
	
	¿Es el conjunto $S = \{ \alpha_1, \alpha_2 \}$, con $\alpha_1 = (1,2,3)$ y $\alpha_2 = (4,5,1)$, un conjunto \li?\\
	
	
	\textit{Solución:}\\
	Sean $c_1, c_2 \in \mathbb{R}$, si
	$$\sum\limits_{i=1}^{2} c_i \alpha_i = 0_{\mathbb{R}^{3}}, \text{ i.e.}$$
	$c_1 (1,2,3) + c_2 (4,5,1) = (0,0,0) \implies (c_1 + 4c_2, 2c_1 + 5c_2, 3c_1 + c_2) = (0, 0, 0)$, de donde obtenemos el siguiente sistema de ecuaciones:
	
	$\begin{array}{ccccc}
		c_1 &+&  4c_2 &=& 0 \\
		2c_1 &+&  5c_2 &=& 0 \\
		3c_1 &+&  c_2 &=& 0
	\end{array} \longrightarrow
	\begin{matrizau}{2}
		1 & 4 & 0 \\
		2 & 5 & 0 \\
		3 & 1 & 0
	\end{matrizau}
	\begin{array}{c}
		R_2 \to R_2 - 2R_1 \\
		R_3 \to R_3 - 3R_1
	\end{array}
	\begin{matrizau}{2}
		1 & 4 & 0 \\
		0 & -3 & 0 \\
		0 & -11 & 0
	\end{matrizau}\\
	\begin{array}{c}
		R_2 \to -\frac{1}{3}R_2 \\
		R_3 \to R_3 + 11 R_2 \\
		R_1 \to R_1 - 4 R_2
	\end{array}$
	$\begin{matrizau}{2}
		1 & 0 & 0 \\
		0 & 1 & 0 \\
		0 & 0 & 0
	\end{matrizau}$.\\
	
	Entonces $c_1 = 0$ y $c_2 = 0$, luego, $S$ es \li.
\end{ejem}

\begin{dfn}[Conjunto generador de un espacio vectorial]
	Sea $V$ un \Fev{F} y $S \subseteq V$, se dice que $S$ genera a $V$ si
	$$ \Legen{S} = V .$$
\end{dfn}

\begin{obs}
	Como conjunto dos bases pueden ser iguales, pero nos interesa el orden a la hora de obtenerlas.
\end{obs}

\begin{dfn}[Base ordenada de un \Fev{F}]
	Sea $V$ un \Fev{F}, $V \not= \{ 0_V \}$, un conjunto $\base{}$ se dice \textbf{base ordenada} de $V$ si satisface:
	\begin{enumerate}[label=\roman*), leftmargin=2em]
		\item $\base{}$ es linealemente independiente sobre $F$,
		\item $\Legen{\base{}} = V$,
		\item $\base{}$ es ordenada por un conjunto de índices bien ordenado.
	\end{enumerate}
\end{dfn}

En adelante cuando se menciona una base $\base{}$ de un \Fev{F} se hace referencia a una base ordenada.

\begin{teo}[Existencia de bases para un \Fev{F}]
	Si $V$ es un $F$-espacio vectorial, $V \not= \{0_V\}$, entonces $V$ admite una base.
	
	\begin{proof}
		Sea $\Omega_{\li} := \{ X \in \mathcal{P}(V) \mid  X \text{ es \li} \}$.  Puesto que $V \not= \{0_V\}$, $V \setminus \{0_V\} \not= \varnothing$, así $\forall \alpha \in V \setminus \{0_V\}$ se tiene que $\{\alpha\}$ es un conjunto \li. Así el conjunto $\Omega_{\li} \not= \varnothing$.
		
		$\Omega_{\li}$ es claramente un conjunto parcialmente ordenado respecto a la relación de contención.\\
		
		Considere $\mathscr{C}$ una cadena arbitraria del conjunto $\Omega_{\li}$ y sea $B = \bigcup\limits_{c \in \mathscr{C}} c$, $n \in \mathbb{N} \setminus \{0\}$ y $\alpha_1,\ldots,\alpha_n \in B$ tal que $\alpha_1,\ldots,\alpha_n \in c$. Como $\cv{\alpha}{n} \subseteq c$, entonces $\cv{\alpha}{n}$ es \li, luego $B$ es \li, si no lo fuera, entonces $\cv{\beta}{n} \subset B$ es un conjunto \ld, entonces para algún $c \in \mathscr{C}$ $\beta_i \in c$, $i \in \inte{1}{n}$, como $\mathscr{C}$ es una cadena, $\mathscr{C}$ es un orden total. Entonces existe un $C$ \li tal que $c \in C$ $\forall c \in \mathscr{C}$, luego $\cv{\beta}{n} \subseteq C$, entonces $C$ no es \li. Luego $B$ es \li y cota superior. Por el Lema \ref{lema:Zorn} existe $\base{}$ elemento maximal de $\Omega_{\li}$.\\
		
		$\Legen{\base{}} = V$. En efecto, sea $\alpha \in V$ y $A = \base{} \cup \{\alpha\}$. Si $\alpha \in \base{}$ se tiene lo deseado. Así, suponga que $\alpha \not\in \base{}$, luego $\base{} \subset A$. Por maximalidad de $\base{}$, se sigue que $A \not\in \Omega_{\li}$, luego $A$ es \ld. Sean $\alpha_1,\ldots,\alpha_n \in A$ elementos distintos, y $c_1,\ldots,c_n \in F$, $\{c_1, \ldots, c_n\} \not= \{0_F\}$, tal que $\sum\limits_{i = 1}^{n} c_i \alpha_i = 0$.
		
		Como $A$ es \ld, necesariamente existe $i_0 \in \inte{1}{n}$ tal que $\alpha_{i_0} = \alpha$ y $c_{i_0} \not= 0$. Entonces, sin pérdida de generalidad, suponga que $\alpha = \alpha_1$ y $c_1 = c_{i_0} \not= 0$. Entonces $\alpha = \beta_2 \alpha_2 + \cdots + \beta_n \alpha_n$ con $\beta_i = -\frac{c_i}{c_1}$, así $\alpha \in \Legen{\base{}}$, pero como $\alpha$ es arbitrario $\Legen{ \base{} } = V$.
	\end{proof}
\end{teo}

\begin{prop}
	Existen espacios vectoriales con bases cuya cardinalidad no es finita.
	
	\begin{proof}
		Considere $\mathbb{R}$ como $\Fev{\mathbb{Q}}.$ Veamos que cualquier base tiene cardinalidad infinita. Procedemos por introducción de la negación.
		
		Suponga que existe $\{\rho_1, \ldots, \rho_n\}$ base de $\mathbb{R}$ sobre $\mathbb{Q}$. Así, para todo $r \in \mathbb{R}$ existen $c_1, \ldots, c_n \in \mathbb{Q}$ tal que
		$$r = \sum_{i = 1}^{n} c_i \rho_i.$$
		
		Observe que $\forall i \in \inte{1}{n}$, definiendo $\mathbb{Q}_{\rho_i} := \{c\rho_i \in \mathbb{R} \mid c \in \mathbb{Q} \}$, si  $\fdefc{\mathscr{F}}{\mathbb{Q}_{\rho_i}}{\mathbb{Q}}{c\rho_i}{c}$ entonces si $c\rho_i = d\rho_i$ se tiene que $c = d$. Luego $\mathscr{F}(\mathbb{Q}_{\rho_i}) = \mathbb{Q}$ y entonces $\mathscr{F}$ es biyectiva por lo tanto $\mathbb{Q}_{\rho_i} \sim \mathbb{Q} \sim \mathbb{N}$, es así que $\times_{i=1}^{n} \mathbb{Q}_{\rho_i}$ es numerable, y $\fdefc{+}{\times_{i=1}^{n} \mathbb{Q}_{\rho_i}}{\mathbb{R}}{(c_1 \rho_1, \ldots, c_n \rho_n)}{\sum\limits_{i=1}^{n} c_i \rho_i}$ es una función.
		
		$+$ es una función biyectiva (¿por qué?), luego $\mathbb{Q}_{\rho_i} \sim \mathbb{R}$ y entonces $\mathbb{R}$ es numerable. Pero se sabe que $\mathbb{R}$ es no numerable, luego $\mathbb{R}$ como $\Fev{\mathbb{Q}}$ no tiene base finita. \\
	\end{proof}
\end{prop}

\begin{teo} \label{teo:InvarianzaFinita}
	Toda base de un \Fev{F} tiene la misma cardinalidad (caso finito).
	
	\begin{proof}
		Ejercicio.
	\end{proof}
\end{teo}

Así, para todo \Fev{F} $V \not= \{ 0_V \}$ se define como la dimensión $V$ a la cardinalidad de cualquiera de sus bases. Denotamos $\dim(V)$ a la dimensión del \Fev{F} $V$.

\begin{conven}
	La dimensión del espacio vectorial cero tiene dimensión $0$.
\end{conven}

\begin{col} \label{col:CriteriosParaBases}
	Si $V$ es un \Fev{F} de dimensión finita $n$, $n \in \mathbb{N} \setminus \{0\}$, se tiene que
	
	\begin{enumerate}[label=\roman*), leftmargin=2em]
		\item Cualquier subconjunto de $V$ que tenga más de $n$ vectores es linealmente dependiente.
		
		\item Ningún subconjunto de $V$ que tenga menos de $n$ vectores genera a $V$.
	\end{enumerate}
	
	\begin{proof}
		Ejercicio.
	\end{proof}
\end{col}

\begin{lema}
	Si $S$ es un subconjunto linealmente independiente de un \Fev{F} $V$ y $\beta \in V$ tal que $\beta \not\in \Legen{S}$ entonces $S \cup \{\beta\}$ es un conjunto \li.
	
	\begin{proof}
		Suponga que $\alpha_1, \ldots, \alpha_n$ son vectores distintos en $S$ y que $\sum\limits_{i=1}^{n} c_i \alpha_i + b \beta = 0$, $c_i \in F$, $\forall i \in \inte{1}{m}$ y $b \in F$.\\
		
		Si $b \not= 0_F$, entonces $\beta = - \frac{\sum\limits_{i=1}^{m} c_i \alpha_i}{b} = \sum\limits_{i=1}^{m} -\frac{c_i}{b} \alpha_i$, así $\beta \in \Legen{S}$, pero por hipótesis $\beta \not\in \Legen{S}$. Aplicando introducción de la negación $b = 0_F$.\\
		
		Así, $\sum\limits_{i=1}^{m} c_i \alpha_i = 0$, luego entonces $c_i = 0$ $\forall i \in \inte{1}{m}$. Como $S$ es \li se sigue que $S \cup \{\beta\}$ es \li.
		
	\end{proof}
\end{lema}

\begin{teo}
	Si $W$ es un \Fsev{F} del \Fev{F} $V$ de dimensión finita $n$, entonces todo conjunto \li de $W$ es finito y es parte de una base de $W$.
	
	\begin{proof}
		Suponga que $S_0 \subseteq W$ es \li. Si $S \subseteq W$ es un conjunto \li tal que $S_0 \subseteq S$ entonces $S$ también es un conjunto \li de $V$ sobre $F$. Como $V$ es de dimensión finita, $S$ no tiene más de $n$ elementos.\\
		
		Se extiende $S_0$ a una base en $W$ como sigue:\\
		Si $\Legen{S_0} = W$, entonces $S_0$ es base de $W$ y se tiene lo deseado. Si $\Legen{S_0} \not= W$, i.e. $\Legen{S} \subset W$, así $\exists \beta_1 \in W \setminus \Legen{S_0}$, por el lema anterior $S_0 \cup \{\beta_1\}$ es un conjunto linealmente independiente.\\
		
		Así considere $\Legen{S_0 \cup \{\beta_1\}} = W$, se tiene lo deseado. En caso contrario se aplica de nuevo el lema anterior en, a lo más, $n-1$ aplicaciones del lema se encuentra una base para $W$, de lo cual $S_0$ es parte de la base.
	\end{proof}
\end{teo}

\begin{col} \label{col:dimWMenordimV}
	Si $W$ es un \Fsev{F} del \Fev{F} $V$ de dimensión finita, $W \not= V$, entonces $W$ es de dimensión finita y $\dim(W) < \dim(V)$.
	
	\begin{proof}
		Ejercicio.
	\end{proof}
\end{col}

\begin{col} \label{col:ConjuntosLISonParteDeUnaBase}
	En un \Fev{F} $V$ de dimensión finita $n$, todo conjunto linealmente independiente de vectores es parte de una base.
	
	\begin{proof}
		Ejercicio.
	\end{proof}
\end{col}

\begin{col} \label{col:MatricesYEspaciosVectoriales}
	Sea $A \in \Mmnc{m}{n}{F}$ y suponga que los vectores fila de $A$ forman un conjunto linealmente independiente de vectores de $F^n$. Entonces $A$ es invertible.
	
	\begin{proof}
		Ejercicio.
	\end{proof}
\end{col}

\begin{dfn}[Suma directa]
	Si $W_1, W_2$ son \Fsevs{F} del $F$-espacio vectorial $V$ se dice que $V$ es \textbf{suma directa} de $W_1$ y $W_2$ si se tiene:
	\begin{enumerate}[label=\roman*), leftmargin=2em]
		\item $V = W_1 + W_2$,
		\item $W_1 \cap W_2 = \{0_V\}$.
	\end{enumerate}
	y denotamos la suma como $W_1 \oplus W_2$.
\end{dfn}

\begin{teo} \label{teo:DimensionDeSumaDeSubespacios}
	Si $W_1, W_2$ son \Fsevs{F} del \Fev{F} $V$ de dimensión finita $n$ entonces $W_1 + W_2$ es un \Fev{F} de $V$ de dimensión finita y $\dim(W_1) + \dim(W_2) = \dim(W_1 + W_2) + \dim(W_1 \cap W_2)$.
	
	\begin{proof}
		Ejercicio. Considerar $W_1 \cap W_2 = \{0_V\}$, y $W_1 + W_2$ con $W_1 = \{0_V\}$ y $W_2 = \{0_V\}$.
	\end{proof}
\end{teo}

\section{Coordenadas}

\begin{dfn}
	Sea $V$ un \Fev{F}. $\base{}$ una base ordenada por un conjunto de índices bien ordenado $\Omega$. Se define la relación
	$$\fdefc{\text{Coord}_k}{V}{F}{\alpha}{c_{\alpha_k}}$$
	donde $\alpha = \sum\limits_{k \in \Omega} C_{\alpha_k} \alpha_k$.
\end{dfn}

\begin{obs}
	Recuerde que $\alpha \in V$ es combinación lineal de $\alpha_{k1}, \ldots, \alpha_{kn}$ si existen $c_{k1}, \ldots c_{kn} \in F$ tal que $\alpha = \sum\limits_{i=1}^{n} c_{ki} \alpha_{ki}$. Los elementos no cero en el campo $F$ en una combinación lineal forman un conjunto finito. Es decir, si $\alpha = \sum\limits_{k \in \Omega} c_{\alpha_k} \alpha_k$, entonces $c_{\alpha_k} = 0$  para casi todo $k \in \Omega$.
\end{obs}

Como $\Legen{\base{}} = V$, entonces $\fdefc{\text{Coord}_k}{V}{F}{\alpha}{c_{\alpha_k}}$ es de asignamiento total.

La unicidad se sigue de la independencia lineal. Es fácil ver esto para el caso finito. Para el caso infinito considere un conjunto bien ordenado comparando $\phi_1$ y $\phi_2$.\\

Así $\fdefc{\text{Coord}_k}{V}{F}{\alpha}{c_{\alpha_k}}$ es una función. Luego, $(C_{\alpha_k})_{k \in \Omega}$ es el vector de coordenadas de $\alpha$ en la base \base{} ordenada por el conjunto bien ordenado $\Omega$.

En particular, si $V$ es un \Fev{F} de dimensión finita y $\alpha = \sum\limits_{i=1}^{n} c_{\alpha_i} \alpha_i$, entonces $\left[\begin{array}{c}
	c_{\alpha_i} \\
	\vdots \\
	c_{\alpha_n}
\end{array}\right]$ es el vector de coordenadas (o matriz de coordenadas) de $\alpha$ en la base \base{}.

Frecuentemente, es conveniente para nosotros utilizar la matriz de coordenadas de $\alpha$ respecto a la base ordenada $\base{}$ en vez del vector de coordenadas. Para indicar la dependencia de la matriz de coordenadas en la base $\base{}$ denotamos

$$[\alpha]_{\base{}}$$

para la matriz de coordenadas del vector $\alpha$ respecto a la base $\base{}$.

\begin{teo} \label{teo:CambioDeBase1}
	Sea $V$ un \Fev{F} de dimensión finita $n$, y sean $\base{}, \base{'}$ dos bases ordenadas de $V$. Entonces hay una única matriz invertible $P \in \Mnc{n}{F}$ tal que, si $\alpha \in V$
	\begin{enumerate}[label=\roman*), leftmargin=2em]
		\item $[\alpha]_{\base{}} = P[\alpha]_{\base{'}}$,
		\item $[\alpha]_{\base{'}} = P^{-1}[\alpha]_{\base{}}$.
	\end{enumerate}
	
	Las columnas de $P$ están dadas por
	$$P_j = [\alpha'_j]_{\base{}} \; \forall j \in \inte{1}{n}$$
	
	\begin{proof}
		Sean $\base{} = \cv{\alpha}{n}$, $\base{'} = \cv{\alpha'}{n}$.\\
		Entonces existen $P_{1j}, \ldots, P_{nj} \in F$ únicos tales que $\alpha'_j = \sum\limits_{i=1}^{n} P_{ij} \alpha_i \; \forall j \in \inte{1}{n}$.
		
		Sean $x'_1, \ldots, x'_n$ las coordenadas de $\alpha$ en la base $\base{'}$. Entonces $\alpha = \sum\limits_{i=1}^{n} x'_j \alpha'_j$, luego $\alpha = \sum\limits_{j=1}^{n} x'_j \sum\limits_{i=1}^{n} P_{ij} \alpha_i = \sum\limits_{j=1}^{n} \sum\limits_{i=1}^{n} P_{ij} x'_j \alpha_i = \sum\limits_{j=1}^{n} \left(\sum\limits_{i=1}^{n} P_{ij} x'_j \right) \alpha_i$. Entonces 
		$$\alpha = \sum\limits_{j=1}^{n} \left(\sum\limits_{i=1}^{n} P_{ij} x'_j \right) \alpha_i,$$
		es decir que las coordenadas $x_1, \ldots, x_n$ de $\alpha$ respecto a la base $\base{}$ están determinadas por
		
		$$x_i = \sum\limits_{i=1}^{n} P_{ij} x'_j \; \forall i \in \inte{1}{n}.$$
		
		Luego, sea $P \in \Mnc{n}{F}$ tal que $P((i,j)) = P_{ij}$ y $[\alpha]_{\base{}}, [\alpha]_{\base{'}}$ las matrices de coordenadas respecto a la bases $\base{}$ y $\base{'}$ respectivamente. Entonces tenemos
		
		$$[\alpha]_{\base{}} = P [\alpha]_{\base{'}}.$$ 
		
		Como $\base{}$ y $\base{'}$ son conjuntos \li se tiene que $[\alpha]_{\base{}} = 0_{\Mnc{n}{F}}$ si y sólo si $[\alpha]_{\beta{'}} = 0_{\Mnc{n}{F}}$. Se sigue entonces que $P$ es invertible, y se tiene
		
		$$[\alpha]_{\base{'}} = P^{-1} [\alpha]_{\base{}} .$$
	\end{proof}
\end{teo}

\begin{teo}
	Sea $P \in \Mnc{n}{F}$ invertible. Sea $V$ un \Fev{F} de dimensión finita $n$, y sea $\base{}$ una base ordenada de $V$. Entonces existe una única base ordenada $\base{'}$ tal que
	\begin{enumerate}[label=\roman*), leftmargin=2em]
		\item $[\alpha]_{\base{}} = P[\alpha]_{\base{'}}$
		\item $[\alpha]_{\base{'}} = P^{-1}[\alpha]_{\base{}}$
	\end{enumerate}
	$\forall \alpha \in V$
	
	\begin{proof}
		Sea $\base{} = \cv{\alpha}{n}$. Si $\base{'} = \cv{\alpha'}{n} \subseteq V$ es \li tal que $[\alpha]_{\base{}} = P[\alpha]_{\base{'}}$, es claro que $\alpha'_j = \sum\limits_{i=1}^{n} P_{ij} \alpha_i$.
		
		Entonces solo debemos demostrar que $\base{'}$ es base de $V$. Sea $Q := P^{-1}$. Entonces $\sum\limits_{j=1}^{n} Q_{jk} \alpha'_j = \sum\limits_{j=1}^{n} Q_{jk} \sum\limits_{i = 1}^{n} P_{ij} \alpha_i = \sum\limits_{j=1}^{n} \sum\limits_{i=1}^{n} P_{ij}Q_{jk} \alpha_i = \sum\limits_{i=1}^{n} \left(\sum\limits_{j=1}^{n} P_{ij}Q_{jk} \right) \alpha_i = \alpha_k.$
		
		Entonces $\base \subseteq \Legen{\base{'}}$, luego $\Legen{\base{'}} = V$. Aplicando el Teorema \ref{teo:CambioDeBase1} se tiene lo deseado.
	\end{proof}
\end{teo}

\section{Ejercicios}
\begin{enumerate}[label=\arabic*.]
	\item Demostrar la Proposición \ref{prop:SumaDeSubEspaciosEsSubespacio}
	\item Demostrar la Proposición \ref{prop:SiWEsLaSumaDeSubespaciosEntWEsElGeneradoDeLaUnion}
	\item Demostrar el Teorema \ref{teo:InvarianzaFinita}
	\item Demostrar el Corolario \ref{col:CriteriosParaBases}
	\item Demostrar el Corolario \ref{col:dimWMenordimV}
	\item Demostrar el Corolario \ref{col:ConjuntosLISonParteDeUnaBase}
	\item Demostrar el Corolario \ref{col:MatricesYEspaciosVectoriales}
	\item Demostrar el Teorema \ref{teo:DimensionDeSumaDeSubespacios}
	\item Demostrar que $\conmodr{7} \times \conmodr{7} \times \{[0]_7\}$ es $\Fsev{\conmodr{7}}$ de $\conmodr{7} \times \conmodr{7} \times \conmodr{7}$.
	\item Demostrar que el conjunto cociente es un espacio vectorial.
	\item \begin{enumerate}[label=\alph*), leftmargin=2em]
		\item Demostrar que $\{0_F\}^n$ es un \Fsev{F} de $F^n$.
		\item Determinar si $W = \{(x_1, x_2, x_3) \in \conmodr{3} \times \conmodr{3} \conmodr{3} \;|\; x_3 = x_1 + 3x_2\}$ es un $\Fsev{\conmodr{3}}$ de $\conmodr{3} \times \conmodr{3} \times \conmodr{3}$.
		\item Determinar si $W = \{(x_1, \ldots, x_n) \in \mathbb{R}^n\} \mid x_2 \in \mathbb{Q}$ es un $\Fsev{\mathbb{R}}$ de $\mathbb{R}^n$.
	\end{enumerate}
	\item \begin{enumerate}[label=\alph*), leftmargin=2em]
		\item Demostrar que $\Mnc{n}{F}$ es un $\Fsev{F}$.
		\item Determinar si los siguientes subconjuntos de $\Mnc{n}{F}$ con las operaciones usuales son $\Fsev{F}$ de $\Mnc{n}{F}$.
		\begin{enumerate}[label=\alph*), leftmargin=2em]
			\item $W = \{A \in \Mnc{n}{F} \mid A \text{ es invertible}\}$,
			\item $W = \{A \in \Mnc{n}{F} \mid A \text{ no es invertible}\}$,
			\item $W = \{A \in \Mnc{n}{F} \mid A^2 = A\}$,
			\item $W = \{A \in \Mnc{n}{F} \mid AB = BA, \text{ para algún } B \in \Mnc{n}{F}\}$.
		\end{enumerate}
	\end{enumerate}
	\item Demostrar que si $W_1, \ldots, W_k$ son \Fsevs{F} de un \Fev{F} $V$, entonces $\sum\limits_{i=1}^{k} W_i$ es un \Fsev{F} de $V$.
	\item Demostrar que $\fdefc{\text{Coord}_k}{V}{F}{\alpha}{C_{\alpha_k}}$ es una relación de asignamiento único.
	\item Sea el $\Fsev{F}$ $\Mnc{2}{F}$, 
	
	$W_1 = \left\{\begin{bmatrix}
		x & -x \\
		y & z
	\end{bmatrix} \in \Mnc{2}{F} \mid x,y,z \in F \right\}$ y $W_2 = \left\{\begin{bmatrix}
		x & y \\
		-x & z
	\end{bmatrix} \in \Mnc{2}{F} \mid x,y,z \in F \right\}$.
	\begin{enumerate}[label=\alph*), leftmargin=2em]
		\item Demostrar que $W_1$ y $W_2$ son \Fsevs{F} de $V$.
		\item Demostrar que $\dim(W_1 + W_2) = 4$ y $\dim(W_1 \cap W_2) = 2$.
	\end{enumerate}
	\item Considere las siguientes secuencias en $\mathbb{R}^2$, $\base{_1} = \{\alpha_1, \alpha_2\}$ y $\base{_2} = \{\beta_1, \beta_2\}$ con $\alpha_1 = (1,2), \alpha_2 = (-2, 1), \beta_1 = (2,3), \beta_3 = (1, -1)$.
	\begin{enumerate}[label=\alph*), leftmargin=2em]
		\item Demostrar que $\base{_1}$ y $\base{_2}$ son bases ordenadas de $\mathbb{R}^2$.
		\item Calcule $\left[\begin{array}{c}
			\gamma_1 \\
			\gamma_2
		\end{array}\right]_{\base{_1}}$ para todo $(\gamma_1, \gamma_2) \in \mathbb{R}^2$.
		\item Calcule $\left[\begin{array}{c}
			\delta_1 \\
			\delta_2
		\end{array}\right]_{\base{_2}}$ para todo $(\delta_1, \delta_2) \in \mathbb{R}^2$.
		\item Sea $P = \begin{bmatrix}
			p_{11} & p_{12} \\
			p_{21} & p_{22} \\
		\end{bmatrix} \in \Mnc{n}{\mathbb{R}}$ tal que $P \begin{bmatrix}
			\gamma_1 \\
			\gamma_2 
		\end{bmatrix}_{\base{_2}} = \begin{bmatrix}
			\gamma_1 \\
			\gamma_2
		\end{bmatrix}_{\base{_1}}$. Calcule $P$ (A $P$ se le llama \textbf{Matriz cambio de base}).
		\item Sea $Q = \begin{bmatrix}
			q_{11} & q_{12} \\
			q_{21} & q_{22} \\
		\end{bmatrix} \in \Mnc{n}{\mathbb{R}}$ tal que $\begin{bmatrix}
			\delta_1 \\
			\delta_2 
		\end{bmatrix}_{\base{_2}} Q = \begin{bmatrix}
			\delta_1 \\
			\delta_2
		\end{bmatrix}_{\base{_1}}$. Calcule $Q$.
		\item Verifique que $P$ y $Q$ son matrices invertibles.
		\item Sea $R = \begin{bmatrix}
			1 & 2 \\
			3 & 4
		\end{bmatrix}$ y la base ordenada $\base{_3} = \{\epsilon_1, \epsilon_2\}$ de $\mathbb{R}^2$, tal que $\begin{bmatrix}
			\gamma_1 \\
			\gamma_2
		\end{bmatrix}_{\base{_1}} = R \begin{bmatrix}
			\gamma_1 \\
			\gamma_2
		\end{bmatrix}_{\base{_3}}$. Determine $\epsilon_1$ y $\epsilon_2$.
	\end{enumerate}
	
	\item Considere el \Fev{\conmodr{3}} $V:=(\conmodr{3})^n$. ¿Cuántos subespacios vectoriales de dimensión uno tiene $V$?
\end{enumerate}

\newpage
\thispagestyle{empty}