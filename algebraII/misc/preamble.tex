\usepackage{multicol}
\usepackage{tikz}

%---------------------------------------------------------------
% Símbolos y notaciones. ---------------------------------------
%---------------------------------------------------------------
\usepackage[backend=biber, style=alphabetic, maxbibnames=4]{biblatex}
\addbibresource{biblo.bib}
%---------------------------------------------------------------

%%%%%%%%%%%%%%%5%%%%%%%%%%%%%%%5%%%%%%%%%%%%%%%5%%%%%%%%%%%%%%%5%%%%%%%%%%%%%%%5%%%%%%%%%%%%%%%5%%%%%%%%%%%%%%%5%%%%%%%%%%%%%%%5%%%%%%%%%%%%%%%5%%%%%%%%%%%%%%%5%%%%%%%%%%%%%%%5
%%%%%%%%%%%%%%%5%%%%%%%%%%%%%%%5%%%%%%%%%%%%%%%5%%%%%%%%%%%%%%%5%%%%%%%%%%%%%%%5%%%%%%%%%%%%%%%5%%%%%%%%%%%%%%%5%%%%%%%%%%%%%%%5%%%%%%%%%%%%%%%5%%%%%%%%%%%%%%%5%%%%%%%%%%%%%%%5%%%%%%%%%%%%%%%5
%%%%%%%%%%%%%%%5%%%%%%%%%%%%%%%5%%%%%%%%%%%%%%%5%%%%%%%%%%%%%%%5%%%%%%%%%%%%%%%5%%%%%%%%%%%%%%%5%%%%%%%%%%%%%%%5%%%%%%%%%%%%%%%5%%%%%%%%%%%%%%%5%%%%%%%%%%%%%%%5%%%%%%%%%%%%%%%5%%%%%%%%%%%%
%%%%%%%%%%%%%%%5%%%%%%%%%%%%%%%5%%%%%%%%%%%%%%%5%%%%%%%%%%%%%%%5%%%%%%%%%%%%%%%5%%%%%%%%%%%%%%%5%%%%%%%%%%%%%%%5%%%%%%%%%%%%%%%5%%%%%%%%%%%%%%%5%%%%%%%%%%%%%%%5%%%%%%%%%%%%%%%5v
%%%%%%%%%%%%%%%5%%%%%%%%%%%%%%%5%%%%%%%%%%%%%%%5%%%%%%%%%%%%%%%5%%%%%%%%%%%%%%%5%%%%%%%%%%%%%%%5%%%%%%%%%%%%%%%5%%%%%%%%%%%%%%%5%%%%%%%%%%%%%%%5%%%%%%%%%%%%%%%5%%%%%%%%%%%%%%%5

%\usepackage[utf8]{inputenc}
\usepackage[spanish, es-lcroman]{babel}
\usepackage[
    paperwidth=7in,
    paperheight=10in,
    inner=25mm,
    outer=20mm,
    top=0.8in,
    bottom=0.8in,
    %headsep=0.2in,
    headheight=14pt,
]{geometry}

%---------------------------------------------------------------
% Símbolos y notaciones. ---------------------------------------
%---------------------------------------------------------------
\usepackage{amsmath, amssymb, amsthm}
\usepackage{mathrsfs}
\usepackage{systeme}
\usepackage{etoolbox}
\usepackage{mathtools}
\usepackage{faktor}
\usepackage{xfrac}
%---------------------------------------------------------------
\newcommand{\cond}{%
	\mathbin{\raisebox{0pt}{\makebox[1.5em]{%
			\setlength{\unitlength}{0.5em}%
			\linethickness{0.3pt}%
			\begin{picture}(-0.1,0.1)
				\polygon*(0.5,0.9)(0.5,0.1)(1,0.5)
				\polygon*(0.5,0.560)(0.5,0.440)(-1.25,0.440)(-1.25,0.560)
			\end{picture}%
	}}}%
}
\newcommand{\bicond}{%
	\mathbin{\raisebox{0pt}{\makebox[1.5em]{%
			\setlength{\unitlength}{0.5em}%
			\linethickness{0.3pt}%
			\begin{picture}(-0.1,0.1)
				\polygon*(0.85,0.9)(0.85,0.1)(1.35,0.5)
				\polygon*(-1.35,0.5)(-0.85,0.1)(-0.85,0.9)
				\polygon*(-0.85,0.560)(-0.85,0.440)(0.85,0.440)(0.85,0.560)
			\end{picture}%
	}}}%
}

\newcommand{\iarr}{%
	\raisebox{0pt}{\makebox[0.5em]{%
				\setlength{\unitlength}{0.5em}%
				\linethickness{0.6pt}%
				\begin{picture}(-0.1,0.1)
					\polyline(0.2,-0.3)(-0.2,0.5)(0.2,1.3)
				\end{picture}%
	}}%
}

\newcommand{\darr}{%
	\raisebox{0pt}{\makebox[0.4em]{%
			\setlength{\unitlength}{0.5em}%
			\linethickness{0.6pt}%
			\begin{picture}(-0.1,0.1)
				\polyline(-0.2,-0.3)(0.2,0.5)(-0.2,1.3)
			\end{picture}%
	}}%
}

\newcommand{\cad}[1]{\iarr #1 \darr}

\newcommand{\fbf}{\textsl{f.b.f.}\;}
\newcommand{\fbfs}{\textsl{f.b.f.s}\;}
\newcommand{\raz}{\vdash}

\newcommand{\uclass}{{\mathcal{V}}}
\newcommand{\nat}{\mathbb{N}}
%\newcommand{\nat}{{\boldsymbol{\omega}}}
\newcommand{\intg}{\mathbb{Z}}
\newcommand{\rat}{\mathbb{Q}}
\newcommand{\real}{\mathbb{R}}
\newcommand{\complex}{\mathbb{C}}

\newcommand{\fdef}[3]{#1 : #2 \longrightarrow #3}
\newcommand{\suc}[1]{#1^{+}}
\newcommand{\Sucf}{\mathrm{S}}
\newcommand{\sucf}[1]{\mathrm{S}(#1)}

\newcommand{\biota}{\boldsymbol{\iota}}

\newcommand{\id}{\mathrm{Id}}

\newcommand{\fdefc}[5]{
	\raisebox{-5pt}{
		\(
		\begin{array}{@{}l@{\,}l@{}}
			#1 : & #2 \longrightarrow #3 \\[-1pt]
			& \scalebox{0.75}{$#4 \mapsto #5$}
		\end{array}
		\)
	}
}
\newcommand{\idfunc}[2]{\fdefc{\id}{#1}{#1}{#2}{#2}}

\newcommand{\difsim}{%
	\raisebox{0.2pt}{\makebox[0.77778em]{%
			\setlength{\unitlength}{0.5em}%
			\linethickness{0.3pt}%
			\begin{picture}(-0.1,0.1)
				\polygon(-0.5,0)(0.5,0)(0,0.8660)
			\end{picture}%
	    }
    }%
}

\newcommand{\equip}{\sim}
\newcommand{\isom}{\cong}
\makeatletter
\patchcmd{\endproof}
  {\popQED}
  {\vspace{-0.5cm}\par\pushQED{\qed}\popQED}
  {}{}
\makeatother
\renewcommand{\qedsymbol}{Q.E.D.}

\newcommand{\pp}{\mathrm{p}}
\newcommand{\pP}{\mathrm{P}}
\newcommand{\pps}{\mathrm{p}}
\newcommand{\pPs}{\boldsymbol{\mathrm{P}}}
\newcommand{\pqs}{\mathrm{q}}
\newcommand{\pQs}{\boldsymbol{\mathrm{Q}}}
\newcommand{\prs}{\mathrm{r}}
\newcommand{\pRs}{\boldsymbol{\mathrm{R}}}

\newcommand{\asign}[2]{#1 \mapsto #2} % a → f(a)
\newcommand{\tl}[4]{\underset{#4 \mapsto #1(#4)}{#1 : #2 \longrightarrow #3}} % T : V → W y \alpha → T(\alpha)
%\newcommand{\tl}[4]{\fdefc{#1}{#2}{#3}{#4}{#1(#4)}} % T : V → W y \alpha → T(\alpha)
\newcommand{\tlcomb}[4]{\underset{#4 \mapsto [#1](#4)}{#1 : #2 \longrightarrow #3}} % Combinación de transformaciones lineales

% Macros para conjuntos:
\newcommand{\Mnc}[2]{\mathcal{M}_{#1} (#2)} % Conjunto de matrices n×n
\newcommand{\Mmnc}[3]{\mathcal{M}_{#1 \times #2} \left( #3 \right)} % Conjunto de matrices m×n
\newcommand{\Legen}[1]{\mathscr{L}(#1)} % Espacio generado
%\newcommand{\conmodr}[1]{\faktor{\intg}{\sim3}} % Conjunto de clases de equivalencia módulo n
\newcommand{\gconmodr}[2]{%
	{{\displaystyle#1}}\big/{{\displaystyle #2}}
}
\newcommand{\conmodr}[1]{%
	\gconmodr{\intg}{{\sim}_{#1}}
}
\newcommand{\clamodr}[2]{[\,#1\,]_{#2}} % Clase de equivalencia módulo R
\newcommand{\dual}[1]{#1^*} % Espacio\Base dual
\newcommand{\inte}[2]{[\![#1,#2]\!]} % Intervalo entero [[1,m]]
\newcommand{\cv}[2]{ \{#1_1, \ldots, #1_#2\}} % Conjunto de vectores {x1, ..., xn}

% Macros para abreviaciones:
\newcommand{\li}{\textit{l.i.}\;} % Linealmente independiente
\newcommand{\ld}{\textit{l.d.}\;} % Linealmente dependiente
\newcommand{\Fev}[1]{\text{$#1$-espacio vectorial}} % F-espacio vectorial
\newcommand{\Fsev}[1]{\text{$#1$-subespacio vectorial}} % F-subespacio vectorial
\newcommand{\Fevs}[1]{\text{$#1$-espacios vectoriales}} % F-espacios vectoriales
\newcommand{\Fsevs}[1]{\text{$#1$-subespacios vectoriales}} % F-subespacios vectoriales
\newcommand{\base}[1]{\text{$\mathscr{B}#1$}} % Base de un F-e.v.
\newcommand{\tlc}{\textit{T.l.}\;} % Transformación lineal

% Macros para notaciones
\newcommand{\ran}[1]{\text{$\mathrm{rango}(#1)$}} % Rango de una transformación lineal
\newcommand{\nul}[1]{\text{$\mathrm{nulidad}(#1)$}} % Nulidad de una transformación lineal
\newcommand{\sgn}[1]{\text{$\mathrm{sgn}$}(#1)} % Función signo
\newcommand{\card}[1]{\text{$\mathrm{card}$}(#1)} % Cardinalidad de un conjunto

% Macros para sistemas de ecuaciones lineales y matrices
\newenvironment{matrizau}[1]{ % Entorno matriz aumentada
	\left[\begin{array}{@{}*{#1}{c}|c@{}}
	}{%
	\end{array}\right]
}

\newenvironment{sde}[1]{ % Entorno sistema de ecuaciones
	\left\{ \begin{array}{@{}*{#1}{c} c @{}}
	}{
	\end{array} \right.
}
\newcommand{\arrmatriz}[2]{\begin{align*} % Notación matriz
		#1 =
		\left[
		\begin{array}{c c c}
			#2_{11} & \cdots & #2_{1n} \\
			\vdots & \ddots & \vdots \\
			#2_{m1} & \cdots & #2_{mn}
		\end{array}
		\right]
\end{align*}}
\newcommand{\arrmatrizau}[3]{\begin{align*} % Notación matriz aumentada
		#1 =
		\left[
		\begin{array}{c c c | c}
			#2_{11} & \cdots & #2_{1n} & #3_1\\
			\vdots & \ddots & \vdots & \vdots\\
			#2_{m1} & \cdots & #2_{mn} & #3_m
		\end{array}
		\right]
\end{align*}}
\newcommand{\sisec}[1]{ % Notación sistema de ecuaciones lineales
	\begin{align*}
		\begin{array}{cc ccc c @{\extracolsep{2.5pt}}c@{\extracolsep{2.5pt}}c}
			a_{11}#1_{1} & + & \cdots & + & a_{1n}#1_{n} & = & y_{1} \\
			a_{21}#1_{1} & + & \cdots & + & a_{2n}#1_{n} & = & y_{2} \\
			& & \vdots & & & & \\
			a_{m1}#1_{1} & + & \cdots & + & a_{mn}#1_{n} & = & y_{m} \\
		\end{array}
	\end{align*}
}
\newcommand{\sisecn}[2]{ % Notación sistema de ecuaciones lineales nombrado
	\begin{align*}
		#1 =
		\left\{
		\begin{array}{cc ccc c @{\extracolsep{2.5pt}}c@{\extracolsep{2.5pt}}c}
			a_{11}#2_{1} & + & \cdots & + & a_{1n}#2_{n} & = & y_{1} \\
			& & \vdots & & & & \\
			a_{m1}#2_{1} & + & \cdots & + & a_{mn}#2_{n} & = & y_{m} \\
		\end{array}
		\right.
	\end{align*}
}

\newcommand{\mfL}{\mathcal{L}}
\newcommand{\mfT}{\mathfrak{T}}
\newcommand{\mfC}{\mathfrak{C}}
\newcommand{\mfV}{\mathfrak{V}}
\newcommand{\mfF}{\mathfrak{F}}
\newcommand{\mfR}{\mathfrak{R}}
\newcommand{\mfN}{\mathfrak{N}}
\newcommand{\mfQ}{\mathfrak{Q}}

\newcommand{\Clf}{C_{\mfL}}
\newcommand{\Vlf}{V_{\mfL}}
\newcommand{\Flf}{F_{\mfL}}
\newcommand{\Rlf}{R_{\mfL}}
\newcommand{\Slf}{S_{\mfL}}
\newcommand{\Qlf}{Q_{\mfL}}

\newcommand{\gen}{\bigwedge}

\newcommand{\lpo}{\mathcal{L}_{1}}
\newcommand{\lp}{\mathcal{L}_{\text{prop}}}

\newcommand{\Clfpo}{C_{\lpo}}
\newcommand{\Vlfpo}{V_{\lpo}}
\newcommand{\Flfpo}{F_{\lpo}}
\newcommand{\Rlfpo}{R_{\lpo}}
\newcommand{\Slfpo}{S_{\lpo}}
\newcommand{\Qlfpo}{Q_{\lpo}}

\newcommand{\cdA}{\mathscr{A}}
\newcommand{\cdB}{\mathscr{B}}
\newcommand{\cdC}{\mathscr{C}}
\newcommand{\cdD}{\mathscr{D}}
\newcommand{\cdE}{\mathscr{E}}
\newcommand{\cdG}{\mathscr{G}}
\newcommand{\cdH}{\mathscr{H}}
\newcommand{\cdI}{\mathscr{I}}

\newcommand{\olc}{\overline{c}}
\newcommand{\olf}{\overline{f}}
\newcommand{\olt}{\overline{t}}
\newcommand{\olR}{\overline{R}}

\newcommand{\ls}{<}
\newcommand{\gs}{>}
\newcommand{\sfl}{\vDash}

\newcommand{\com}[1]{\text{``#1"}}

%---------------------------------------------------------------
% Fuentes. -----------------------------------------------------
%---------------------------------------------------------------
\usepackage{helvet}
\usepackage{times}
\usepackage{bookman}
\usepackage{times}
\usepackage{lmodern}
\usepackage{calligra}

\usepackage{xeCJK}
\setCJKmainfont{IPAMincho}
\setCJKsansfont{IPAGothic}
\setCJKmonofont{IPAGothic}

\newcommand{\jpvert}[1]{
	\begin{minipage}[t]{1em}
		\raggedright
		#1
	\end{minipage}
}

%---------------------------------------------------------------
% Esconder hipervínculos. --------------------------------------
%---------------------------------------------------------------
\usepackage[hidelinks]{hyperref}

%---------------------------------------------------------------
% Utilidades de texto
\usepackage{setspace}
\usepackage{enumerate}
\usepackage{enumitem}
\usepackage{varwidth}

%---------------------------------------------------------------
% Gráficos, tablas y portada. ----------------------------------
%---------------------------------------------------------------
\usepackage{tikz}
\usepackage{tikz-cd}
\usepackage{graphicx}

\usepackage{caption}
\usepackage{parskip}
\usepackage{ragged2e}
\usepackage{background}
\usepackage{pict2e}

\usetikzlibrary{fadings}
\usetikzlibrary{babel}
%---------------------------------------------------------------
\captionsetup[table]{name=Tabla, labelfont=sl, font=it}

%---------------------------------------------------------------
% Estilos, entornos y organización de secciones. ---------------
%---------------------------------------------------------------
\usepackage{subfiles}
\usepackage{fancyhdr}
\usepackage{titlesec}
%---------------------------------------------------------------
\titleformat{\chapter}[display]
{\fontsize{30}{0}\selectfont\centering\bfseries}
{\fontsize{40}{0}\selectfont\thechapter\vspace{-1cm}}
{1em}
{}

\titleformat{\section}
{\normalfont\Large\bfseries}
{{\normalfont\S} \hspace{-0.5cm}}
{1em}
{}

\titleformat{\subsection}
{\itshape\bfseries\large}
{\hspace{-0.5cm}}
{0.8em}
{}

\newtheoremstyle{axioma}
{10pt} % Space above
{10pt} % Space below
{} % Body font
{} % Indent amount
{\bfseries \color{black}} % Theorem head font
{:} % Punctuation after theorem head
{.5em} % Space after theorem head
{ } % Theorem head spec
\theoremstyle{axioma}
\newtheorem{axiom}{Axioma}

\newtheoremstyle{definicion}
{10pt} % Space above
{10pt} % Space below
{} % Body font
{} % Indent amount
{\itshape\bfseries \color{black}} % Theorem head font
{:} % Punctuation after theorem head
{.5em} % Space after theorem head
{ } % Theorem head spec
\theoremstyle{definicion}
\newtheorem*{dfn}{{Definición}}
\newtheorem*{obs}{{Observación}}
\newtheorem*{conven}{{Convención}}

\newtheoremstyle{them}
{3pt}{3pt}
{\itshape}{}
{\scshape \bfseries \color{black}}{:}
{.5em}{ }
\theoremstyle{them}
\newtheorem{teo}{{Teorema}}[chapter]
\newtheorem{prop}{Proposición}[chapter]
\newtheorem{col}{Corolario}[chapter]
\newtheorem{lema}{Lema}[chapter]
\newtheorem{ejem}{Ejemplo}[chapter]

\newtheoremstyle{mteo}
{3pt}{3pt}
{\itshape}{}
{\calligra \color{black}}{:}
{.5em}{ }
\theoremstyle{mteo}
\newtheorem{mprop}{Proposición}[chapter]

%---------------------------------------------------------------
% Colorsitos. --------------------------------------------------
%---------------------------------------------------------------
\usepackage[dvipsnames]{xcolor}
%---------------------------------------------------------------
\definecolor{mazul}{HTML}{212529}
\definecolor{sazul}{HTML}{6c757d}
\definecolor{hazul}{HTML}{ced4da}
\definecolor{sgazul}{HTML}{D3D3D3}
\definecolor{gwhite}{HTML}{D3D3D3}

%---------------------------------------------------------------
% Tabla de contenidos. -----------------------------------------
%---------------------------------------------------------------
\usepackage[nottoc]{tocbibind}
\iffalse
\addto\captionsenglish{
    \renewcommand{\contentsname}{\normalfont \color{blue} \calligra Tabla de contenidos}
}
\fi

%---------------------------------------------------------------
% Texto en cajas por si acaso. ---------------------------------
%---------------------------------------------------------------
\usepackage{tcolorbox}
\tcbuselibrary{theorems}
\tcbuselibrary{breakable}
\tcbuselibrary{skins}
