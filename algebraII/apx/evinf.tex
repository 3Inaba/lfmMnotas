\chapter{Espacios vectoriales de dimensión infinita}
%\chapter{Infinite dimensional vector spaces}
El estudiante interesado puede encontrar una sección dedicada a espacios vectoriales de dimensión infinita en \cite{algebrajacobson}

Las siguientes definiciones y resultados son utilizados en su mayoría de manera implícita en la sección 2.2.

\section{El axioma de elección}

\begin{dfn}[Función selectora]
	Sea $A$ un conjunto. Una \textbf{función selectora o de elección} para $A$ es una función $\fdef{f}{\mathcal{P}(A) \setminus \{\varnothing\}}{A}$ tal que, para todo $$B \in \mathcal{P}(A)\setminus \{\varnothing\}, \;\; f(B) = f_B \in B.$$
\end{dfn}

Con esto es momento de interpretar A.7:

\begin{axiom}[de Elección]
	Todo conjunto no vacío tiene una función selectora.
\end{axiom}

Aceptar el axioma de Elección en nuestra teoría equivale a aceptar el Teorema de Zermelo (del Buen orden) así como el Lema de Zorn y otros tantos resultados. Entre ellos, está la equivalencia con la existencia de las bases de cualquier \Fev{F}.

Implícitamente se utiliza esto para poder elegir un vector $\alpha \in V \setminus \{0_V\}$, al igual que en muchos otros teoremas se elige un elemento arbitrario del conjunto sin importar si este es un conjunto finito o infinito. El axioma de elección evita el problema de definir exactamente qué significa \com{tomar} una infinidad (o de una) de elementos, garantizando la existencia de una función de elección para cualquier conjunto.

\section{Cardinales}
\subsection{Cardinalidad en conjuntos infinitos}
\begin{dfn}
	La cardinalidad de $A$ es menor o igual a la cardinalidad de $B$, si existe una función inyectiva $\fdefc{f}{A}{B}{a}{f(a)}$.
\end{dfn}

\begin{teo} \label{teo:NumerabilidadDeProductoCartesiano}
	Si $A$ y $B$ son conjuntos numerables, entonces $A \times B$ es numerable.
	\begin{proof}
		\cite{settheoryhernandez}. Una prueba utilizando el teorema de Cantor-Bernstein puede encontrarse en libros de Cálculo IV u de otras maneras en \cite{settheoryraymond}.
	\end{proof}
\end{teo}

\begin{col} \label{col:NumerabilidadDelCartesianoNumberableDeNumerables}
	El producto cartesiano de una cantidad finita de conjuntos numerables es numerable. Consecuentemente, $\mathbb{N}^m$ es numerable para todo $m \in \mathbb{N}$.
	\begin{proof}
		Ejercicio.
	\end{proof}
\end{col}

\begin{col}
	El conjunto de los números enteros $\mathbb{Z}$ y el conjunto de los números racionales $\mathbb{Q}$ son numerables.
	\begin{proof}
		\cite{settheoryhernandez}. Sería buena idea ir preparándose para álgebra IV a la vez que ve la demostración de este corolario.
	\end{proof}
\end{col}

\begin{teo}
	El conjunto de los números reales $\mathbb{R}$ es un conjunto no numerable.
	\begin{proof}
		\cite{settheoryhernandez}. O, si lo prefiere, de una vez prepárase y vea en conjunto topología de $\real^n$ en \cite{kolmogorov}.
	\end{proof}
\end{teo}

\begin{obs}
	La relación de orden en los números cardinales $\leq$ es un orden parcial.\\
\end{obs}

\begin{lema}
	Si $A_1 \subseteq B \subseteq A$ y $\card{A_1} = \card{A}$, entonces $\card{B} = \card{A}$.
	\begin{proof}
		\cite{settheoryhernandez}
	\end{proof}
\end{lema}

\subsection{Aritmética de cardinales}
Las siguientes definiciones son utilizadas de manera implícita en la demostración del caso infinito para la cardinalidad de las bases de un \Fev{F}.

\begin{dfn}
	Si $\card{A} = \kappa, \card{B} = \lambda$ y $A \cap B = \varnothing$, entonces
	$$ \kappa + \lambda = \card{A \cup B} .$$
\end{dfn}

La suma de cardinales no depende de la elección de los conjuntos $A$ y $B$. El axioma de elección implica que si $\kappa$ es infinito, entonces $\kappa + \kappa = \kappa$.

\begin{dfn}
	Si $\card{A} = \kappa$ y $\card{B} = \lambda$, entonces $$\kappa \mathrel{\cdot} \lambda = \card{A \times B}.$$
\end{dfn}

\section{Un último teorema}

\begin{teo}[Invarianza en la cardinalidad de las bases de un F-e.v.]
	Toda base de un $F$-espacio vectorial tiene la misma cardinalidad.
	
	\begin{proof}(Caso infinito)
		Sean $\mathscr{A}$ y $\mathscr{B}$ bases de un $F$-espacio vectorial $V$ ordenadas por conjuntos de índices bien ordenados $I$ y $J$ respectivamente, con $\card{\mathscr{A}} \geq \aleph_0$ e $i \in I$  y $j \in J$. Demostrado el caso finito, entonces tenemos que $\card{\mathscr{B}} \geq \aleph_0$\\
		
		Cada vector $\alpha_i \in \mathscr{A}$ es representado de forma única como una combinación lineal finita de vectores $\beta_j \in \mathscr{B}$. En efecto, sean $J_{i1} \subseteq J$ y $J_{i2} \subseteq J$ conjuntos finitos. Suponga que existen $\{c_k\}_{k \in J_{i1}} \subseteq F$ y $\{d_k\}_{k \in J_{i2}} \subseteq F$ y se tiene que
		
		\begin{equation*}
			\alpha_i = \sum\limits_{k \in J_{i1}} c_k \beta_k \text{ y } \alpha_i = \sum\limits_{k \in J_{i2}} d_k \beta_k
		\end{equation*}
		
		Entonces
		
		\begin{align*}
			\sum\limits_{k \in J_{i1}} c_k \beta_k = \sum\limits_{k \in J_{i2}} d_k \beta_k \\
			\sum\limits_{k \in J_{i1}} c_k \beta_k - \sum\limits_{k \in J_{i2}} d_k \beta_k = 0
		\end{align*}
		
		Luego, separando las sumas en índices en común e índices no en común, se tiene
		
		\begin{equation*}
			\sum\limits_{k \in J_{i1} \cap J_{i2}} (c_k - d_k) \beta_k + \sum\limits_{k \in J_{i1} \setminus J_{i2}} c_k \beta_k - \sum\limits_{k \in J_{i2} \setminus J_{i1}} d_k \beta_k = 0
		\end{equation*}
		
		$\mathscr{B}$ es base, entonces $\forall c_k, k \in J_{i1} \setminus J_{i2}$ se tiene que $c_k = 0$ y del mismo modo $\forall d_k, k \in J_{i2} \setminus J_{i1}$ se tiene que $d_k = 0$. Luego, $\forall k \in J_{i1} \cap J_{i2}$, $c_k - d_k = 0$. Luego entonces $c_k = d_k$. Así, tenemos que la representación es única. \\
		
		Ahora, cada $\beta_j \in \mathscr{B}$ aparece en una combinación lineal finita de un $\alpha_i \in \mathscr{A}$. Si no fuera así, entonces suponga que $\beta_{j_0}$ no aparece en ninguna combinación lineal finita para todo $\alpha_i \in \mathscr{A}$. Luego, sea $I_{j_0} \subseteq I$ conjunto finito y sea $\{e_l\}_{l \in I_{j_0}} \subseteq F$, como $\mathscr{A}$ es base, entonces
		
		\begin{align*}
			\beta_{j_0} = \sum\limits_{l \in I_{j_0}} e_l \alpha_l
		\end{align*}
		
		Cada $\alpha_l, l \in I_{j_0}$ se representa como una combinación lineal finita de vectores en $\mathscr{B}$, sea $J_\alpha \subseteq J$ conjunto finito y $\{c_k\}_{k \in J_{\alpha}}$, supongamos entonces
		
		\begin{equation*}
			\beta_{j_0} = \sum\limits_{k \in J_{\alpha}} c_k \beta_k
		\end{equation*}
		
		luego entonces, $\mathscr{B}$ es linealmente dependiente, lo que contradice que $\mathscr{B}$ sea una base de $V$. Por lo tanto, todo vector $\beta_j \in \mathscr{B}$ aparece en alguna combinación lineal que representa a los $\alpha_i \in \mathscr{A}$. \\
		
		Entonces se define $\fdefc{f}{\mathscr{B}}{\mathscr{A}}{\beta_j}{\alpha_i}$ que a un $\beta_j \in \mathscr{B}$ le asigna sólo un $\alpha_i \in \mathscr{A}$ donde $\beta_j$ forma parte de la combinación lineal finita única que representa a $\alpha_i$. Entonces $\fdefc{f}{\mathscr{B}}{\mathscr{A}}{\beta_j}{\alpha_i}$ es una función.\\
		
		Luego, sea $J_i \subseteq J$ conjunto finito y $\alpha_i' \in f(\mathscr{B})$, entonces $f^{-1}(\{\alpha_i'\}) = \{\beta_j\}_{j \in J_i}$, $J_i \subset J$ es un conjunto tal que $\beta_j$ forma parte de la combinación lineal finita única que representa a $\alpha_i$, entonces $f^{-1}(\{\alpha_i'\})$ es un conjunto finito.\\
		
		Sea $\Gamma := \{f^{-1}(\{\alpha_i'\}) \in \mathcal{P}(\mathscr{B}) \mid \alpha_i' \in f(\mathscr{B})\}$. Luego $\mathscr{B} = \bigcup\limits_{\gamma \in \Gamma} \gamma$, la cual, como $f$ es función, es una unión disjunta. Así $\card{\mathscr{B}} = \sum\limits_{\gamma \in \Gamma} \card{\gamma} \leq \card{f(\mathscr{B})}$ y como $f(\mathscr{B}) \subseteq \mathscr{A}$, $\card{f(\mathscr{B})} \leq \card{\mathscr{A}}$. Luego entonces $\card{\mathscr{B}} \leq \card{\mathscr{A}}$ y se tiene que existe $\fdef{\phi}{\mathscr{B}}{\mathscr{A}}$ función inyectiva. \\
		
		De manera análoga, intercambiando los papeles de $\mathscr{B}$ y $\mathscr{A}$, se llega a que $\card{\mathscr{A}} \leq \card{\mathscr{B}}$ y entonces existe una función inyectiva $\fdef{\psi}{\mathscr{A}}{\mathscr{B}}$. Luego, por el Teorema \ref{teo:CantorBernstein} se tiene que $\card{\mathscr{A}} = \card{\mathscr{B}}$.
	\end{proof}
\end{teo}

\section{Ejercicios}
\begin{enumerate}[label=\arabic*)]
	\item Demostrar el Teorema \ref{teo:NumerabilidadDeProductoCartesiano}.
	\item Demostrar el Corolario \ref{col:NumerabilidadDelCartesianoNumberableDeNumerables}.
	\item \begin{enumerate}[label=\alph*), leftmargin=2em]
		\item Demostrar que $\mathbb{R}^\mathbb{N}$ es un $\Fev{\mathbb{R}}$.
		\item Sea $e_n = (\delta_{nk})_{k \in \mathbb{N}} \in \mathbb{R}^\mathbb{N}$. ¿Es el conjunto $\mathscr{B} = \{e_n \in \mathbb{R}^\mathbb{N} \mid n \in \mathbb{N} \}$ una base de $\mathbb{R}^{\mathbb{N}}$?
	\end{enumerate}
	
	\item Sea $F$ un campo. Encontrar una base para $F[x]$ como \Fev{F}.
\end{enumerate}

%\include{dembases}