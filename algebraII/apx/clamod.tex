\chapter{Clases de equivalencia módulo $n$}
\section{Deducciones propias de las clases de equivalencia módulo $n$}

Sea $n \in \intg$. Definamos la relación $\sim_n$ por $$a \sim_n b \bicond n \mid a - b.$$ Esta relación es una relación de equivalencia. El conjunto cociente correspondiente se denota $\conmodr{n}$ y sus clases de equivalencia $\clamodr{x}{n}$.

\begin{dfn}[Operaciones en $\conmodr{n}$]
	\item La suma en $\conmodr{n}$ es la función $\fdefc{+}{\conmodr{n}\times\conmodr{n}}{\conmodr{n}}{(a,b)}{\clamodr{a+b}{n}}$.
	\item La multiplicación en $\conmodr{n}$ es la función $\fdefc{\cdot}{\conmodr{n}\times\conmodr{n}}{\conmodr{n}}{(a,b)}{\clamodr{a\cdot b}{n}}$.
\end{dfn}

Por sugerencia de un compañero intenté explicar algunos de los casos de interés utilizando el teorema del residuo. Note que si $n \mid a - b$ entonces $a = nk + b$ y por el teorema del residuo $a = nq + r$, luego es fácil ver que $r \sim_n b$.

\begin{enumerate}[label=\roman*), leftmargin=2em]
	\item Si $a > n$, por el teorema del residuo $a = nq + r$, $0 \leq r < n$, entonces si $\alpha \in \clamodr{a}{n}$ $\alpha = nk + a = nk + nq + r = n(k + q) + r$, entonces $\alpha \in \clamodr{r}{n}$. Luego, si $\beta \in \clamodr{r}{n}$ entonces $\beta = nq' + r$, como $r = a - nq$ $\beta = nq' + a - nq = n(q' - q) + a$, luego $\beta \in \clamodr{a}{n}$. Entonces 
	$$\clamodr{a}{n} = \clamodr{r}{n}.$$
	\item Si $a < 0$, por el teorema del residuo $a = nq + r$, $0 \leq r < n$, luego $r = a - nq$ y como $r \geq 0$ y $a < 0$ entonces $a < -nq$ y $-nq \geq 0$. Entonces si $\gamma \in \clamodr{a}{n}$ $\gamma = nk + a = nk + nq + r = n(k + q) + r = n(k + q) + (-nq + a)$, luego $\gamma \in \clamodr{-nq + a}{n}$ con $-nq + a \geq 0$. Así, podemos llegar a que 
	$$\clamodr{a}{n} = \clamodr{-nq + a}{n}$$
\end{enumerate}

\begin{ejem}
	\begin{enumerate}[label=\arabic*)]
		\item $\clamodr{2}{6}\clamodr{3}{6} = \clamodr{2\cdot3}{6} = \clamodr{6}{6} = \clamodr{0}{6}$.
		\item $\clamodr{4}{13} + \clamodr{12}{13} = \clamodr{4 + 12}{13} = \clamodr{16}{13} = \clamodr{3}{13}$.
		\item $\clamodr{-1}{5} + \clamodr{16}{5} = \clamodr{4}{5} + \clamodr{16}{5} = \clamodr{20}{5} = \clamodr{0}{5}$.
		\item $\clamodr{851}{17}\clamodr{5135}{17} + \clamodr{1000000}{17} = \clamodr{1}{17}\clamodr{1}{17} + \clamodr{1000000}{17} = \clamodr{1}{17} + \clamodr{1000000}{17} = \clamodr{1000001}{17} = \clamodr{10}{17}$
	\end{enumerate}
\end{ejem}

Note que $(\conmodr{n}, +)$ es un grupo abeliano y además $(\conmodr{n},+,\cdot)$ es un anillo. Cuando $n$ es primo tenemos entonces un campo finito. Si ahora consideramos las unidades del anillo, se tiene entonces que $(\mathcal{U}_{\conmodr{n}}, \cdot)$ es un grupo, llamado el grupo de unidades de $\conmodr{n}$.

\section{Ejercicios}
\begin{enumerate}[label=\arabic*)]
	\item Probar que la relación $\sim_n$ es de equivalencia.
	\item Pruebe que si $p$ es primo entonces $\conmodr{p}$ es un campo con cardinalidad finita.
	\item ¿Es $2^{19}-1$ primo?
	\item ¿12039809185092830197283782828282828282828282828282828282 tiene raíz $n$-ésima entera para todo $n \in \mathbb{N} \setminus \{0,1\}$?
\end{enumerate}
