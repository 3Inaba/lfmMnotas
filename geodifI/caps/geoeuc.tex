\section{Espacios vectoriales euclidianos}

Notación. Sean $e_1 := (1,0,0,\ldots, 0), e_2 := (0,1,0,\ldots,0), \ldots, e_n := (0,0,0,\ldots,1) \in \real^n$. Al conjunto $\bcanon := \{e_1, e_2, \ldots, e_n\}$ lo llamaremos {\bfseries base canónica} de $\real^n$.

\begin{dfn}
	Sea $V$ un $\real$-espacio vectorial de dimensión finita. $V$, dotado de un funcional lineal $\esc{\cdot,\cdot} : V \times V \longrightarrow \real$ definido por $(x,y) \mapsto \esc{x,y}$, se llamará un {\bfseries espacio vectorial euclidiano} si se cumple:
	\begin{enumerate}[label=(\alph*)]
		\item {(Bilinealidad)}: $\esc{\alpha_1 x_1 + \alpha_2x_2, y} = \alpha_1\esc{x_1,y} + \alpha_2\esc{x_2,y} \quad \forall x_1,x_2,y \in V$ y $\alpha_1,\alpha_2 \in \real$.
		
		$\esc{x, \beta_1 y_1 + \beta_2 y_2} = \beta_1\esc{x,y_1} + \beta_2\esc{x,y_2} \quad \forall y_1, y_2,x \in V$ y $\beta_1,\beta_2 \in \real$.
		\item {(Simetría)}:
		$\esc{x,y} = \esc{y,x} \quad \forall x,y \in V$.
		\item {(No negatividad)} $\esc{x,x} \geq 0 \quad x \in V$, y $\esc{x,x} = 0 \iff x = 0$.
	\end{enumerate}
\end{dfn}

\begin{lrbox}{\diagCanonico}
	\begin{tikzcd}
		S \times S \arrow[d, "\iota \times \iota"] \arrow[r, "\delta"] & \real \\
		\real^n \times \real^n \arrow[ur, "\esc{\cdot, \cdot}"', dashed] & 
	\end{tikzcd}
\end{lrbox}

El funcional simétrico y bilineal $\func{\esc{\cdot,\cdot}}{V \times V}{\real}$ se llama {\bfseries producto interno}. El producto interno canónico\footnote{
	El nombre canónico puede verse como consecuencia de ser la única forma bilineal que hace ortogonal a la base canónica de $\real^n$. Formalmente, si $S = \inte{1,n}$, $\func{\delta}{S \times S}{\real}$ es la delta de Kronecker y $\func{\iota}{S}{\real^n}$ es la inclusión de $S$ en $\real^n$ dada por $k \mapsto e_k$, entonces es conmutativo el diagrama
	\[ \usebox{\diagCanonico} \]
	$(k_1, k_2) \xrightarrow{\iota \times \iota} (e_{k_1}, e_{k_2}) \xrightarrow{\iprod{\cdot, \cdot}}$ $
	\left\{\begin{array}{c}
		1, \text{ si } k_1 = k_2 \\
		0, \text{ si } k_1 \not= k_2
	\end{array} \right.$
}
para el $\real$-espacio vectorial $\real^n$ está dado por la regla:

\[
	\forall x = (x_1, \ldots, x_n), z = (z_1 \ldots, z_n) \in \real^n \left( \esc{x,z} = \sum_{i=1}^n x_i z_i \right)
\]

Cada producto interno induce una función $\func{\|\cdot\|}{V}{\real}$ con $x \mapsto \esc{x, x}$ a la cual se le define por

\begin{dfn}
	La función $\func{\|\cdot\|}{\real^n}{[0,+\infty[}$ dada por $x \mapsto \iprod{x,x}$ se le llama {\bfseries norma}. Si $\iprod{\cdot, \cdot}$ es el producto interno canónico entonces a $\|\cdot\|$ le decimos {\bfseries norma euclidiana}.
\end{dfn}

\begin{prop}
	La norma inducida por un producto interno satisface las propiedades:
	\begin{enumerate}[label=\alph*.]
		\item $\|x\| \geq 0$, $\forall x \in \real^n$. $\|x\| = 0$ si y sólo si $x = 0$.
		\item $\|\lambda x\| = |\lambda|\|x\|$, $\forall x \in \real^n$, $\forall \lambda \in \real$.
		\item $\|x + z\| \leq \|x\| + \|z\|$, $\forall x,z \in \real^n$.
	\end{enumerate}
\end{prop}
\begin{proof}
	Ejercicio.
\end{proof}

\begin{thm}[Desigualdad de Cauchy-Schwarz]
	$\forall x, z \in \real^n$ se verifica $|\iprod{x,z}| \leq \|x\|\|z\|$.
\end{thm}
\begin{proof}
	Ejercicio.
\end{proof}

\begin{dfn}
	Si $x, z \in \real^n \setminus \{0_{\real^n}\}$, definimos el {\bfseries ángulo} $\theta \in [0,\pi]$ entre los vectores $x$ y $z$ como el número real que satisface
	\[ \cos(\theta) = \frac{\iprod{x, z}}{\|x\|\|z\|} \]
\end{dfn}

Observación. $\iprod{x, z} = \|x\|\|z\|\cos(\theta)$.

\begin{dfn}
	Si $x, z \in \real^n$, decimos que estos vectores son {\bfseries ortogonales} si $\iprod{x,z} = 0$.
\end{dfn}

\begin{dfn}
	Sean $x, z \in \real^n \setminus \{0_{\real^n}\}$. Definimos la {\bfseries proyección} del vector $x$ sobre el vector $z$, denotada por $x_z$ como
	\[ x_z := \frac{\iprod{x, z}}{\|z\|^2}z \]
\end{dfn}

\subsection{Orientación en $\real^n$.}
Sean $\{\alpha_1, \ldots, \alpha_n\}$ y $\{\beta_1, \ldots, \beta_n\}$ bases de $\real^n$. Existe $\{a_{ij}\}_{i,j=1}^n \subset \real$ tal que
	$$\alpha_i = \sum_{j=1}^n a_{ij}\beta_j.$$
Sea $A \in \mtxn{n}$ con $(A)_{ij} = a_{ij}$ y $\func{T_A}{\real^n}{\real^n}$ con $\alpha \mapsto A \alpha$. Supóngase que $\alpha = \sum_{k=1}^{n}c_k\alpha_k$. Note que $T_A(\alpha) = \sum_{k=1}^n d_k\beta_k$ para algunos $d_k \in \real$. Si se tiene que $T_A(\alpha) = 0_{\real^n}$, como $\{b_k\}_{k=1}^n$ es base, se tiene que $d_k = 0, \quad \forall k = 1, \ldots, n$. Por lo tanto $\alpha = 0$, luego $T_A$ es no singular. Aún más, $T_A$ es un isomorfismo, por lo que $A$ es invertible.
$A$ se dice ser la matriz de cambio de base de la base $\{\alpha_k\}_{k=1}^n$ a la base $\{\beta_k\}_{k=1}^n$.

Sea $\fkB$ el conjunto de todas las bases de $\real^n$. Sobre este conjunto, se define:
\begin{center}
	dadas $B_1, B_2 \in \fkB$ decimos que: $B_1 \sim B_2$ si y sólo si la matriz de cambio de base de la base $B_1$ a la base $B_2$ tiene determinante positivo.
\end{center}

Claramente la matriz cambio de base de $B_1$ a $B_1$ es $\id_{\mtxn{n}}$ por lo que $B_1 \sim B_2$. Como la matriz de cambio de base es invertible y $|A^{-1}| = 1/|A|$ entonces si $B_1 \sim B_2$ entonces $B_2 \sim B_1$. Si $C$ es la matriz cambio de base de $B_1$ a $B_2$ y $D$ es la matriz cambio de base de $B_2$ a $B_3$, se tiene que la matriz de cambio de base de $B_1$ a $B_3$ es $DC$. De este modo, si $B_1 \sim B_2$ y $B_2 \sim B_3$ entonces $|DC| = |D||C| > 0$. Luego $\sim$ define una relación de equivalencia.

Notamos que $\fkB/{\sim}$ es un conjunto con dos elementos. Si $B \in \fkB$ y $B \in [\{e_k\}_{k=1}^n]$ entonces diremos que el espacio $\real^n$ dotado con la base $B$ tiene orientación positiva. Si $B \not\in [\{e_k\}_{k=1}^n]$ entonces decimos que $\real^n$ dotado con la base $B$ tiene orientación negativa.

Ejercicio. Determina si la base $\{ (1,3,5), (2,3,7), (4,8,3) \}$ proporciona una orientación positiva a $\real^3$.

\begin{dfn}
	Sean $x = (x_1, x_2, x_3), z = (z_1, z_2, z_3) \in \real^3$. Definimos el producto vectorial (también llamado producto cruz) de $x$ y $z$ denotado por $x \times z$ como sigue\footnote{
	Simbólicamente se suele escribir
		\[
			x \times z =
			\begin{vmatrix}
				e_1 	& e_2 	& e_3	\\
				x_1	& x_2	& x_3	\\
				z_1	& z_2	& z_3
			\end{vmatrix}
		\]
}
	\[
		x \times z = 
			\begin{vmatrix} x_2 & x_3 \\ z_2 & z_3 \end{vmatrix}e_1 -
			\begin{vmatrix} x_1 & x_3 \\ z_1 & z_3 \end{vmatrix}e_2 +
			\begin{vmatrix} x_1 & x_2 \\ z_1 & z_2 \end{vmatrix}e_3
	\]
\end{dfn}


\begin{prop}
	El producto vectorial cumple las siguiente propiedades. Si $x, y, z\in \real^3$ entonces se tiene:
	\begin{enumerate}[label=\alph*.]
		\item $x \times z = - z \times x $
		\item $x \times (y + z) = x \times y + x \times z $
		\item $(\lambda x) \times z = \lambda (x \times z)$
		\item $x \times (y \times z) = \esc{x,z}y - \esc{x, y}z $
		\item $\esc{x \times y, z} =
			\begin{vmatrix}
			x_1 & x_2 & x_3 \\
			y_1 & y_2 & y_3 \\
			z_1 & z_2 & z_3 
			\end{vmatrix}$
	\end{enumerate}
\end{prop}

\begin{proof}
	Ejercicio.
\end{proof}

Ejercicio. Demuestre que $\iprod{x \times y, z \times w } = \iprod{x, z}\iprod{y, w} - \iprod{x, w}\iprod{y, z}$.

Ejercicio. Considere el paralelogramo $\mathcal{P}:$
\[
	\begin{tikzpicture}[scale=1.5]
		\draw[-{Stealth}, thick] (0,0) -- (0.5,1) node[midway, left]{$y$};
		\draw[-{Stealth}, thick] (0,0) -- (2,0) node[midway, below]{$x$};
		\draw[thick] (2,0) -- (2.5,1); 
		\draw[thick] (0.5,1) -- (2.5,1); 
	\end{tikzpicture}
\]
Calcular el área de $\mathcal{P}$ en función de $x \times y$.
