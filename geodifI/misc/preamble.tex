\usepackage{multicol}
\usepackage{tikz}

%---------------------------------------------------------------
% Símbolos y notaciones. ---------------------------------------
%---------------------------------------------------------------
\usepackage[backend=biber, style=alphabetic, maxbibnames=4]{biblatex}
\addbibresource{biblo.bib}
%---------------------------------------------------------------

%%%%%%%%%%%%%%%5%%%%%%%%%%%%%%%5%%%%%%%%%%%%%%%5%%%%%%%%%%%%%%%5%%%%%%%%%%%%%%%5%%%%%%%%%%%%%%%5%%%%%%%%%%%%%%%5%%%%%%%%%%%%%%%5%%%%%%%%%%%%%%%5%%%%%%%%%%%%%%%5%%%%%%%%%%%%%%%5
%%%%%%%%%%%%%%%5%%%%%%%%%%%%%%%5%%%%%%%%%%%%%%%5%%%%%%%%%%%%%%%5%%%%%%%%%%%%%%%5%%%%%%%%%%%%%%%5%%%%%%%%%%%%%%%5%%%%%%%%%%%%%%%5%%%%%%%%%%%%%%%5%%%%%%%%%%%%%%%5%%%%%%%%%%%%%%%5%%%%%%%%%%%%%%%5
%%%%%%%%%%%%%%%5%%%%%%%%%%%%%%%5%%%%%%%%%%%%%%%5%%%%%%%%%%%%%%%5%%%%%%%%%%%%%%%5%%%%%%%%%%%%%%%5%%%%%%%%%%%%%%%5%%%%%%%%%%%%%%%5%%%%%%%%%%%%%%%5%%%%%%%%%%%%%%%5%%%%%%%%%%%%%%%5%%%%%%%%%%%%
%%%%%%%%%%%%%%%5%%%%%%%%%%%%%%%5%%%%%%%%%%%%%%%5%%%%%%%%%%%%%%%5%%%%%%%%%%%%%%%5%%%%%%%%%%%%%%%5%%%%%%%%%%%%%%%5%%%%%%%%%%%%%%%5%%%%%%%%%%%%%%%5%%%%%%%%%%%%%%%5%%%%%%%%%%%%%%%5v
%%%%%%%%%%%%%%%5%%%%%%%%%%%%%%%5%%%%%%%%%%%%%%%5%%%%%%%%%%%%%%%5%%%%%%%%%%%%%%%5%%%%%%%%%%%%%%%5%%%%%%%%%%%%%%%5%%%%%%%%%%%%%%%5%%%%%%%%%%%%%%%5%%%%%%%%%%%%%%%5%%%%%%%%%%%%%%%5

%\usepackage[utf8]{inputenc}
\usepackage[spanish]{babel}
\usepackage[
    paperwidth=17cm,
    paperheight=24cm,
    inner=15mm,,
    outer=10mm,
    top=15mm,
    bottom=15mm,
    %headsep=0.2in,
    headheight=14pt,
]{geometry}

%---------------------------------------------------------------
% Símbolos y notaciones. ---------------------------------------
%---------------------------------------------------------------
\usepackage{amsmath, amssymb, amsthm}
\usepackage{mathrsfs}
\usepackage{systeme}
\usepackage{etoolbox}
\usepackage{mathtools}
\usepackage{faktor}
\usepackage{xfrac}
%---------------------------------------------------------------
\newcommand{\com}[1]{\text{``#1"}}
\newcommand{\esc}[1]{\langle #1 \rangle}
\newcommand{\iprod}[1]{\langle #1 \rangle}

\newcommand{\real}{\mathbb{R}}

\newcommand{\mtxn}[1]{\mathbb{M}_{n}}
\newcommand{\id}{\mathrm{id}}
\newcommand{\func}[3]{#1: #2 \longrightarrow #3}
\newcommand{\rfunc}[5]{
	\raisebox{-5pt}{
		\(
		\begin{array}{@{}l@{\,}l@{}}
			#1 : & #2 \longrightarrow #3 \\[-1pt]
			& \scalebox{0.75}{$#4 \mapsto #5$}
		\end{array}
		\)
	}
}

\newcommand{\fkB}{\mathfrak{B}}
\newcommand{\inte}[1]{[\![#1]\!]} % Intervalo entero [[1,m]]

\newcommand{\bcanon}{\mathcal{C}}

%---------------------------------------------------------------
% Fuentes. -----------------------------------------------------
%---------------------------------------------------------------
\usepackage{helvet}
\usepackage{times}
\usepackage{bookman}
\usepackage{times}
\usepackage{lmodern}
\usepackage{calligra}

\newcommand{\jpvert}[1]{
	\begin{minipage}[t]{1em}
		\raggedright
		#1
	\end{minipage}
}

%---------------------------------------------------------------
% Esconder hipervínculos. --------------------------------------
%---------------------------------------------------------------
\usepackage[hidelinks]{hyperref}

%---------------------------------------------------------------
% Utilidades de texto
\usepackage{setspace}
\usepackage{enumerate}
\usepackage{enumitem}
\usepackage{varwidth}

%---------------------------------------------------------------
% Gráficos, tablas y portada. ----------------------------------
%---------------------------------------------------------------
\usepackage{tikz}
\usepackage{tikz-cd}
\usetikzlibrary{matrix}
\usepackage{graphicx}

\usepackage{caption}
\usepackage{parskip}
\usepackage{ragged2e}
\usepackage{background}
\usepackage{pict2e}

\usepackage{pgfplots}

\usetikzlibrary{fadings}
\usetikzlibrary{babel}
%---------------------------------------------------------------
\captionsetup[table]{name=Tabla, labelfont=sl, font=it}

%---------------------------------------------------------------
% Estilos, entornos y organización de secciones. ---------------
%---------------------------------------------------------------
\usepackage{subfiles}
\usepackage{fancyhdr}
\usepackage{titlesec}
%---------------------------------------------------------------
\titleformat{\chapter}[display]
{\fontsize{30}{0}\selectfont\centering\bfseries\vspace{0cm}}
{\fontsize{40}{0}\selectfont\thechapter\vspace{-1cm}}
{1em}
{}

\titleformat{\section}
{\normalfont\Large\bfseries}
{\thesection}
{1em}
{}

\titleformat{\subsection}
{\itshape\bfseries\large}
{\hspace{-0.5cm}}
{0.8em}
{}

\newtheoremstyle{axioma}
{10pt} % Space above
{10pt} % Space below
{} % Body font
{} % Indent amount
{\bfseries \color{black}} % Theorem head font
{:} % Punctuation after theorem head
{.5em} % Space after theorem head
{ } % Theorem head spec
\theoremstyle{axioma}
\newtheorem{axiom}{Axiom}

\newtheoremstyle{definicion}
{10pt} % Space above
{10pt} % Space below
{} % Body font
{} % Indent amount
{\itshape\bfseries \color{black}} % Theorem head font
{.} % Punctuation after theorem head
{.5em} % Space after theorem head
{ } % Theorem head spec
\theoremstyle{definicion}
\newtheorem*{dfn}{{Definición}}
\newtheorem*{obs}{{Observación}}
\newtheorem*{conven}{{Convención}}

\newtheoremstyle{them}
{3pt}{3pt}
{\itshape}{}
{\scshape \bfseries \color{black}}{.}
{.5em}{ }
\theoremstyle{them}
\newtheorem{thm}{{Teorema}}[chapter]
\newtheorem{prop}{Proposición}[chapter]
\newtheorem{col}{Corolario}[chapter]
\newtheorem{lema}{Lema}[chapter]
\newtheorem{ejem}{Ejemplo}[chapter]

\newtheoremstyle{mteo}
{3pt}{3pt}
{\itshape}{}
{\calligra \color{black}}{.}
{.5em}{ }
\theoremstyle{mteo}
\newtheorem{mprop}{Proposición}[chapter]

%---------------------------------------------------------------
% Colorsitos. --------------------------------------------------
%---------------------------------------------------------------
\usepackage[dvipsnames]{xcolor}
%---------------------------------------------------------------
\definecolor{mazul}{HTML}{1F2A44}
\definecolor{sazul}{HTML}{4B6587}
\definecolor{hazul}{HTML}{AFCDE7}
\definecolor{sgazul}{HTML}{6C7A89}
\definecolor{gwhite}{HTML}{F1F6FA}

%---------------------------------------------------------------
% Tabla de contenidos. -----------------------------------------
%---------------------------------------------------------------
\usepackage[nottoc]{tocbibind}
\iffalse
\addto\captionsenglish{
    \renewcommand{\contentsname}{\normalfont \color{blue} \calligra Tabla de contenidos}
}
\fi

%---------------------------------------------------------------
% Texto en cajas por si acaso. ---------------------------------
%---------------------------------------------------------------
\usepackage{tcolorbox}
\tcbuselibrary{theorems}
\tcbuselibrary{breakable}
\tcbuselibrary{skins}
